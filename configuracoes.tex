\usepackage{CJKutf8}
\usepackage{indentfirst}
\usepackage[utf8]{inputenc}
\usepackage[a4paper, left=20mm, right = 8mm, top = 22mm, bottom = 35 mm]{geometry}
\usepackage{helvet}
\renewcommand\familydefault{\sfdefault}
%\usepackage[titletoc,toc,page]{appendix}
\usepackage{multicol,multirow}
\usepackage{tocloft}
\usepackage{enumitem}
\setcounter{tocdepth}{1}
\setcounter{secnumdepth}{4}
%\usepackage{sectsty}
\usepackage[colorlinks=true]{hyperref}
\usepackage[dvipsnames]{xcolor}
%setup new colors
\hypersetup{
	linkcolor=RawSienna
	,citecolor=ForestGreen
	,filecolor=Mulberry
	,urlcolor=NavyBlue
	,menucolor=BrickRed
	,runcolor=Mulberry
	,linkbordercolor=BrickRed
	,citebordercolor=Green
	,filebordercolor=Mulberry
	,urlbordercolor=NavyBlue
	,menubordercolor=BrickRed
	,runbordercolor=Mulberry
}

\usepackage{graphicx}
\usepackage[portuges]{babel}
\usepackage{lastpage}

\usepackage{tikz}
\usetikzlibrary{calc,spy}
\usetikzlibrary{decorations.pathmorphing}

\usepackage{draftwatermark}
%\SetWatermarkText{
%	\begin{tikzpicture}[remember picture, overlay]
%	\node[opacity=0.4,inner sep=0pt] at (current page.center)
%	{\includegraphics[width=200mm]{fig/Logo-Nikkei-Curitiba2a}};
%	\end{tikzpicture}
%	}

\SetWatermarkText{
	\begin{tikzpicture}[remember picture, overlay]
	\node[opacity=0.1,inner sep=0pt] at (current page.center)
	{\includegraphics[width=170mm, angle = -45]{fig/perfil}};
	\end{tikzpicture}
}

\let\printglossary\relax
\let\theglossary\relax
\let\endtheglossary\relax

\usepackage[framemethod=tikz]{mdframed}

%\usepackage[noredefwarn]{glossaries}

\usepackage[noredefwarn]{glossaries}
\renewcommand*{\glstextformat}[1]{\textcolor{green!60!black}{#1}}


\usepackage{fancyhdr}
\pagestyle{fancy}
\renewcommand{\sectionmark}[1]{%
	\markboth{\thesection.\ #1}{}}

\rhead{}
\chead{}
%\fancyhead[RO,LE]{\small Edição das regras oficiais do Softbol 2018-2021}

\fancyhead[RO,LE]{\small Edição das regras oficiais do Softbol 2022-2025}
\fancyhead[RE,LO]{\small{}}

%\fancyhead[RE,LO]{\small Regras oficiais do Softbol 2018-2021}
%\fancyfoot[LE][RO]{}
%\fancyfoot[LO][RE]{}
\lfoot{emilio.e.k@gmail.com}
\cfoot{WBSC - ASB}
\rfoot{\thepage\ de \pageref{LastPage}}
\renewcommand{\headrulewidth}{0.4pt}
\renewcommand{\footrulewidth}{0.4pt}

\setlength{\columnsep}{4mm}
\setlength{\columnseprule}{0.2pt}
%\setlength{\unitlength}{18mm}
%\newcommand{\blob}{\rule[-.2\unitlength]{2\unitlength}{.5\unitlength}}
%\newcommand\rblob{\thepage
%	\begin{picture}(0,0)
%	\put(1,-\value{section}){\blob}
%	\end{picture}}
%\newcommand\lblob{%
%	\begin{picture}(0,0)
%	\put(-3,-\value{section}){\blob}
%	\end{picture}%
%	\thepage}
%\pagestyle{fancy}
%\cfoot{}
%\newcounter{line}
%\newcommand{\secname}[1]{\addtocounter{line}{1}%
%	\put(1,-\value{line}){\blob}
%	\put(-7.5,-\value{line}){\Large \arabic{line}}
%	\put(-7,-\value{line}){\Large #1}}
%\newcommand{\overview}{\thepage
%	\begin{picture}(0,0)
%	\secname{Introduction}
%	\secname{The first year}
%	\secname{Specialisation}
%	...etc...
%	\end{picture}}


% Configuracao do sumario
%-----------------------------------------------------------------------------
% Modifica o espaçamento no sumário
% Nao ha espacos, exceto para as entradas de capitulos
% \setlength{\cftbeforeparagraphskip}{0pt}
\setlength{\cftbeforesubsectionskip}{-2pt}
\setlength{\cftbeforesectionskip}{3pt}
\setlength{\cftbeforesubsubsectionskip}{0pt}
\setlength{\cftbeforechapterskip}{-5pt}%\onelineskip}


\usepackage{titlesec}
\titleformat{\chapter}{\normalfont\large}{\bfseries\thechapter.}{18pt}{\bfseries}
\titleformat*{\section}{\large}
\titleformat*{\subsection}{\normalsize\bfseries}
\titleformat*{\subsubsection}{\normalsize}

\titlespacing\section{0pt}{12pt plus 2pt minus 2pt}{0pt plus 2pt minus 2pt}
\titlespacing\subsection{0pt}{1pt plus 2pt minus 2pt}{0pt plus 2pt minus 2pt}
\titlespacing\subsubsection{0pt}{1pt plus 2pt minus 2pt}{0pt plus 2pt minus 2pt}


%\sectionfont{\fontsize{12}{15}\selectfont}
%\subsectionfont{\fontsize{11}{15}\selectfont}

\newcommand{\tocfont}{\normalsize}	% define tamanho de fonte para sumario como normal

%\setenumerate{itemsep=5mm,topsep=-1mm}

\usepackage[portuges]{minitoc}
\mtcsetoffset{minitoc}{10pt}
\nomtcrule

\newcounter{Examplecount}
\setcounter{Examplecount}{0}
%\refstepcounter{Examplecount}

\definecolor{MyGray}{rgb}{0.96,0.97,0.98}
%%%%%%%%%%%%%%%%%%%%%%%%%%%%%%%%%%%%%%%%%%%%%%%%%%%%%%%
\makeatletter
\newenvironment{exemplo}{%
	%\rule{1ex}{1ex}\hspace{\stretch{1}}
	\par
	\refstepcounter{Examplecount}
	\begin{lrbox}{\@tempboxa}
		\begin{minipage}{.975\columnwidth}\footnotesize
			\colorbox{blue!10}{\parbox{.975\columnwidth}{
					%\fbox{
					Exemplo \arabic{Examplecount}: }}%}
			\par }
		{\vspace{2mm}
		\end{minipage}
	\end{lrbox}%
	\colorbox{MyGray}{\usebox{\@tempboxa}}
	\hspace{\stretch{1}}\rule{1 mm}{1 mm}
}
\makeatother
%-----------------------------------------------
%%%%%%%%%%%%%%%%%%%%%%%%%%%%%%%%%%%%%%%%%%%%%%%%%%%%%%%
\makeatletter
\newenvironment{exemploc}[1]%{\arabic{Examplecount}}%
%\rule{1 mm}{1 mm}\hspace{\stretch{1}}\par
%\stepcounter{Examplecount}
%\refstepcounter{Examplecount}
{\par
	\vspace{-1 mm}
	\hspace{-4 mm}
	\begin{lrbox}{\@tempboxa}\begin{minipage}{.93\columnwidth-3mm}
			\colorbox{blue!10}{
				\parbox{.98\columnwidth}
				
			}}
			\small%\par
		}
		{\vspace{1  mm}
		\end{minipage}
	\end{lrbox}%
	\colorbox{MyGray}{\usebox{\@tempboxa}}
	%\hspace{\stretch{1}}\rule{1 mm}{1 mm}
}\makeatother
%-----------------------------------------------

\selectlanguage{portuges}
%\renewcommand{\sectionname}{SEC\c{C}\~AO}
%\fancyhead[R]{\overview}\mbox{}\newpage % This produces the overview page
%\fancyhead[R]{} % Front matter may follow here
%\clearpage
%\fancyhead[RE]{\rightmark}
%\fancyhead[RO]{\rblob}
%\fancyhead[LE]{\lblob}
%\fancyhead[LO]{{\leftmark}}


\renewcommand{\appendixname}{AP\^ENDICE}
\setlist[enumerate]{topsep=1mm,itemsep=1.75mm,partopsep=1.5mm,parsep=.5mm}
% Alteração da indentação dos itens do sumário
\cftsetindents{chapter}{0pt}{62pt}
\cftsetindents{section}{0pt}{45pt}
\cftsetindents{subsection}{7pt}{50pt}
\cftsetindents{subsubsection}{14pt}{55pt}

\abnormalparskip{8pt}



\usepackage{etoc}
\etocsetstyle{section}
{}
{\etocifnumbered
	{\etocsavedsectiontocline{\numberline{\etocnumber}\etocname}{\etocpage}}
	{\etocsavedsectiontocline{\numberline{}\etocname}{\etocpage}}%
}
{}
{}

\setlength{\lineskip}{3pt plus 4pt minus 0pt}
\lineskiplimit=\baselineskip


\usepackage{soul}
\sethlcolor{lightgray}

\makeatletter

\newcommand{\defhighlighter}[3][]{%
	\tikzset{every highlighter/.style={color=#2, fill opacity=#3, #1}}%
}

\defhighlighter{blue!30}{.3}

\newcommand{\highlight@DoHighlight}{
	\fill [ decoration = {random steps, amplitude=1pt, segment length=15pt}
	, outer sep = -15pt, inner sep = 0pt, decorate
	, every highlighter, this highlighter ]
	($(begin highlight)+(0,8pt)$) rectangle ($(end highlight)+(0,-3pt)$) ;
}

\newcommand{\highlight@BeginHighlight}{
	\coordinate (begin highlight) at (0,0) ;
}

\newcommand{\highlight@EndHighlight}{
	\coordinate (end highlight) at (0,0) ;
}

\newdimen\highlight@previous
\newdimen\highlight@current

\DeclareRobustCommand*\highlight[1][]{%
	\tikzset{this highlighter/.style={#1}}%
	\SOUL@setup
	%
	\def\SOUL@preamble{%
		\begin{tikzpicture}[overlay, remember picture]
		\highlight@BeginHighlight
		\highlight@EndHighlight
		\end{tikzpicture}%
	}%
	%
	\def\SOUL@postamble{%
		\begin{tikzpicture}[overlay, remember picture]
		\highlight@EndHighlight
		\highlight@DoHighlight
		\end{tikzpicture}%
	}%
	%
	\def\SOUL@everyhyphen{%
		\discretionary{%
			\SOUL@setkern\SOUL@hyphkern
			\SOUL@sethyphenchar
			\tikz[overlay, remember picture] \highlight@EndHighlight ;%
		}{%
		}{%
			\SOUL@setkern\SOUL@charkern
		}%
	}%
	%
	\def\SOUL@everyexhyphen##1{%
		\SOUL@setkern\SOUL@hyphkern
		\hbox{##1}%
		\discretionary{%
			\tikz[overlay, remember picture] \highlight@EndHighlight ;%
		}{%
		}{%
			\SOUL@setkern\SOUL@charkern
		}%
	}%
	%
	\def\SOUL@everysyllable{%
		\begin{tikzpicture}[overlay, remember picture]
		\path let \p0 = (begin highlight), \p1 = (0,0) in \pgfextra
		\global\highlight@previous=\y0
		\global\highlight@current =\y1
		\endpgfextra (0,0) ;
		\ifdim\highlight@current < \highlight@previous
		\highlight@DoHighlight
		\highlight@BeginHighlight
		\fi
		\end{tikzpicture}%
		\the\SOUL@syllable
		\tikz[overlay, remember picture] \highlight@EndHighlight ;%
	}%
	\SOUL@
}
\makeatother


%%%%%%%%%%%%%%%%%%%%%%%%%%%%%%%%%%%%%%%%%%%%%%%%%%%%%%%%%%%%
% Grid com coordenadas

\makeatletter
\def\grd@save@target#1{%
	\def\grd@target{#1}}
\def\grd@save@start#1{%
	\def\grd@start{#1}}
\tikzset{
	grid with coordinates/.style={
		to path={%
			\pgfextra{%
				\edef\grd@@target{(\tikztotarget)}%
				\tikz@scan@one@point\grd@save@target\grd@@target\relax
				\edef\grd@@start{(\tikztostart)}%
				\tikz@scan@one@point\grd@save@start\grd@@start\relax
				\draw[minor help lines] (\tikztostart) grid (\tikztotarget);
				\draw[major help lines] (\tikztostart) grid (\tikztotarget);
				\grd@start
				\pgfmathsetmacro{\grd@xa}{\the\pgf@x/1cm}
				\pgfmathsetmacro{\grd@ya}{\the\pgf@y/1cm}
				\grd@target
				\pgfmathsetmacro{\grd@xb}{\the\pgf@x/1cm}
				\pgfmathsetmacro{\grd@yb}{\the\pgf@y/1cm}
				\pgfmathsetmacro{\grd@xc}{\grd@xa + \pgfkeysvalueof{/tikz/grid with coordinates/major step}}
				\pgfmathsetmacro{\grd@yc}{\grd@ya + \pgfkeysvalueof{/tikz/grid with coordinates/major step}}
				\foreach \x in {\grd@xa,\grd@xc,...,\grd@xb}
				\node[anchor=north] at (\x,\grd@ya) {\scriptsize \pgfmathprintnumber{\x}};
				\foreach \y in {\grd@ya,\grd@yc,...,\grd@yb}
				\node[anchor=east] at (\grd@xa,\y) {\scriptsize \pgfmathprintnumber{\y}};
			}
		}
	},
	minor help lines/.style={
		help lines,
		step=\pgfkeysvalueof{/tikz/grid with coordinates/minor step}
	},
	major help lines/.style={
		help lines,
		line width=\pgfkeysvalueof{/tikz/grid with coordinates/major line width},
		step=\pgfkeysvalueof{/tikz/grid with coordinates/major step}
	},
	grid with coordinates/.cd,
	minor step/.initial=2,
	major step/.initial=10,
	major line width/.initial=1pt,
}
\makeatother

%%%%%%%%%%%%%%%%%%%%%%%%%%%%%%%%%%%%%%%%%%%%%%%%%%%%%%%%%%%%

\usepackage{pdflscape}

