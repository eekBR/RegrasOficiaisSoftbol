
\longnewglossaryentry{inning}
{
 name= \textit{innning},
 plural= \textit{innings},
 description={ \'e aquela parte de um jogo durante a qual ambas as equipes atacam/defendem. A primeira metade do inning termina quando ocorrem tr\^es elimina\c{c}\~oes do ataque.\par A nova metade de \gls{inning} inicia imediatamente após a última elimina\c{c}\~ao da primeira metade},
 descriptionplural={ s\~ao partes de um jogo durante a qual ambas as equipes atacam/defendem. \gls{inning}},
 firstplural={\glsentrydescplural{inning} (\glsentryplural{inning})}
}

\longnewglossaryentry{coach}
{
 name= \textit{coach},
 plural= \textit{coaches},
}{\'e uma pessoa que \'e respons\'avel pelas a\c{c}\~oes de sua equipe no campo e pela comuni\-ca\-\c{c}\~ao com o \'arbitro e com a equipe contr\'aria}

\longnewglossaryentry{base coach}
{
 name= \textit{base coach},
 plural= \textit{base coaches},
}{\'e uma pessoa que \'e respons\'avel pelas a\c{c}\~oes de sua equipe no campo e pela comuni\-ca\-\c{c}\~ao com o \'arbitro e com a equipe contr\'aria}


\newglossaryentry{play ball}
{
 name= \textit{play ball},
 description={\'e o an\'uncio de que o jogo ir\'a iniciar ou reiniciar, \'e feito pelo \'arbitro de \textit{home} e deve sinalizar que a bola est\'a viva},
}

\newglossaryentry{play}
{
 name= \textit{play},
 description={\'e o mesmo que \gls{play ball}},
}

\newglossaryentry{bat}
{
 name= \textit{bat},
 plural= \textit{bats},
 description={\'e o an\'uncio de que o jogo ir\'a iniciar ou reiniciar, \'e feito pelo \'arbitro de \textit{home} e deve sinalizar que a bola est\'a viva},
}

\newglossaryentry{batter's box}
{
 name= \textit{batter's box},
 description={\'area do batedor},
}

\newglossaryentry{catcher's box}
{
 name= \textit{catcher's box},
 description={\'area do receptor},
}

\newglossaryentry{dugout}
{
 name= \textit{dugout},
 description={abrigo para membros da equipe. \'E a \'area em territ\'orio de bola morta, destinada somente a membros da equipe},
}

\newglossaryentry{infield}
{
 name= \textit{infield},
 description={Campo interno. \'E a \'area do campo em territ\'orio \textit{fair} normalmente coberta por defensores do campo interno},
}


\newglossaryentry{outfield}
{
 name= \textit{outfield},
 description={campo externo. \'E aquela parte do campo de jogo em territ\'orio \textit{fair}, que est\'a al\'em do campo interno},
}

\newglossaryentry{bench}
{
 name= \textit{bench},
 description={ banco onde se acomodam as equipes em jogo},
}

\longnewglossaryentry{homerun}
{
 name= \textit{home run},
 plural= \textit{home runs},
}{é uma rebatida na qual o rebatedor é capaz de circular todas as bases, terminando na casa base e anotando uma corrida (junto com uma corrida anotada por cada corredor que já estava em base), com nenhum erro cometido pelo time defensivo na jogada que resultou no batedor-corredor avançando bases extras. O feito é geralmente conseguido rebatendo a bola sobre a cerca do campo externo entre os postes de falta (ou fazendo contato com um deles), sem que ela antes toque o chão. Ou ainda rebatendo a bola dentro do campo de jogo mas ainda assim percorrendo todas as bases voltando ao \gls{homeplate}. Jogada conhecida como inside-the-park home run}

\longnewglossaryentry{homeplate}
{
 name= \textit{home plate},
}
{ Ao contr\'ario das outras bases que são quadradas, o \textit{home plate} \'e uma laje de borracha branca com cinco lados que \'e colocada no nível do campo, na posi\c{c}\~ao \gls{home},
}

\newglossaryentry{time}
{
 name=\textit{time},
 description={\'E o termo usado por um \'arbitro para ordenar a paralisa\c{c}\~ao de jogada num jogo}
}


\newglossaryentry{pitching plate}
{
 name=\textit{pitching plate},
 description={\'E a laje de borracha branca na qual o \gls{pitcher} se posiciona para os arremessos}
}

\newglossaryentry{pitcher's plate}
{
 name=\textit{pitcher's plate},
 description={\'E onde o arremessador fica quando arremessa a bola. No centro de uma circunfer\^encia est\'a uma laje de borracha branca, que \'e chamada \textit{pitcher's plate} ou \textit{pitcher's rubber}}
}

\longnewglossaryentry{fair}
{
 name=\textit{fair},
}{\'E aquela parte dentro do campo de jogo, incluindo as linhas de \gls{foul} da primeira e terceira base, que vai do \gls{homeplate} at\'e a parte inferior da cerca do campo externo e perpendicularmente para cima}


\newglossaryentry{foul}
{
 name=\textit{foul},
 description={\'E qualquer parte do campo de jogo que n\~ao est\'a em Territ\'orio \gls{fair}},
}



\longnewglossaryentry{fly de sacrificio}
{
 name={\textit{fly} de sacrif\'icio},
}{ Uma bola batida \'e considerada um \textit{fly} de sacrif\'icio (\textit{sacrifice fly}, \textbf{SF}) se os quatro crit\'erios seguintes forem satisfeitos:
 \begin{enumerate}
 \item H\'a menos de dois eliminados quando a bola \'e rebatida.
 \item A bola \'e rebatida ao campo externo.
 \item O batedor est\'a eliminado porque um defensor apanha a bola (ou teria sido eliminado se n\~ao fosse por um erro).
 \item Um corredor que j\'a est\'a em base anota na jogada.
\end{enumerate}}

\newglossaryentry{home base}
{
 name=\textit{home base},
 description ={Tamb\'em designado \textit{home} nas regras, \'e a base final que um jogador deve tocar para anotar uma corrida}
}

\longnewglossaryentry{home}
{
 name=\textit{home},
}{
 Formalmente designado \textit{home base} nas regras, \'e a base final que um jogador deve tocar para anotar uma corrida}

\newglossaryentry{strike}
{
 name=\textit{strike},
 plural=\textit{strikes},
 description={O \gls{pitcher} arremessa a bola dentro de sua \'area, visando a zona de \textit{strike} (um retângulo imaginário entre os joelhos e os ombros do \gls{catcher}). Se a bola for dentro da zona de \textit{strike}, conta-se um \textit{strike}. \'E tamb\'em denominada de bola "boa"},
}


\newglossaryentry{called strike}
{
 name=\textit{called strike},
 description={Arremessar a bola na zona de strike e o rebatedor n\~ao tentar a rebatida}}

\newglossaryentry{out}
{
 name=\textit{out},
 plural=\textit{outs},
 description={\'E a contagem de atletas eliminados num \gls{inning}}
}


\newglossaryentry{pitcher}{name={\textit{pitcher}},
 description={ ou Arremessador \'e o respons\'avel por realizar os arremessos do time, em busca da elimina\c{c}\~ao advers\'aria)}}

\longnewglossaryentry{foul tip}
{
 name=\textit{foul tip},description={Se o jogador apenas “raspar” o taco na bolinha e o \gls{catcher} pegar, ser\'a considerado uma \textit{Foul Tip}, ou seja, ser\'a \gls{strike} e n\~ao elimina\c{c}\~ao autom\'atica. Por\'em, se o rebatedor j\'a estiver com 2 \glspl{strike} e rebater uma \textit{Foul Tip}, ser\'a um \gls{strikeout} (ao contrário das \textit{Fouls} comuns, um \textit{Foul Tip} pode ser o 3º \gls{strike}). Se o \gls{catcher} n\~ao segurar, ser\'a uma \gls{foul} normal}
}


\newglossaryentry{catcher}
{
 name=\textit{catcher},
 description={ou Receptor \'e responsável por recepcionar as bolas arremessadas pelo \gls{pitcher}}
}

\newglossaryentry{fly}
{
 name=\textit{fly},
 description={\'E uma bola rebatida que vai ao ar (alto)}
}

\newglossaryentry{fair fly}
{
 name=\textit{fair fly},
 description={\'E uma bola rebatida para o territ\'orio \gls{fair}}
}


\newglossaryentry{strikeout}
{
 name=\textit{strike out},
 plural=\textit{strike outs},
 description={ A quantidade de batedores eliminados por \gls{strike} (\textit{strike outs}) por cada arremessador}
}

\newglossaryentry{coin toss}
{
 name=\textit{coin toss},
 description={cara ou coroa}
}

\newglossaryentry{passedball}
{
 name=\textit{passed ball},
 plural=\textit{passed balls},
 description={ bola arremessada defens\'avel que passa para tr\'as do receptor.}
}

\longnewglossaryentry{ball}
{ 	name=\textit{ball},
	plural=\textit{balls},
	description={Caso o arremesso do \gls{pitcher} não atinja a zona de \gls{strike} e o rebatedor não tente a rebatida, o árbitro faz a chamada de \textit{ball}. A cada quatro \textit{balls} contra um mesmo adversário, o arremessador cede um \gls{walk}, que consiste na passagem automática do rebatedor para a 1ª base. Caso já exista um jogador do time na 1ª base, este avançará para a 2ª base e o rebatedor, que recebeu o \gls{walk}, vai para a 1ª base}
}

\newglossaryentry{swing strike}
{ name=\textit{swing strike},description={Arremessar a bola em qualquer área e, ao tentar a rebatida, o rebatedor não acerta a bola}}

\newglossaryentry{backstop}{ 
	name=\textit{backstop},
	plural=\textit{backstops},
	description={barreira situada atr\'as do \gls{homeplate}}}

\newglossaryentry{fly out}
{ name=\textit{fly out},description={Se, após a rebatida, um jogador da defesa pega a bola sem que ela toque no chão, o rebatedor é automaticamente eliminado}}

\newglossaryentry{sliding}{
	name=\textit{sliding},
	plural={\textit{slidings}},
	description={ \'E o ato de deslizar a uma base}}


\newglossaryentry{bunt}{
 name=\textit{bunt},
 description={é uma forma sutil de rebater com um toque do bastão na bola. Com uma mão em uma extremidade e outra na parte mediana do taco, dá-se contato na bola arremessada resultando uma rebatida curta}
}


\newglossaryentry{foulball}
{
	name=\textit{foul ball},
 description={bola rebatida ilegal. \'E quando uma bola rebatida cai e permanece na área de \gls{foul}}
}

\newglossaryentry{OPO}
{ name=\textit{OPO},description={Jogador somente da Ofensiva}}

\newglossaryentry{warm-up bat}
{ name=\textit{warm-up bat},
	description={\gls{bat} para fazer aquecimento. O \textit{warm-up bat} tem de ser feito em uma só peça, e deve sujeitar-se aos requisitos exigidos aos dispositivos de segurança (empunhadura de segurança e saliência arredondada na extremidade do cabo) do “bat” oficial. Tem de estar marcado \textit{warm-up}, com letras de 3,2cm (1 1/4 polegada), na extremidade do cilindro. A extremidade do cilindro precisa ter mais de 5,7cm (2 1/4 polegadas)}}

\newglossaryentry{batting order}
{ name=\textit{batting order},description={ordem de batedores}}

\newglossaryentry{throwing}
{ name=\textit{throwing},description={Sempre que um jogador da defesa lançar a bola para outro defensor de maneira errada, ou seja, muito alta/baixa ou na direção errada, impossibilitando que o companheiro apanhe a bola, será considerado um erro}}

\newglossaryentry{tag}
{ name=\textit{tag},description={\'E quando um jogador da defesa encosta a bola num corredor, que não est\'a em contato com uma base, e portanto o atleta tocado \'e automaticamente eliminado}}
\newglossaryentry{bat boy}{ name=\textit{bat boy},description={recolhedor de \gls{bat}, t\^em de usar um capacete aprovado enquanto est\~ao no campo ou dentro do \Gls{dugout}}}
\newglossaryentry{bat girl}{ name=\textit{bat girl},description={recolhedora de \gls{bat}, t\^em de usar um capacete aprovado enquanto est\~ao no campo ou dentro do \Gls{dugout}}}

\longnewglossaryentry{mitt}{
 name=\textit{mitt},
 plural=\textit{mitts}
}{A luva de um \gls{catcher} é mais comumente chamada de \textit{mitt} porque não tem posição separada para cada dedo, como as luvas de outras posições de defesa. Isto permite aos \glspl{catcher} apanharem bolas rápidas durante um jogo inteiro sem se desgastarem rapidamente ou as capturas se tornarem dolorosas}

\newglossaryentry{line-up}{ name=\textit{line-up},description={escala\c{c}\~ao da equipe}}

\newglossaryentry{tag up}{ name=\textit{tag up},description={\'e o corredor retocar ou permanecer na base de momento do arremesso at\'e (ou depois) da bola ser primeiro tocada por um defensor}}

\newglossaryentry{tagging up}{ name=\textit{tagging up},description={\'e o ato de (re)tocar a base}}

\longnewglossaryentry{infieldfly}{
	name=\textit{infield fly},
	description={\'E quando se tem jogadores nas bases e numa bola que vai muito alta, \gls{fly}, que não sai do \gls{infield} faz com que o rebatedor seja eliminado mesmo antes dos defensores tocarem na bola. Isso ocorre para evitar que os infielders deixem a bola cair de propósito, obrigando os jogadores que já estavam em base a correrem e serem facilmente eliminados}
}

\newglossaryentry{pickoff play}{ name=\textit{pickoff play},description={\'e a jogada em que o arremessador tenta segurar o corredor na base, ou eliminar o corredor que est\'a fora da base}}

\newglossaryentry{ID}{name=\textit{ID},description={A marca de identifica\c{c}\~ao da atleta no uniforme}}

\newglossaryentry{walk} {
 name=\textit{walk},
  plural=\textit{walks},
 description={Quatro \glspl{ball} equivalem a conquista da primeira base. O mesmo que \gls{baseonballs}}
}

\newglossaryentry{baseonballs}{
 name=\textit{base on ball},
 description={\'E a defini\c{c}\~ao de \gls{walk}, ou seja, o \gls{pitcher} cede uma base para o rebatedor atrav\'es de bolas fora da zona de \gls{strike}}
}

\longnewglossaryentry{box}{
 name=\textit{box},}{
 cada um dos ret\^angulos demarcados em ambos os lados do \gls{home base}, dentro da qual o rebatedor deve posicionar-se para rebater a bola arremessada, no caso do \gls{battersbox}.

 No caso do \gls{catcher's box}: \'area onde o \gls{catcher} deve posicionar-se para receber a bola arremessada.
}

\newglossaryentry{catch}{
	name=\textit{catch},
	description={Ocorre uma pegada legal quando um defensor, com sua(s) m\~ao(s) ou luva, pega uma bola batida ou lan\c{c}ada}}

\longnewglossaryentry{check swing}{ name=\textit{check swing}}
{
 Quando o rebatedor faz o movimento da rebatida mas, no último momento, segura o bastão. Isso acontece, na maioria das vezes, quando o rebatedor percebe que o arremesso é ruim para rebater e prefere receber um \gls{ball} do que ir para o swing de maneira errônea.

 Quando isso acontece, cabe ao árbitro julgar se foi \gls{strike} ou \gls{ball} – se o bastão passar da metade do plano de base é \gls{strike}}

\longnewglossaryentry{crow hop}{ name=\textit{crow hop}}{\'E o ato de um arremessador que:

 \begin{enumerate}[label=\alph*)]
 \item d\'a o impulso de um lugar que n\~ao o \gls{pitcher's plate} para soltar a bola; ou

 \item d\'a um passo para fora do \gls{pitcher's plate}, estabelecendo um segundo impulso (ou ponto de partida), e depois, iniciando desse novo ponto de partida, completa o
 arremesso.
\end{enumerate}}

\longnewglossaryentry{delayeddeadball}{ name=\textit{delayed dead ball}}{ Bola morta demorada: \'E uma situa\c{c}\~ao de jogo em que a bola permanece viva at\'e a conclus\~ao de uma jogada; quando a jogada estiver totalmente conclu\'ida, e se for necess\'ario, um \'arbitro declarar\'a que a bola est\'a morta e aplicar\'a a regra apropriada. }

\newglossaryentry{doubleplay}{ 
	name=\textit{double play},
	plural=\textit{double plays},
	description={\'E uma jogada executada pela equipe na defensiva, na qual dois jogadores da equipe na ofensiva s\~ao declarados \gls{out} em consequ\^encia de a\c{c}\~ao cont\'inua}}

\newglossaryentry{fake tag}{ name=\textit{fake tag},description={ simular um toque (\gls{fake tag}), sem estar de posse da bola; }}

\newglossaryentry{forced out}{ name=\textit{forced out},description={Eliminação Forçada é aquela que pode ser feita somente quando um corredor perde o direito à base que está ocupando porque o batedor se torna um batedor-corredor, e antes que esse batedor-corredor (ou um corredor subseqüente) tenha sido eliminado}}

\newglossaryentry{forfeited game}{ name=\textit{forfeited game},description={Confisco de jogo é o ato mediante o qual o árbitro de \gls{home} dá o jogo por encerrado, declarando vitoriosa a equipe não infratora}}

\newglossaryentry{running start}{name=\textit{running start},description={ \'E quando a atleta se posiciona atr\'as da base e n\~ao fica em contato com ela, para iniciar a corrida \`a base seguinte tomando impulso a partir dessa posi\c{c}\~ao}}

\newglossaryentry{blocked ball}{ name=\textit{blocked ball},description={\'E uma bola rebatida, lançada ou arremessada que é tocada, parada ou manuseada por uma pessoa que não está atuando no jogo, ou que toca qualquer objeto que não faz parte do equipamento oficial ou da área oficial de jogo. }}

\newglossaryentry{foul fly}{ name=\textit{foul fly},description={rebatida \gls{fly} para o território \gls{foul}}}

%\newglossaryentry{hitbypitch}{ name=\textit{hit by pitch},description={o batedor é atingido por uma bola arremessada}}

\newglossaryentry{line drive}{ name=\textit{line drive},description={bola rebatida em linha reta}}

\longnewglossaryentry{over-slide}{ name=\textit{over-slide},}{É o ato de um jogador da ofensiva, que, atuando como corredor, desliza para uma base que está tentando alcançar e ultrapassa-a. Ocorre geralmente quando o impulso da corrida faz o corredor perder o contato com a base, colocando-o em risco. O batedor-corredor pode ultrapassar a primeira base, sem correr o risco de ser eliminado, desde que regresse imediatamente a essa base}

\newglossaryentry{tagging}{ name=\textit{tagging},description={\'e o ato de um defensor, que consiste em tocar }}

\newglossaryentry{overthrow}{ name=\textit{overthrow},description={É uma jogada em que uma bola é mau lançada de um defensor a outro e ultrapassa os limites do campo de jogo, ou se torna uma Bola Bloqueada}}

\longnewglossaryentry{run ahead rule}{ name=\textit{run ahead rule},}{
 \begin{enumerate}
 \item A Regra de Vantagem de Pontos tem de ser usada em todos os campeonatos da FIS (ISF).
 \begin{enumerate}
 \item (SOMENTE AR e AM) Vinte (20) pontos depois de três (3) \textit{innings}, quinze (15) pontos depois de quatro (4) \textit{innings} ou sete (7) pontos depois de cinco (5) \textit{innings}.
 \item (SOMENTE AL) Vinte (20) pontos depois de quatro (4) \textit{innings} ou quinze (15) pontos depois de cinco (5) \textit{innings}.
 \end{enumerate}
 \item É preciso jogar \glspl{inning} completos, a menos que a equipe local anote a quantidade necessária de pontos antes de terminar a sua metade de \gls{inning}. Quando a equipe visitante alcança a quantidade necessária de pontos na primeira metade do \gls{inning}, a equipe local precisa ter sua oportunidade de bater na segunda metade desse \gls{inning}.
\end{enumerate} }

\longnewglossaryentry{coachsbox}{
	name=\textit{coach's box},
	plural=\textit{coach's boxes},
	description={área destinada ao \gls{coach} fica atrás de uma linha, traçada fora do campo (\textit{diamond}) e paralelamente à linha de base. }
 }

\newglossaryentry{safe}{ name=\textit{safe},description={Se o batedor rebater a bola e chegar a salvo; ou um corredor que chega a salvo a uma base}}

\longnewglossaryentry{slap hit}{ name=\textit{slap hit},}{
 \'E uma bola rebatida mediante um movimento curto controlado com o \gls{bat} , e não com um \gls{swing} completo. Os dois tipos mais comuns de \textit{slap hit} são:
 \begin{enumerate}
 \item Aquele em que o batedor assume uma posição como se fosse executar \gls{bunt}, mas depois, ou impulsiona a bola contra o solo com um \gls{swing} rápido e curto, ou empurra a bola por cima do campo interno.
 \item Aquele em que o batedor dá passos acelerados (dentro do \textit{batter's box}) na direção do arremessador antes de fazer contato com o arremesso.
 \end{enumerate}
 NOTA: Um \textit{slap hit} não é considerado um \gls{bunt}}

\newglossaryentry{slap}{ name=\textit{slap},description={ato de executar um \gls{slap hit}}}

\newglossaryentry{squeeze play}{ name=\textit{squeeze play},description={É uma jogada em que a equipe na ofensiva, com um corredor na terceira base, tenta anotar ponto com esse corredor por meio de um toque na bola dado pelo batedor}}

\newglossaryentry{steal}{ name=\textit{steal},description={ \'e o ato de executar um \gls{stealing}}}

\newglossaryentry{stealing}{ name=\textit{stealing},description={Roubo de Base é a ação de um corredor que tenta avançar durante ou depois de um arremesso ao batedor. O Roubo de Base não é permitido no Arremesso Lento}}

\newglossaryentry{swing}{ name=\textit{swing},description={ato de girar o \gls{bat}},}

\longnewglossaryentry{trappedball}{ name=\textit{trapped ball},
plural=\textit{trapped balls},}
{
 \'E:
 \begin{enumerate}
 \item Uma bola rebatida legalmente para o ar (\gls{fly}), inclusive um \gls{line drive}, que toca o solo ou uma cerca antes de ser apanhada.

 \item Uma bola rebatida legalmente para o ar (\gls{fly}) que é apanhada contra a cerca, com a luva ou a mão.

 \item Uma bola lançada a qualquer base para efetivar uma eliminação forçada (\gls{forced out}), que é apanhada com a luva sobre a bola que está tocando o solo.

 Não será um \textit{Trapped Ball} se a pegada for feita com a luva sob a bola.

 \item (SOMENTE AR) Uma bola arremessada que, após ter tocado o solo antes de ser agarrada pelo receptor, é declarada \gls{strike}.
 \end{enumerate}
}

\newglossaryentry{wild pitch}{ name=\textit{wild pitch},description={É um arremesso que passa tão alto, tão baixo, ou tão fora do \gls{homeplate}, que o receptor não pode ou não consegue pará-lo e controlá-lo com um esforço normal}}

\newglossaryentry{windmill}{ name=\textit{windmill},description={\'E o giro do braço do atleta antes do arremesso.
}}

\newglossaryentry{run-down play}{name=\textit{run-down play},description={\'E quando a defesa está no controle, tentando pegar o corredor que se foge para evitar o \gls{tag}}}

\longnewglossaryentry{tie-breaker rule}{name=\textit{tie-breaker rule},}{
 No final de sete \glspl{inning}, a regra do International \textit{Tie-Breaker} é obrigatória para ajudar determinar um vencedor.
 O jogador programado para bater no nono \gls{inning} (primeira parte do oitavo e cada metade da próxima entrada) começa na segunda base. 

 Por exemplo, se o batedor 8 está programado para bater, o número 1 começa no segundo; se o número 2 é programado para bater, o número 2 começa na segunda base
}

\newglossaryentry{undershirt}{ name=\textit{undershirt},description={Camisetas internas, segunda pele}}

\newglossaryentry{vaping}{ name=\textit{vaping},description={\'E o ato de usar cigarros eletr\^onicos/vaporizadores}}

\newglossaryentry{underhanded motion}{name=\textit{underhanded motion},description={\'E o movimento de arremesso feito com um movimento em que a mão do arremessador fica em nível inferior ao do cotovelo}}

\newglossaryentry{ground rule double}{
 name=\textit{ground rule double},
 description={Indica que o batedor-corredor (ou corredor) tem direito a duas bases}
}

\newglossaryentry{ground}{ name=\textit{ground},description={\'E a bola rasteira proveniente de uma batida}}

\newglossaryentry{sweatband}{ name=\textit{sweat band},description={Faixa de transpiração. Faixa que impede o suor de correr pela testa, ou dos braços para as mãos}}

\newglossaryentry{plate}{ name=\textit{plate},description={Almofadas indicativas das bases ou a placa de borraçao do \gls{home}}}

\newglossaryentry{indicator}{ name=\textit{indicator},description={marcador de ball, strike e out}}

\newglossaryentry{scorebook}{ name=\textit{score book},description={Livro de Anotações}}

%%%%%%%%%%%%%%%%%%%%%%%

\newglossaryentry{slingshot}{name=\textit{slingshot},description={ qquer coisa}}

%%%%%%%%%%%%%%%%%%%%%%%%%%%%%%%%%%%%%%%%%%%%%%%%%%%%%%%%%%%%%%%%%%%%%%%%%%%%%%%%%%%%%%%%%%%%%%%%%%%%%%%%%%

\longnewglossaryentry{2B}{ name=\textit{2B},plural=\textit{2B},}{ Double: um batedor ganha um 2B quando a batida que ele deu proporcionou avanço de duas bases}

\longnewglossaryentry{foultip}{
 	name=\textit{Foul tip},
    plural=\textit{Foul tip},
    description={uma bola batida que vai direto do bastão para as mãos do catcher e é legalmente capturada}
}

% \longnewglossaryentry{3B }{ name=\textit{3B },plural=\textit{3B },}{ Triple: um batedor ganha um 3B quando a batida que ele deu proporcionou avanço de três bases}
% \longnewglossaryentry{APP }{ name=\textit{APP },plural=\textit{APP },}{ Appearance: é creditado uma appearance todas vezes que o pitcher entra em um determinado jogo}
% \longnewglossaryentry{Assists}{ name=\textit{Assists},plural=\textit{Assists},}{ termo utilizado para designar uma assistência de outro defensor para que o assistido consiga eliminar fisicamente um adversário no jogo}
% \longnewglossaryentry{AVG }{ name=\textit{AVG },plural=\textit{AVG },}{ Batting Average: é um índice conseguido dividindo se a quantidade de batidas conseguidas por um batedor pela quantidade de vezes que esteve batendo. É uma das mais comentadas estatísticas neste jogo. Expressa entre zero ( .000 ) e um ( 1.000)}
% \longnewglossaryentry{Ball}{ name=\textit{Ball},plural=\textit{Ball},}{ é a bola "ruim", que é arremessada fora da "zona de strike". A cada quatro bolas ruins lançadas pela arremessadora, o time adversário ganha o direito de avançar uma base}
% \longnewglossaryentry{Bat}{ name=\textit{Bat},plural=\textit{Bat},}{ taco, bastão}
%
% \longnewglossaryentry{Bata}
% { name=\textit{Bata},plural=\textit{Batas},}
% { é o termo japonês para "Bastão" ou \textit{Bat}}
%
\longnewglossaryentry{battersbox}
{
	name= \textit{Batter's Box} (Área do Batedor),}
{
	description= {um campo regular tem duas áreas de batedores. À esquerda e à direita da placa de home, de onde ele deve se posicionar para bater a bola}
}
%%
%% \longnewglossaryentry{BB }{ name=\textit{BB },plural=\textit{BB },}{ Base on Balls: é uma concessão de bases. Ao batedor é permitido avançar à primeira base mediante 4 arremessos inválidos cedidos pelo técnico do time na defensiva. Geralmente ocorre quando o batedor é um bom batedor e time está receoso de tomar vários pontos}
%% \longnewglossaryentry{Blocked Ball}{ name=\textit{Blocked Ball},plural=\textit{Blocked Ball},}{ bola desviada por algum material estranho espalhado no campo, bola rebatida, lançada ou arremessada que fica alojada na cerca}
%% \longnewglossaryentry{Bola Morta:}{ name=\textit{Bola Morta:},plural=\textit{Bola Morta:},}{ Diz-se bola morta quando por algum motivo ou necessidade o árbitro paralisa o jogo, as jogadas não podem ocorrer, e os jogadores de ataque devem permanecer nas suas bases. Uma bola rebatida fora do campo é bola morta; uma interferência, é bola morta}
%% \longnewglossaryentry{Bola Viva}{ name=\textit{Bola Viva},plural=\textit{Bola Viva},}{ quando a bola não está invalidada. As jogadas e eliminações podem ocorrer e os atacantes podem, por sua conta e risco, tentar roubar bases}

%% \longnewglossaryentry{Boro}{ name=\textit{Boro},plural=\textit{Boro},}{ é o termo japonês para "Bola" ou \textit{Ball}}
%% \longnewglossaryentry{C }{ name=\textit{C },plural=\textit{C },}{ Chances: representa o número de oportunidades que o jogador teve com o intuito de eliminar um adversário. É utilizado em estatísticas de jogo}
%% \longnewglossaryentry{Catcher}{ name=\textit{Catcher},plural=\textit{Catcher},}{ é o receptor, jogador do time da defesa que se posiciona atrás do rebatedor e apanha as bolas arremessadas e não são rebatidas}
%% \longnewglossaryentry{Catcher's Interference (Interferência do Catcher)}{ name=\textit{Catcher's Interference (Interferência do Catcher)},plural=\textit{Catcher's Interference (Interferência do Catcher)},}{ se o catcher ou qualquer outro defensor interferir com o batedor durante um arremesso, é concedida a primeira base ao batedor. A interferência pode ser, por exemplo, encostar a luva no bastão do batedor na hora da batida}
%% \longnewglossaryentry{CH }{ name=\textit{CH },plural=\textit{CH },}{ Changeup: é um tipo de arremesso mais lento do softball e faz com que a bola caia bem perto do \textit{home plate}. É um arremesso de efeito}
%% \longnewglossaryentry{CI }{ name=\textit{CI },plural=\textit{CI },}{ Catcher's Interference: quando um catcher (ou outro defensor) interfere com o batedor em qualquer momento em que ele está tentando realizar uma rebatida. Neste caso, o batedor ganha a primeira base. (de novo?)}
%% \longnewglossaryentry{Collision at Home Plate (Colisão no Home)}{ name=\textit{Collision at Home Plate (Colisão no Home)},plural=\textit{Collision at Home Plate (Colisão no Home)},}{ o corredor não deve de maneira nenhuma desviar do seu caminho direto para o Homebase. Mas o Catcher deve, se não tiver de posse da bola, abrir espaço para a passagem do corredor. Caso haja um contato físico entre os dois, caracteriza esta colisão}
%% \longnewglossaryentry{CS }{ name=\textit{CS },plural=\textit{CS },}{ Caught Stealing: número de vezes que um jogador é eliminado (por toque) ao tentar roubar uma base}
%% \longnewglossaryentry{CSB }{ name=\textit{CSB },plural=\textit{CSB },}{ Caught Stealing (Pitcher/Catcher): número de vezes que um determinado jogador foi eliminado por tentar roubar bases}
%% \longnewglossaryentry{CU }{ name=\textit{CU },plural=\textit{CU },}{ Curveball: bola curva. É um arremesso de efeito}
%% \longnewglossaryentry{Deto Boro}{ name=\textit{Deto Boro},plural=\textit{Deto Boro},}{ é o termo japonês para "bola morta"}


%% \longnewglossaryentry{Double Play}{ name=\textit{Double Play},plural=\textit{Double Play},}{ é o ato de fazer dois eliminados durante a mesma jogada contínua. As jogadas duplas são relativamente comuns, pois podem ocorrer sempre que houver pelo menos um corredor em base e menos de dois outs}
%% \longnewglossaryentry{DP }{ name=\textit{DP },plural=\textit{DP },}{ \textit{Designated Player}: jogador colocado na lista de jogadores que serve de batedor no lugar do FLEX, que geralmente é colocado na lista como sendo o 10o jogador}
%% \longnewglossaryentry{E }{ name=\textit{E },plural=\textit{E },}{ Errors: um defensor é creditado um erro quando, no julgamento do anotador oficial, ele falhou ao tentar converter uma eliminação que um defensor normal conseguiria}
%% \longnewglossaryentry{ERA }{ name=\textit{ERA },plural=\textit{ERA },}{ Earned Run Average: é o numero de pontos concedidos pelo arremessador dividido pelo número de innings jogados (o que depende da categoria). Importante que os pontos considerados são os pontos que foram concedidos sem ajuda de algum erro ou bolas passadas (não agarradas pelo catcher)}
%% \longnewglossaryentry{FA }{ name=\textit{FA },plural=\textit{FA },}{ Fastball (Bola rápida): é um arremesso direto e rápido}
%% \longnewglossaryentry{Fair Ball }{ name=\textit{Fair Ball },plural=\textit{Fair Ball },}{ Bola Válida: é uma bola rebatida que autoriza o batedor a tentar alcançar a primeira base}
%% \longnewglossaryentry{Fasto}{ name=\textit{Fasto},plural=\textit{Fasto},}{ é o termo japonês para \textit{first}, geralmente designa o defensor que joga na posição F3 ou na primeira base}

%% \longnewglossaryentry{FLD\% }{ name=\textit{FLD\% },plural=\textit{FLD\% },}{ Fielding Percentage: indica quão frequentemente um defensor ou até um time consegue realizar jogadas corretas ao receber bolas arremessadas para realizar eliminações. Geralmente tem fórmula que a descreve: número total de putouts e assistências feitas por um defensor dividido pelo numero total de chances (putouts assists e erros)}
%% \longnewglossaryentry{Flex}{ name=\textit{Flex},plural=\textit{Flex},}{ jogador que pode jogar nas posições de defesa. Geralmente faz par com Jogador designado JD ou DP ("Designated Play").
%% \longnewglossaryentry{Foul Ball (Bola Inválida)}{ name=\textit{Foul Ball (Bola Inválida)},plural=\textit{Foul Ball (Bola Inválida)},}{ é uma bola rebatida que não autoriza o batedor a tentar alcançar a primeira base.
%% \longnewglossaryentry{Furay }{ name=\textit{Furay },plural=\textit{Furay },}{ é o termo japonês para \textit{fly} ou "bola aérea".
%% \longnewglossaryentry{GDP }{ name=\textit{GDP },plural=\textit{GDP },}{ Ground into Double Play: ocorre quando um defensor agarra uma bola rasteira e consegue efetuar jogadas eliminando dois ou mais jogadores nas bases.

%% \longnewglossaryentry{Gorô}{ name=\textit{Gorô},plural=\textit{Gorô},}{ é o termo japonês para \textit{ground} ball, ou bola rasteira, no chão}
%% \longnewglossaryentry{Ground Ball (Gorô)}{ name=\textit{Ground Ball (Gorô)},plural=\textit{Ground Ball (Gorô)},}{ uma bola rebatida ao chão que vai rolando aos defensores}
%% \longnewglossaryentry{HBP }{ name=\textit{HBP },plural=\textit{HBP },}{ Hit by Pitch: acontece quando um arremessador acerta voluntária ou involuntariamente o batedor, fora da zona de strike, desde que este não realize a movimentação de batida com o bastão. Os \textit{strikes} se sobrepõem ao Hit by Pitch}
%% \longnewglossaryentry{Hit (Batida)}{ name=\textit{Hit (Batida)},plural=\textit{Hit (Batida)},}{ acontece quando o batedor consegue acertar a bola e esta cai em território válido, dentro do campo.

%% \longnewglossaryentry{Home Plate}{ name=\textit{Home Plate},plural=\textit{Home Plate},}{ formalmente designada nas regras como \textit{home}, é a base final que um jogador deve tocar para marcar ponto}
%% \longnewglossaryentry{Home run}{ name=\textit{Home run},plural=\textit{Home run},}{ jogada em que o rebatedor lança a bola para fora do limite do campo, acima da cerca de proteção, sem que esta toque no chão. Desta forma, o jogador é capaz de percorrer as quatro bases numa mesma corrida, marcando um ponto para a sua equipe}
%% \longnewglossaryentry{HR }{ name=\textit{HR },plural=\textit{HR },}{ Home Run ocorre quando um batedor consegue acertar uma bola para fora de campo, ou correr as quatro bases sem ser eliminado}
%% \longnewglossaryentry{IP }{ name=\textit{IP },plural=\textit{IP },}{ Illegal Pitch: ocorre quando o arremessador efetua alguma manobra proibida antes, durante ou após o arremesso. Existem várias possibilidades para um arremesso ilegal e o árbitro deve estar atento a regra}


%% \longnewglossaryentry{Kuniguê}{ name=\textit{Kuniguê},plural=\textit{Kuniguê},}{ é aquela jogada que no terceiro strike (termo japonês isturaiku) o catcher deixa a bola escapar e o corredor corre para a primeira base. Nota kuiniguê é comer e fugir (sem pagar a conta)}
%% \longnewglossaryentry{Lana}{ name=\textit{Lana},plural=\textit{Lana},}{ é o \textit{runner}, é o corredor}
%% \longnewglossaryentry{Line Ball}{ name=\textit{Line Ball},plural=\textit{Line Ball},}{ uma bola rebatida em linha reta para dentro do terreno de jogo}
%% \longnewglossaryentry{Obstruction (Obstrução)}{ name=\textit{Obstruction (Obstrução)},plural=\textit{Obstruction (Obstrução)},}{ é considerado o ato de um defensor que não está de posse da bola ou no processo de pegá-la impede o avanço de qualquer corredor}
%% \longnewglossaryentry{Passed Ball}{ name=\textit{Passed Ball},plural=\textit{Passed Ball},}{ bola defensável que passa para trás do receptor}
%% \longnewglossaryentry{PB }{ name=\textit{PB },plural=\te}xtit{ PB },}{ Passed Ball (Catcher): é um termo estatístico que determina a quantidade de bolas lançadas ao catcher que ele não consegue segurar e como resultado deste erro um jogador consegue avançar uma base.
%% \longnewglossaryentry{Pickle: sanduíche }{ name=\textit{Pickle: sanduíche },plural=\textit{Pickle: sanduíche },}{ é quando um corredor fica correndo entre bases para evitar ser tocado por um defensor que está de posse da bola | a rundown}
%% \longnewglossaryentry{Pickoff Play}{ name=\textit{Pickoff Play},plural=\textit{Pickoff Play},}{ jogada em que o arremessador tenta segurar o corredor na base, ou eliminar o corredor que está fora da base}
%% \longnewglossaryentry{Pitcher}{ name=\textit{Pitcher},plural=\textit{Pitcher},}{ arremessador, jogador da defesa que faz os lançamentos ao batedor}

%% \longnewglossaryentry{R }{ name=\textit{R },plural=\textit{R },}{ Run (Corridas ou Pontos): ocorre quando o corredor consegue cruzar a base principal (home) e marcar um ponto}
%% \longnewglossaryentry{RBI }{ name=\textit{RBI },plural=\textit{RBI },}{ Run Batted In: é creditado estatísticamente a um batedor na maioria dos casos nos quais ele vai aparecer no bater box pra bater e pelo menos um ponto é conseguido. Existem algumas exceções: por exemplo, ele não ganha um RBI quando o ponto for resultado de um erro de um defensor}
%% \longnewglossaryentry{Rise}{ name=\textit{Rise},plural=\textit{Rise},}{ é uma bola de efeito na qual o arremessador da um efeito de giro na bola fazendo com que a mesma "suba" na hora que o batedor iria efetuar a batida, provocando o erro dele}
%% \longnewglossaryentry{Runner (Corredor/Lana)}{ name=\textit{Runner (Corredor/Lana)},plural=\textit{Runner (Corredor/Lana)},}{ nomenclatura usada para denominar o rebatedor que, depois de bater na bola, chega salvo a uma das bases (seja ela a primeira, a segunda ou a terceira). Sua nova função consiste em correr para alcançar o maior número possível de bases enquanto o seu time estiver no ataque (rebatendo a bola)}

%% \longnewglossaryentry{Sanchin}{ name=\textit{Sanchin},plural=\textit{Sanchin},}{ é o termo japonês para \textit{strike out}, ou o ato de um batedor ser eliminado pelo terceiro strike virado no qual não acerta a bola}
%% \longnewglossaryentry{SB }{ name=\textit{SB },plural=\textit{SB },}{ Stolen Base: número de vezes que um jogador consegue roubar uma base}
%% \longnewglossaryentry{SBA }{ name=\textit{SBA },plural=\textit{SBA },}{ Stolen Bases Allowed (Pitcher and Catcher): número de vezes que um jogador conseguiu "avançar bases" sem ser eliminado e sem que houvesse uma jogada ocorrendo (bases roubadas)}
%% 
%% \longnewglossaryentry{SF }{ name=\textit{SF },plural=\textit{SF },}{ Sacrifice Fly: ocorre quando um batedor acerta uma batida com bola aérea para fora do campo interno, que pode ser facilmente pega por um defensor, mas que permite um corredor marcar ponto (após retocar a base depois da catada)}
%% \longnewglossaryentry{SH }{ name=\textit{SH },plural=\textit{SH },}{ \textit{Foul tip}}: ocorre quando um jogador consegue marcar pontos por uma batida no campo interno ("Bunt") feita por um batedor com este intuito.
%% \longnewglossaryentry{SHO }{ name=\textit{SHO },plural=\textit{SHO },}{ Shutout: um arremessador é premiado com um \textit{Shutout} quando ele entra para arremessar e arremessa o jogo inteiro pelo seu time sem permitir que adversário consiga marcar nenhum ponto}

%% \longnewglossaryentry{SO }{ name=\textit{SO },plural=\textit{SO },}{ Strike Out: ocorre quando o arremessador consegue eliminar o batedor por uma combinação de 3 viradas de bastão ou 3 strikes determinados pelo árbitro de \textit{home}}
%% \longnewglossaryentry{SO }{ name=\textit{SO },plural=\textit{SO },}{ Strike Out: representa a quantidade de vezes que o batedor foi eliminado pelo 3 strike (quer seja apenas olhando ou virando o bastão).
%% \longnewglossaryentry{Southpaw}{ name=\textit{Southpaw},plural=\textit{Southpaw},}{ é um arremessador canhoto}
%% \longnewglossaryentry{Squeeze Play}{ name=\textit{Squeeze Play},plural=\textit{Squeeze Play},}{ é uma manobra que consiste em um sacrifício com um corredor na terceira base. O batedor bate na bola, esperando ser eliminado na primeira base, mas proporcionando ao corredor na terceira base uma oportunidade para marcar um ponto}
%% \longnewglossaryentry{Strike}{ name=\textit{Strike},plural=\textit{Strike},}{ o mesmo que "bola boa". É o arremesso válido feito pela defesa, que não é rebatido pelo ataque}


%% \longnewglossaryentry{TB }{ name=\textit{TB },plural=\textit{TB },}{ \textit{Total Bases}: quantidade de bases conquistadas por um batedor através de suas batidas}

%\longnewglossaryentry{fair}{
%	name=\textit{\textit{fair}},
%	plural=\textit{fairs},}
%	{ é definido como a área do campo de jogo entre as duas linhas laterais que definem o campo de jogo, e inclui as próprias linhas e os postes delimitadores}

%% \longnewglossaryentry{Terreno \textit{FOUL}}{ name=\textit{Terreno \textit{FOUL}},plural=\textit{Terreno \textit{FOUL}},}{ é definido como qualquer área fora do campo de jogo}

%% \longnewglossaryentry{Wild Pitch (arremesso descontrolado)}{ name=\textit{Wild Pitch (arremesso descontrolado)},plural=\textit{Wild Pitch (arremesso descontrolado)},}{ é quando um jogador da defesa joga uma bola totalmente "torta" para outro defensor que não consegue pegá-la}
%% \longnewglossaryentry{Zona de Strike}{ name=\textit{Zona de Strike},plural=\textit{Zona de Strike},}{ para que um arremesso seja considerado válido, a bola precisa se manter na chamada zona de strike, um retângulo imaginário de mais ou menos 35 centímetros de largura e cuja altura se mede do joelho até axilas do rebatedor. A bola arremessada fora desta área é considerada "ruim". A análise dos arremessos é feita por um juiz que fica posicionado atrás da receptora do time que está defendendo (e do rebatedor)}
%% \longnewglossaryentry{Appeal Plays (Jogadas de Apelação)}{ name=\textit{Appeal Plays (Jogadas de Apelação)},plural=\textit{Appeal Plays (Jogadas de Apelação)},}{ O time na defensiva tem direito a apelar de algumas jogadas que não tiveram a regra correta aplicada por uma árbitro no entendimento do técnico. Estas apelações servem para alertar o árbitro de infrações que poderiam ser permitidas sem serem apeladas. Nota. Não existem apelações para jogadas de decisão (Ball, strike, safe, out, foul, fair)}
%
%

\newglossaryentry{lineup}{
	name=\textit{line up},
	description={Formulários de Escalação da Equipe}
}

\longnewglossaryentry{diamond}{
	name=\textit{diamond},
	plural=\textit{diamonds},
	description={ diamante ou campo de jogo. Campo interno}}

\longnewglossaryentry{hito}{
	name=\textit{hito},
	plural=\textit{hitos},}{ é o termo japonês para \textit{hit}}

\longnewglossaryentry{hit}{
	name=\textit{hit},
	plural=\textit{hits},}{ou batida, acontece quando o batedor consegue acertar a bola e esta cai em
	território válido, dentro do campo}

\longnewglossaryentry{globo}{ name=\textit{Globo},plural=\textit{Globos},}{ é o termo japonês para \textit{glove} (luva)}

\longnewglossaryentry{glove}{ name=\textit{glove},plural=\textit{gloves},}{ Luva (Globo)}

\longnewglossaryentry{suberi}{ name=\textit{suberi},plural=\textit{suberis},}{ é o termo japonês para \gls{slide}}


\longnewglossaryentry{slide}{ name=\textit{slide},plural=\textit{slides},}{é quando o corredor tenta entrar na base escorregando com a perna esticada e tentando se esquivar de um possível toque do defensor da base}

 \longnewglossaryentry{putout}{ name=\textit{put out},plural=\textit{put outs},}{ um defensor é creditado com um \textit{put out} quando ele, fisicamente, consegue eliminar um jogador do time adversário (quer seja por tocá-lo com a bola ou pisando uma base com a posse da bola em jogadas forçadas, ou até mesmo catando um terceiro strike). Também usado em estatísticas de jogo}

\longnewglossaryentry{basehit}{ name=\textit{base hit},plural=\textit{base hits},}{ batidas indefensáveis}

% \longnewglossaryentry{homeplate}{ name=\textit{home plate},}{formalmente designada nas regras como \gls{home}, é a base final que um jogador deve tocar para marcar ponto}

 \longnewglossaryentry{firstbasecoach}{ name=\textit{First-base Coach},}{
 é o técnico ou jogador que fica na área de técnico ao lado da primeira base, e geralmente sinaliza jogadas aos corredores e batedores para orientá-los}

\longnewglossaryentry{jogadaforcada}{
	name={jogada forçada},plural={ jogadas forçadas},}
	{ é uma situação em que um corredor de base é compelido (ou forçado, obrigado) a desocupar sua base por conta de outro corredor que está chegando, e assim tentar avançar a salvo para a próxima base}


\newglossaryentry{lana}
{
	name={\textit{lana}},
	description={é o \gls{runner}, é o corredor}
}

\longnewglossaryentry{runner}
{
	name={\textit{runner} (corredor/\textit{Lana})},
	plural={\textit{runners} (corredores/\textit{Lanas})},
}{
nomenclatura usada para denominar o rebatedor que, depois de bater na bola, chega salvo a uma das bases (seja ela a primeira, a segunda ou a terceira). Sua nova função consiste em correr para alcançar o maior número possível de bases enquanto o seu time estiver no ataque (rebatendo a bola).
}


\newglossaryentry{safehit}
{
	name={\textit{safe hit}},
	plural={\textit{safe hits}},
		description={ batida por meio da qual o batedor-corredor chega a salvo à primeira base}
}

\newglossaryentry{shutout}
{
	name={\textit{shutout}},
	plural={\textit{shutouts}},
	description={um arremessador é premiado com um \textit{shutout} quando ele entra para arremessar e arremessa o jogo inteiro pelo seu time sem permitir que adversário consiga marcar nenhum ponto}
}

\longnewglossaryentry{tripleplay}{ 
 	name=\textit{Triple Play},
 	plural=\textit{Triple Plays},}
 	{ é o ato raro de fazer três eliminações durante durante a mesma jogada contínua. Um \textit{Foul tip} agarrado pelo catcher é considerado um 3o strike, portanto conta como\textit{strike out}}
 
\longnewglossaryentry{windup}{ 
	name=\textit{wind up},
	plural=\textit{wind ups},}
{ é um movimento preparatório para o arremesso no lançamento rápido no softbol. A conclusão é em um ponto específico que a bola é liberada no arremesso. A parte superior do corpo deve permanecer na posição vertical, em vez de curvada. O braço de arremesso do arremessador começa no quadril. (Alguns arremessadores movem o braço de arremesso para trás enquanto transferem seu peso para trás). A partir do quadril, o braço arremessador move-se para cima em círculo, roçando a orelha e retornando ao quadril antes do lançamento. O braço do arremessador deve permanecer firme para manter o controle do arremesso, isso pode ser feito porque a parte inferior do corpo do arremessador está apoiado e orientado em uma linha reta.}

\longnewglossaryentry{shortstop}{ 
	name=\textit{short stop},
	plural=\textit{short stops},}
	{ é o termo japonês para \textit{short Stop}. Geralmente designa o defensor que joga na posição F6 ou na interbases (entre a segunda e terceiras bases)}

\longnewglossaryentry{Shotto}{ name=\textit{Shotto},plural=\textit{Shotto},}{ é o termo japonês para \textit{short Stop}. Geralmente designa o defensor que joga na posição F6 ou na interbases (entre a segunda e terceiras bases)}

\longnewglossaryentry{misplay}{ name=\textit{misplay},plural=\textit{misplays},}{
é uma má jogada ou erro de um time defensivo que prolonga o tempo do batedor ou o tempo na base de um corredor}

\newglossaryentry{ESC}{name={\textit{Equipment Standards Comission}},description={Comissão de Equipamento Padrão }}

\newglossaryentry{stolenbase}{
	 name=\textit{stolen base},
	 plural=\textit{stolen bases},
		description={Uma base roubada ocorre quando um corredor de base avança tomando uma base à qual ele não tem direito. Isso geralmente ocorre quando 
			\begin{itemize}
				\item um arremessador está lançando um arremesso, ou 
				\item o arremessador ainda está em posse da bola ou está tentando uma eliminação, ou 
				\item o receptor está jogando a bola de volta para o arremessador.
			\end{itemize}}
		}

\longnewglossaryentry{first}{ name=\textit{first},plural=\textit{firsts},}{base Coach: é o técnico ou jogador que fica na área de técnico ao lado da primeira base, e geralmente sinaliza jogadas aos corredores e batedores para orientá-los}

\longnewglossaryentry{sado}{ name=\textit{sado},}{ é o termo japonês para \textit{third}. Geralmente designa o defensor que joga na posição F5 ou na terceira base}

\longnewglossaryentry{secano}{ name=\textit{Secano},plural=\textit{Secano},}{ é o termo japonês para \textit{second}. Geralmente designa o defensor que joga na posição F4 ou na segunda base}

\longnewglossaryentry{second}{ name=\textit{second},plural=\textit{second},}{ é uma posição de \textit{fielding} no campo interno, entre a segunda e a primeira base. O segundo base muitas vezes possui mãos e pés rápidos, precisa da capacidade de se livrar da bola rapidamente, e deve ser capaz de fazer o pivô em uma jogada dupla. No beisebol e softball, o segundo base, abreviado 2B}

\newglossaryentry{third}{
	name=\textit{third},
	description={é usado como termo para designar tanto o defensor da 3ª base –\textit{third baseman} (saard béismaen) -- como o local onde está colocada a almofada da 3ª base –\textit{third base} (saard béis)}
}

 \longnewglossaryentry{borbaco}{ name=\textit{Bor Baco},plural=\textit{Bor Bacos},}{ é o termo japonês para \gls{ballback} ou a informação dada pelo árbitro para recolherem as bolas, pois o jogo irá iniciar}
 
\newglossaryentry{ballback}{name=\textit{ball back},description={ é a informação dada pelo árbitro para recolherem as bolas, pois o jogo irá iniciar}} 


\newglossaryentry{DP}{
	name=\textit{Designated Player},
	description={ é o jogador colocado na lista de jogadores que serve de batedor no lugar do FLEX, que geralmente é colocado na lista como sendo o 10° jogador. Abreviatura: DP}
}

\newglossaryentry{flex}{
	name=\textit{flex},
	description={é o jogador que pode jogar nas posições de defesa. Geralmente faz par com \gls{DP} (JD ou DP)}
}

\newglossaryentry{fasto}{
	name=\textit{fasto},
	description={ Fasto: é o termo japonês para \gls{first}, geralmente designa o defensor que joga na posição F3 ou na primeira base.}
}		

\longnewglossaryentry{nottopitch}{
		name=\textit{not to pitch},
		description={Existem alguns pontos de ênfase que regem esses sinais. 
		
		O primeiro sinal de “não arremessar” é transmitido ao arremessador somente se ele estiver na placa de arremesso e prestes a arremessar rapidamente o batedor. O sinal é uma mão aberta apontada para o rosto do arremessador com os dedos bem abertos. Você DEVE sempre usar a mão que está MAIS LONGE do batedor (em direção à caixa vazia do batedor). Se um arremessador constantemente apressar você ou o batedor, aconselhe o \gls{catcher} a diminuir a velocidade do sinal para o arremessador. Se isso não funcionar, dê um passo à frente e explique ao arremessador que eles estão acelerando o rebatedor.
	
		O segundo exemplo para o sinal de ‘não arremessar’ é usado se você não estiver pronto para trabalhar ou se tiver concedido \gls{time} ao batedor antes que o arremessador tenha quebrado a pausa.
	
	Lembre-se de que um batedor pode solicitar tempo, mas o árbitro de prato não precisa conceder isso.
	
	Este pedido nunca deve ser concedido se o arremessador separou as mãos ao iniciar o arremesso. 
	
	Chame o arremesso de \gls{strike} ou \gls{ball} dependendo se passou ou não pela zona de strike. 
	
	Se o sinal de "não lançar" estiver ativo e o arremessador lançar de qualquer maneira (ou arremessos enquanto \gls{time} foi chamado), saia de trás do \gls{catcher} em direção à caixa aberta do batedor e chame “TIME – NO PITCH”.

	Duas mãos estão no ar para o sinal de bola morta. Espere até que seja seguro voltar, coloque sua máscara e vá atrás do apanhador mais uma vez}
}

\longnewglossaryentry{runaheadrule}{name=\textit{run ahead rule}}{é uma regra usada nos desportos para poupar que uma equipe muito inferior a seu oponente seja "humilhada" (ou para evitar à equipe vencedora a satisfação do mesmo), quando o adversário consegue uma vantagem de pontuação muito grande e presumivelmente insuperável sobre a outra, terminando a partida antes do ponto final/tempo programado.

\begin{center}
	\begin{tabular}{p{35mm}|p{21mm}*{3}{|p{20mm}}}\hline
		Categoria & \multirow{2}{*}{\parbox{20mm}{número de \glspl{inning}}} & 
		\multicolumn{3}{c}{Called game - diferença de pontos}\\\cline{3-5}
		& & 15 & 10 & 7 \\[1mm]\hline
		sub 13 &\multirow{2}{*}{6} &\multirow{4}{*}{3}&\multirow{2}{*}{4 ou 5} &\multirow{2}{*}{n/a} \\
		sub 15 &  &&& \\\cline{1-2}\cline{4-5}
		sub 17 &\multirow{2}{*}{7}& & \multirow{2}{*}{4} &\multirow{2}{*}{ 5 ou 6} \\
		demais categorias &&&\\\hline
	\end{tabular}	
\end{center}

}

\newglossaryentry{wildpitch}{
	name=\textit{wild pitch},
	plural=\textit{wild pitches},
	description={ Wild Pitch (arremesso descontrolado): é quando um jogador da defesa joga uma bola totalmente "torta" para outro defensor que não consegue pegá-la.}}

\newglossaryentry{hitbypitch}{name={\textit{Hit by Pitch}},
	description={ acontece quando um arremessador acerta voluntária ou involuntariamente o batedor, fora da zona de strike, desde que este não realize a movimentação de batida com o bastão. Os \glspl{strike} se sobrepõem ao \textit{Hit by Pitch}. Abreviatura: HBP}}


\newglossaryentry{infieldputout}{
	name={\textit{infield put out}},
	description={um \gls{putout} executado por um membro do campo interno. Abreviatura: IPO}
}


\newglossaryentry{runbattedin}{
	name={\textit{run batted in}},
	description={ é uma estatística no beisebol e softball que credita um batedor por fazer uma jogada que permite que uma corrida seja marcada (exceto em certas situações, como quando um erro é cometido na jogada). Por exemplo, se o batedor bater uma base que permite que um companheiro de equipe em uma base mais alta chegue em casa e assim marcar uma corrida, então o batedor é creditado com um RBI. Abreviatura: RBI}
}

\newglossaryentry{sacrificefly}{
	name={\textit{sacrifice fly}},
	description={ Uma jogada de sacrifício ocorre quando um batedor bate uma bola \gls{fly} para fora do campo ou território \gls{foul} que permite que um corredor marque ponto. 
		O batedor recebe crédito por um RBI. (Se a bola é lançada por um erro, mas é determinado que o corredor teria marcado com uma captura, então o batedor ainda é creditado com uma jogada de sacrifício.) Uma jogada de sacrifício não conta como um rebatida e, portanto, não conta contra a média de rebatidas de um jogador. 	
		O pensamento por trás da regra é que com uma corredora na terceira base e menos de dois \glspl{out}, um batedor muitas vezes tentará intencionalmente acertar uma bola voadora, sacrificando seu tempo na batida para ajudar a marcar uma corrida. No entanto, as jogadas de sacrifício contam contra a porcentagem de um jogador na base}
}

\newglossaryentry{fielder'schoice}{
	name={\textit{fielder's choice}},
	description={A escolha de um defensor é uma jogada em que um defensor tenta fazer uma jogada em uma base diferente da primeira base em uma bola de chão. Ao marcar um jogo, a escolha de um defensor. Abreviatura: FC}
}

\newglossaryentry{sain}{
	name={\textit{sain}},
	description={Vem de \textit{sign} (sain), que quer dizer sinal, gesto etc. 
	Dizemos que um técnico ou \gls{coach} está dando \textit{sain} quando ele está fazendo uma série de sinais (gestos) para combinar jogadas}
}


\newglossaryentry{sanshin}{
	name={\textit{sanshin}},
	description={É um termo japonês. Significa eliminação por três \glspl{strike}. Ao pé da letra \textit{sanshin} quer dizer girar (o \gls{bat}) três vezes. ‘sanshin’ = \gls{strike out} (straik aut). Os japoneses costumam dizer que o batedor comeu ‘sanshin’ (‘sanshin o kutta’, ‘sanshin o kuratta’) quando ele é eliminado por três \glspl{strike}. É por isso que o pessoal diz: "VAI COMER!", quando o árbitro de "home" declara o segundo \gls{strike} a um batedor; e: "COMEU!", quando um batedor não consegue rebater o terceiro \gls{strike}}
}

\newglossaryentry{three-base hit}{
	name={\textit{three-base hit}},
	description={ é uma batida que permite a batedora alcançar a terceira base}
}

\newglossaryentry{two-base hit}{
	name={\textit{two-base hit}},
	description={ é uma batida que permite a batedora alcançar a segunda base}
}

\newglossaryentry{totalbase}{
	name={\textit{total base}},
	plural={\textit{total bases}},
	description={ refere-se ao número de bases adquiridas por um batedor através de seus \glspl{hit}. 
		Um batedor registra uma base total para uma única base, duas bases totais duplo, três bases para um triplo e quatro bases totais para um home run. 
		Bases totais são usadas para determinar a porcentagem de \textit{slugging} de um jogador - que é bases totais divididas por \textit{at-bats}. 
		Um jogador só pode adicionar à sua contagem total através de um \gls{hit}. 
		Avançar nos caminhos de base, mesmo por roubo, não tem impacto nas bases totais de um jogador.
	Abreviatura: TB}
}


%
%‘SANRUIDA’ (o R não tem o som áspero): É um termo japonês. Significa rebatida de três bases (‘SANRUI’ = três bases; ‘DA’ = rebatida). ‘SANRUIDA’ = “THREE-BASE HIT” (srii béis hit).
%
%‘sanshin’: É um termo japonês. Significa eliminação por três "strikes". Ao pé da letra ‘SANSHIN’ quer dizer girar (o "bat") três vezes. ‘SANSHIN’ = “STRIKE OUT” (straik aut). Os japoneses costumam dizer que o batedor comeu ‘sanshin’ (‘sanshin o kutta’, ‘sanshin o kuratta’) quando ele é eliminado por três "strikes". É por isso que o pessoal diz: "VAI COMER!", quando o árbitro de "home" declara o segundo "strike" a um batedor; e: "COMEU!", quando um batedor não consegue rebater o terceiro "strike".

