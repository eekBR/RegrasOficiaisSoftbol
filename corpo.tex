\chapter{O JOGO}

\begin{multicols}{2} 
	O jogo de softbol \'e disputado entre duas equipes em rodadas (denominadas \glspl{inning}), dentro de um campo e sujeito a regras espec\'ificas e que podem ser ajustadas de acordo com as condi\c{c}\~oes em que \' realizada a disputa, como veremos ao longo do texto. 
	
	Este texto \'e uma vers\~ao editada do regulamento oficial para pessoas leigas no assunto, e que gostariam de ter um grau de conhecimento dos fundamentos das regras do jogo. Pois a numera\c{c}\~ao dos t\'opicos n\~ao poderia ser alterada.

	
\end{multicols}


\section{DEFINI\c{C}\~OES}

\begin{description}
\item[Reuni\~ao pr\'e-jogo]: \'E a reuni\~ao na \'area do \gls{homeplate}, na hora predeterminada, incluindo o \'arbitro, \glspl{coach} principais, t\'ecnicos ou representantes de ambas as equipes. Nessa reuni\~ao, os Formul\'arios de Escala\c{c}\~ao da Equipe (Line up) s\~ao confirmados e trocados entre as equipes, e o \'arbitro de \gls{home} rev\^e algumas regras especiais. 


\item[Equipe local] pode ser definida de v\'arias maneiras, incluindo sorteio com moeda (\gls{coin toss}), acordo m\'utuo, indica\c{c}\~ao da competi\c{c}\~ao ou indica\c{c}\~ao da Liga;

A equipe local inicia o jogo na defensiva, bate (ataca) na segunda metade do \gls{inning} e ocupa o \Gls{dugout} do lado da terceira base; 

\item[Equipe visitante] inicia o jogo na ofensiva, bate (ataca) na primeira metade de cada \gls{inning} e ocupa o \Gls{dugout} do lado da primeira base. 

\item[Equipe na defensiva]: \'E a equipe cujos jogadores est\~ao ocupando suas posi\c{c}\~oes dentro do campo. 

\item[Equipe na ofensiva]: \'E a equipe que est\'a batendo (atacando). 

\item[\gls{inning}]: \'E aquela parte de um jogo durante a qual ambas as equipes atacam ou defendem, e continuam atacando ou defendendo at\'e que ocorram tr\^es elimina\c{c}\~oes. A nova metade de \gls{inning} inicia imediatamente após a \'ultima elimina\c{c}\~ao da metade de \gls{inning} anterior. 

\item[\Gls{play ball}]: Para iniciar ou reiniciar um jogo, o \'arbitro de \gls{home} deve sinalizar que a bola est\'a viva e declarar \gls{play ball}, desde que: 

\begin{enumerate}[label=\alph*)]
	\item o arremessador esteja segurando a bola dentro do C\'irculo do Arremessador; 

	\item  o receptor esteja dentro do \gls{catcher's box} e todos os outros defensores estejam 
em território \gls{fair}. 
\end{enumerate}

\item[Apela\c{c}\~ao -- bola viva ou bola morta]: Uma apela\c{c}\~ao com bola viva ou bola morta \'e uma jogada ou situa\c{c}\~ao em que um \'arbitro n\~ao pode dar uma decis\~ao, a menos que seja  solicitada por um t\'ecnico, \gls{coach} ou jogador da equipe n\~ao infratora. 

\item[\Gls{time}]\footnote{N.T.: TEMPO}: \'E o termo usado por um \'arbitro para ordenar a paralisa\c{c}\~ao de jogada num jogo. Durante a paralisa\c{c}\~ao de jogada a bola est\'a morta. 


\item[Protesto]: \'E a a\c{c}\~ao de uma equipe na defensiva ou ofensiva, exceto uma apela\c{c}\~ao, para contestar: 

\begin{enumerate}[label=\alph*)]
	\item a m\'a interpreta\c{c}\~ao ou aplica\c{c}\~ao de uma regra de jogo por um \'arbitro; ou 
	\item  a elegibilidade de um membro da lista da equipe. 

\end{enumerate}

\item[Confisco de jogo]: Quando o \'arbitro de \gls{home} encerra o jogo, declarando vencedora a equipe n\~ao infratora. 

\end{description}	

 
\section{JOGO REGULAMENTAR -- REQUISITOS}

\begin{multicols}{2}
\subsection{JOGO REGULAMENTAR}

Um Jogo Regulamentar consiste de sete \glspl{inning} completos.

Um jogo regulamentar não consiste de 7 \glspl{inning} completos quando: 

\begin{enumerate}[label=\alph*)]
	\item N\~ao \'e necess\'ario jogar os sete \glspl{inning} completos se a equipe local anota mais pontos em seis \glspl{inning} ou antes do terceiro \gls{out} na segunda metade do s\'etimo \gls{inning}. 

	\item  Um jogo que est\'a empatado depois de jogados sete \glspl{inning} deve ser prorrogado jogando \glspl{inning} adicionais at\'e que uma das equipes anote mais pontos do que a outra no fim de um \gls{inning} completo, ou at\'e que a equipe local obtenha vantagem no placar em sua metade de \gls{inning}, antes de ser completada o terceiro \gls{out}. 

	\item  Um jogo encerrado pelo \'arbitro antes do s\'etimo \gls{inning} ser\'a um Jogo Regulamentar: 
	\begin{itemize}
		\item  se cinco ou mais \glspl{inning} completos tiverem sido jogados, ou 
		\item se a equipe local tiver anotado mais pontos do que os anotados pela outra equipe em cinco ou mais \glspl{inning}, ou 
		\item se for aplicada a Regra de Vantagem de Pontos (\Gls{run ahead rule}). 
	\end{itemize}
	O \'arbitro tem poderes para encerrar um jogo a qualquer momento, por causa de escurid\~ao, chuva, inc\^endio, p\^anico ou outra causa que ponha os espectadores ou membros da equipe em risco. 

	\item  Ser\'a declarado um Jogo Regulamentar empatado se o placar for igual quando a partida \'e encerrada depois de jogados cinco ou mais \glspl{inning} completos, ou se a equipe local tiver anotado o mesmo n\'umero de pontos da equipe visitante no \gls{inning} incompleto.
	
	\item Estas disposi\c{c}\~oes n\~ao se aplicam a quaisquer atos da parte de jogadores e espectadores que possam dar motivo para o confisco do jogo. O \'arbitro de \gls{home} pode confiscar o jogo se qualquer membro de uma equipe ou espectador atacar fisicamente qualquer \'arbitro. 
	\item  Um jogo que n\~ao \'e considerado um Jogo Regulamentar ou \'e um jogo Regulamentar Empatado ser\'a jogado novamente desde o in\'icio. A escala\c{c}\~ao original pode ser alterada para o novo jogo. 
\end{enumerate}

\subsection{JOGO CONFISCADO}
Um jogo \'e confiscado a favor da equipe n\~ao infratora quando: 

\begin{enumerate}[label=\alph*)]
	\item Uma equipe n\~ao comparece ao campo. 

	\item  Uma equipe que est\'a no campo se recusa a iniciar um jogo para o qual est\'a escalada ou designada, no hor\'ario marcado -- ou dentro de um tempo estabelecido para confisco de jogo -- pelo Regulamento da competi\c{c}\~ao em que ela est\'a jogando. 

	\item  Após iniciado o jogo, uma equipe se recusa a continuar jogando, a menos que o jogo tenha sido interrompido ou encerrado pelo \'arbitro de \gls{home}. 

	\item  Depois que o \'arbitro de \gls{home} paralisa o jogo, uma equipe n\~ao reinicia a partida dentro de dois minutos após o \'arbitro sinalizar e declarar \gls{play ball}. 

	\item  Uma equipe emprega t\'aticas destinadas a retardar ou acelerar o jogo. 

	\item  Após uma advert\^encia do \'arbitro, qualquer uma destas regras \'e violada propositadamente, exceto se o arremessador continua infringindo, reiteradamente, a regra sobre arremesso; nesse caso, ele ser\'a removido da posi\c{c}\~ao de arremessador para o resto do jogo. 

	\item  A ordem para remover ou expulsar um jogador ou qualquer pessoa autorizada a permanecer no \gls{bench} da equipe n\~ao \'e obedecida dentro de um minuto. 

	\item  Por causa da remo\c{c}\~ao ou expuls\~ao de jogador (es) do jogo pelo \'arbitro, ou por qualquer outro motivo, h\'a menos de nove (dez com JD) jogadores em qualquer das duas equipes. 

	\item  Um jogador \textsl{declarado ineleg\'ivel} retorna ao jogo e o arremessador efetua um arremesso. 

	\item  \'E descoberto que um jogador, \gls{coach} ou t\'ecnico expulso est\'a participando do jogo outra vez. 
\end{enumerate}

\subsection{REGRA DE VANTAGEM DE PONTOS}

\begin{enumerate}[label=\alph*)]
	\item Aplica-se a qualquer jogo em todos os Torneios e Campeonatos quando uma equipe est\'a liderando o placar por: 
	\begin{itemize}
		\item 15 pontos depois de tr\^es (3) \glspl{inning}; 
		\item 10 pontos depois de quatro (4) \glspl{inning}; ou 
		\item sete (7) pontos depois de cinco (5) \glspl{inning}. 
	\end{itemize}


	\item  S\~ao jogados \glspl{inning} completos, a menos que a equipe local anote a quantidade necess\'aria de pontos enquanto est\~ao na ofensiva. Quando a equipe visitante alcan\c{c}a a quantidade necess\'aria de pontos na primeira metade de \gls{inning}, a equipe local bate na segunda metade do \gls{inning}. 
	
	Toda a jogada tem de estar conclu\'ida antes de o jogo ser declarado vencido pela contagem da Regra de Vantagem de Pontos. 
	
	Na segunda metade do \gls{inning}, nenhum tento acima da contagem necess\'aria para aplicar a Regra de Vantagem de Pontos deve ser contado, a menos que seja batido um \gls{homerun}, e nesse caso todos os pontos anotados s\~ao contados. 
\end{enumerate}

\subsection{REGRA DE DESEMPATE} 

\begin{enumerate}[label=\alph*)]
	\item A partir da primeira metade do oitavo \gls{inning} e de cada metade de \gls{inning} da\'i em diante at\'e o t\'ermino do jogo, a equipe na ofensiva inicia a sua vez de bater com o jogador a quem cabe bater em nono \footnote{NT: \'ultimo batedor a voltar a bater no momento} nessa respectiva metade de \gls{inning} colocado na segunda base como um corredor. 

	\item  O corredor colocado na segunda base pode ser substitu\'ido de acordo com as regras de substitui\c{c}\~ao. 
	
	\item  Se for colocado um corredor incorreto na segunda base, esse erro poder\'a ser corrigido t\~ao logo seja notada a falha. N\~ao h\'a penalidade. 
\end{enumerate}

\subsection{PONTOS ANOTADOS}

\begin{enumerate}[label=\alph*)]
	\item \'E anotado um ponto cada vez que um corredor toca, em ordem, as tr\^es bases e o \gls{homeplate}, antes do terceiro \gls{out} da metade de \gls{inning}. 

	\item  Quando se usa a regra de desempate, o corredor colocado na segunda base n\~ao precisa tocar a primeira base para que um ponto seja anotado legalmente. 

	\item  N\~ao \'e anotado um ponto se a terceira e/ou \'ultima elimina\c{c}\~ao do \gls{inning} resulta de: 
		\begin{enumerate}[label=\roman* -]
			\item uma jogada em que o batedor-corredor \'e declarado \gls{out} antes de tocar a primeira base; 
			\item uma Jogada For\c{c}ada, inclusive numa Jogada de Apela\c{c}\~ao; 
			\item uma jogada em que o corredor deixa uma base antes de ser efetuado um arremesso; ou 
			\item uma jogada em que um corredor precedente \'e declarado \gls{out}.
		\end{enumerate}
	\item Podem ser feitas apela\c{c}\~oes para uma elimina\c{c}\~ao adicional, depois do terceiro \gls{out}, para invalidar ponto (s). 
	
\end{enumerate}

\subsection{JOGADAS DE APELA\c{C}\~AO}
Numa Jogada de Apela\c{c}\~ao, o corredor ser\'a declarado \gls{out} somente se a apela\c{c}\~ao for feita legalmente. 

\begin{enumerate}[label=\alph*)]
	\item Uma apela\c{c}\~ao pode ser feita enquanto a bola est\'a viva ou morta, mas a equipe na defensiva perde a oportunidade de apelar se n\~ao se manifestar: 
	
	\begin{enumerate}[label=\roman* -]
		\item antes do próximo arremesso (legal ou ilegal), exceto em apela\c{c}\~ao sobre um Substituto Ilegal, Jogador N\~ao Anunciado, Reingresso Ilegal, Jogador de Emerg\^encia ou Jogador Removido e jogadores que mudam de posi\c{c}\~oes nas bases; 
		\item antes que todos os jogadores da defensiva tenham deixado o território \gls{fair}, a caminho do \gls{bench} ou da \'area do \Gls{dugout} (se um defensor faz a apela\c{c}\~ao, ele tem de estar no campo interno quando se manifesta ao \'arbitro); ou 
		\item no caso do \'ultimo lance do jogo, antes que os \'arbitros tenham deixado o campo de jogo. 
	\end{enumerate}
	\item  Os corredores podem deixar suas bases durante uma apela\c{c}\~ao com bola viva quando: 
	\begin{enumerate}[label=\roman* -]
		\item a bola deixa o C\'irculo do Arremessador; 
		\item a bola sai da m\~ao do arremessador; ou 
		\item o arremessador faz um movimento de lan\c{c}amento indicando uma jogada, ou simula um lan\c{c}amento. 
	\end{enumerate}

	\item  APELA\c{C}\~AO COM BOLA MORTA. Uma vez que a bola tenha sido devolvida ao campo interno e o \'arbitro tenha declarado \gls{time}, ou a bola tenha se tornado morta, qualquer membro da equipe na defensiva que esteja no campo interno, com ou sem a posse da bola, pode fazer uma apela\c{c}\~ao verbal sobre um corredor que tenha omitido uma base ou deixado uma base antecipadamente numa bola \gls{fly} pega no ar. 
	
	Um \gls{coach} ou t\'ecnico pode fazer uma apela\c{c}\~ao com bola morta somente após entrar no campo de jogo. O \'arbitro que recebe a apela\c{c}\~ao deve apreci\'a-la e dar a decis\~ao sobre a jogada. 
	
	Nenhum corredor pode deixar sua base durante esse per\'iodo, visto que a bola permanece morta at\'e o próximo arremesso. 
	
EXCE\c{C}\~AO: Um corredor que tenha deixado uma base antecipadamente numa bola \gls{fly} pega no ar, ou tenha omitido uma base, pode tentar retornar para tal base enquanto a bola est\'a morta. 

	\begin{enumerate}[label=\roman* -]
		\item Se a bola fica fora de jogo, a apela\c{c}\~ao com bola morta n\~ao pode ser feita at\'e que o \'arbitro de \gls{home} coloque uma nova bola no jogo. 
		\item Se o arremessador, de posse da bola, est\'a em contato com o \gls{pitcher's plate} quando faz uma apela\c{c}\~ao verbal, n\~ao deve ser declarado um Arremesso Ilegal. 
		\item Se o arremessador faz uma apela\c{c}\~ao depois da ordem \gls{play ball}, o \'arbitro deve declarar \gls{time} outra vez e permitir o processo de apela\c{c}\~ao. 
	\end{enumerate}
	\item  Apela\c{c}\~oes por uma elimina\c{c}\~ao adicional depois do terceiro \gls{out} s\~ao permitidas, desde que elas sejam feitas corretamente e com o objetivo de invalidar um ponto ou restabelecer a ordem de batedores correta. 

	\item  Estes s\~ao os tipos de apela\c{c}\~ao: 
	\begin{enumerate}[label=\roman* -]
		\item omiss\~ao de uma base; 
		\item o corredor deixa a sua base, num \gls{fly} pego no ar, antes de a bola ter sido tocada por um defensor; 
		\item batedor fora de ordem; 
		\item tentativa de avan\c{c}ar \`a segunda base após alcan\c{c}ar a primeira base; 
		\item substitui\c{c}\~oes ilegais; 
		\item o uso de um jogador n\~ao anunciado sob a Regra de Jogador de Emerg\^encia; 
		\item Reingresso Ilegal; 
		\item o uso de um jogador n\~ao anunciado sob a Regra de Jogador Designado; ou 
		\item corredores mudam de posi\c{c}\~oes nas bases que eles ocupavam. 
	\end{enumerate}
\end{enumerate}

\subsection{VENCEDORA DO JOGO}
A vencedora do jogo \'e a equipe que anota mais pontos do que a outra num Jogo Regulamentar. 

\begin{enumerate}[label=\alph*)]
	\item O placar de um Jogo Regulamentar encerrado antes do s\'etimo \gls{inning} \'e aquele registrado no fim do \'ultimo \gls{inning} completo, a menos que a equipe local anote mais pontos do que a equipe visitante na segunda metade do \gls{inning} incompleto. Nesse caso, o placar \'e aquele do \gls{inning} incompleto. 

	\item  O placar de um jogo regulamentar empatado \'e aquele quando o jogo foi encerrado. 

	\item  O placar de um jogo confiscado (\glspl{forfeited game}) \'e 7-0 a favor da equipe n\~ao infratora. 
	\end{enumerate}

\end{multicols}

\section{PROTESTOS}

\begin{multicols}{2}
Um protesto pode envolver tanto um assunto de aprecia\c{c}\~ao como a interpreta\c{c}\~ao de uma regra. 

\footnote{Exemplo de uma situa\c{c}\~ao desse tipo: 
\begin{center}
\begin{minipage}{.90\columnwidth}
	\color{black!80}
	
	Com um \gls{out} e corredor na segunda e na terceira bases, o batedor acertou um \gls{fair fly} (\gls{fly} para o território \gls{fair}) e foi declarado \gls{out}. 
	O corredor da terceira base saiu legalmente da base (fez \gls{tag up} depois que a bola foi pega), mas o da segunda base deixou a base antecipadamente.
	 O corredor da terceira base havia cruzado o \gls{homeplate} antes de a bola ser jogada \`a segunda base para completar o terceiro \gls{out}. 
	 O \'arbitro n\~ao validou o ponto. 
	 As d\'uvidas sobre: se os corredores deixaram suas bases antes da bola ser pega no ar e se a jogada na segunda base foi feita antes de o corredor da terceira base pisar o \gls{homeplate} s\~ao assuntos exclusivamente de aprecia\c{c}\~ao, e n\~ao podem ser protestados. A falha cometida pelo \'arbitro ao invalidar o ponto foi uma interpreta\c{c}\~ao errônea de uma regra de jogo; \'e uma decis\~ao suscet\'ivel 
	de protesto.
\end{minipage}
\end{center} 
}

\subsection{MOTIVOS PARA UM PROTESTO}

\begin{enumerate}[label=\alph*)]
	\item Um protesto que ser\'a recebido e considerado inclui assuntos como: 
	
	\begin{enumerate}[label=\roman* -]
		\item interpreta\c{c}\~ao errônea de uma Regra; 
		\item n\~ao aplica\c{c}\~ao da regra correta a uma determinada situa\c{c}\~ao por um \'arbitro; 
		\item n\~ao aplica\c{c}\~ao da penalidade correta a uma determinada infra\c{c}\~ao. 
	\end{enumerate}
	\item  Depois de efetuado um arremesso (legal ou ilega, nenhuma decis\~ao do \'arbitro pode ser mudada. 
	
	\item  A qualquer momento um protesto pode ser submetido \`a autoridade competente (que n\~ao seja o \'arbitro de \gls{home}) para verificar a situa\c{c}\~ao de um membro da rela\c{c}\~ao da equipe (se est\'a ou n\~ao eleg\'ivel para participar do jogo). 
\end{enumerate}


\subsection{PROTESTOS INV\'ALIDOS}
Nenhum protesto ser\'a recebido ou considerado se ele \'e baseado somente numa decis\~ao que implique na aprecia\c{c}\~ao por parte de um \'arbitro, ou se a equipe que est\'a protestando tiver vencido o jogo. Exemplos de protestos que n\~ao ser\~ao considerados: 

\begin{enumerate}[label=\alph*)]
	\item se uma bola batida foi \gls{fair} ou \gls{foul}; 

	\item  se um corredor foi \gls{safe} ou \gls{out}; 

	\item  se uma bola arremessada foi \gls{strike} ou \gls{ball};

	\item se um arremesso foi legal ou ilegal; 

	\item  se um corredor tocou ou n\~ao uma base; 

	\item  se um corredor deixou a base antecipadamente numa bola \gls{fly} pega no ar; 

	\item  se uma bola \gls{fly} foi ou n\~ao pega no ar; 

	\item  se uma bola \gls{fly} foi ou n\~ao um \gls{homeplate}; 

	\item  se houve ou n\~ao uma Interfer\^encia; 

	\item  se houve ou n\~ao uma Obstru\c{c}\~ao; 

	\item  se um jogador, ou uma bola viva, entrou ou n\~ao numa \'area de bola morta, ou tocou 
ou n\~ao um objeto ou pessoa em \'area de bola morta; 

	\item  se uma bola batida transpôs ou n\~ao, em voo, uma cerca; 

	\item  se o campo est\'a adequado para continuar ou recome\c{c}ar a partida; 

	\item  se h\'a ilumina\c{c}\~ao suficiente para continuar a partida; ou 

	\item  qualquer outro assunto que implique somente uma correta aprecia\c{c}\~ao do \'arbitro. 
	\end{enumerate}

\subsection{COMUNICA\c{C}\~AO DA INTEN\c{C}\~AO DE PROTESTAR}

\begin{enumerate}[label=\alph*)]
	\item Exceto para a desqualifica\c{c}\~ao de um jogador, o protesto tem de ser feito claramente ao \'arbitro, imediatamente antes do próximo arremesso, legal ou ilegal. O protesto no fim de um \gls{inning} tem de ser feito antes que todos os defensores deixem o território \gls{fair}, a caminho do \gls{bench} ou da \'area do \Gls{dugout}, e no \'ultimo lance do jogo, antes que os \'arbitros deixem o campo de jogo. 
	\item  A partir do momento em que \'e apresentado o protesto de acordo com esta Regra, o resto do jogo ter\'a prosseguimento sob protesto. 

	\item  O t\'ecnico ou o t\'ecnico interino da equipe reclamante pode comunicar isso ao \'arbitro de \gls{home}, e este tem de informar o t\'ecnico da equipe oponente e o anotador oficial. 

	\item  Todas as partes interessadas t\^em de ser informadas sobre as condi\c{c}\~oes que influ\'iram na tomada da decis\~ao, as quais ajudar\~ao na correta solu\c{c}\~ao do problema. 
	
 \end{enumerate}
 
\subsection{ÚLTIMO PRAZO PARA FORMALIZAR UM PROTESTO OFICIAL}

Um protesto por escrito oficial tem de ser formalizado dentro de um tempo razo\'avel. 

\begin{enumerate}[label=\alph*)]
	\item Na falta de uma regra da Liga ou da Competi\c{c}\~ao fixando um limite de tempo para formaliza\c{c}\~ao de um protesto, uma reclama\c{c}\~ao deve ser considerada se formalizada dentro de um tempo razo\'avel, dependendo da natureza do caso e da dificuldade para 
se obter a informa\c{c}\~ao em que \'e baseado o protesto. 

	\item  Geralmente, 48 horas após o tempo estabelecido para contesta\c{c}\~ao \'e considerado um tempo razo\'avel. 
\end{enumerate}

\subsection{PROTESTO POR ESCRITO FORMAL -- REQUISITOS}

Um protesto por escrito formal tem de conter a seguinte informa\c{c}\~ao para ser v\'alido: 

\begin{enumerate}[label=\alph*)]
	\item a data, o hor\'ario e o local do jogo; 

	\item  o (s) nome (s) do (s) \'arbitro (s) e anotador (es); 

	\item  as Regras Oficiais ou as regras locais sob as quais \'e feito o protesto; 

	\item  a decis\~ao e as condi\c{c}\~oes que influ\'iram na tomada da decis\~ao; e 

	\item  todos os fatos essenciais envolvidos no assunto protestado. 
\end{enumerate}

\subsection{RESULTADO DE PROTESTO}

A decis\~ao tomada sobre um jogo protestado tem de chegar a um dos seguintes resultados: 

\begin{enumerate}[label=\alph*)]
	\item O protesto \'e indeferido e o placar do jogo \'e mantido. 

	\item  Quando um protesto em raz\~ao de interpreta\c{c}\~ao errônea de uma Regra \'e reconhecido, o jogo deve recome\c{c}ar do ponto em que foi dada a decis\~ao incorreta, com essa decis\~ao corrigida. 

	\item  Quando um protesto sobre a situa\c{c}\~ao irregular de um membro da lista da equipe \'e reconhecido, o jogo deve ser confiscado (\gls{forfeited game}) a favor da equipe prejudicada. 
\end{enumerate}
\end{multicols}



\chapter{CAMPO DE JOGO}

	\section{O CAMPO DE JOGO}	

		\subsection{DEFINI\c{C}\~OES}
  
		\begin{description}
	
	\item[Campo de jogo]: \'E a \'area -- incluindo a linha de bola morta -- dentro da qual a bola pode ser jogada e pega;
	
	\item[Território \gls{fair}]: \'E aquela parte dentro do campo de jogo, incluindo as linhas de \gls{foul} da primeira e terceira base, que vai do \gls{homeplate} at\'e a parte inferior da cerca do campo externo e perpendicularmente para cima; 
	\item[Território \gls{foul}]: \'E qualquer parte do campo de jogo que n\~ao est\'a em Territ\'orio \gls{fair};
	
	\item[Linha de base]: \'E a linha reta entre duas bases consecutivas. 
	
	\item[\Gls{batter's box}]\footnote{ N.T.:\'area do batedor}: \'E a \'area dentro da qual o batedor deve permanecer enquanto est\'a posicionado com a inten\c{c}\~ao de bater o arremesso e ajudar a equipe na ofensiva a anotar pontos. As linhas s\~ao consideradas dentro do \Gls{batter's box}. 
	
	\item[\Gls{catcher's box}]\footnote{ N.T.: \'area do receptor}: \'E aquela \'area dentro da qual o receptor tem de permanecer at\'e o arremessador 
	completar o arremesso. As linhas s\~ao consideradas dentro do \gls{catcher's box}. 
	O receptor \'e considerado estar dentro do \gls{catcher's box}, exceto quando est\'a tocando o solo fora do \gls{catcher's box}. 
	
	\item[\Gls{dugout}]\footnote{N.T.: abrigo para membros da equipe}: \'E a \'area em territ\'orio de bola morta, destinada somente a membros da equipe. 
	\'E proibido fumar, consumir \'alcool ou usar fumo de mascar nessa \'area. O ato de fumar inclui a inala\c{c}\~ao de produtos do fumo, cigarros eletrônicos e \gls{vaping} (vaporiza\c{c}\~ao). 
	 
	
	\item[ \Gls{infield}]\footnote{N.T.: Campo interno}:	\'E a \'area do campo em territ\'orio \gls{fair} normalmente coberta por defensores do campo interno. 
	
	\item[\Gls{outfield}]\footnote{N.T.: Campo externo}: \'E aquela parte do campo de jogo em territ\'orio \gls{fair}, que est\'a al\'em do campo interno. 
	

	\end{description}

%\begin{multicols}{2} 
\subsection{O CAMPO OFICIAL} 

\begin{enumerate}[label=\alph*)]
	\item O leioute do campo oficial tem que obedecer \`as dimens\~oes e especifica\c{c}\~oes 
	estipuladas no \autoref{ap:Campo} e tem que incluir todas as caracter\'isticas mostradas [linhas de \gls{foul}, linha de um metro (3 p\'es), linhas laterais; \gls{coachsbox}, \gls{batter's box} e \gls{catcher's box}; c\'irculo do Batedor Prevenido e c\'irculo do arremessador; e bases e \gls{homeplate}]. 
	
	\item  Se durante o jogo for constatada incorre\c{c}\~ao na dist\^ancia entre bases ou na dist\^ancia do \gls{pitcher's plate}, o erro tem de ser corrigido no in\'icio do \gls{inning} seguinte, e o jogo deve recome\c{c}ar e continuar depois disso. 
\end{enumerate}

\subsection{CAMPO DE JOGO -- REQUISITOS }

\begin{enumerate}[label=\alph*)]
	\item O campo de jogo tem que ter uma \'area limpa e desobstru\'ida, dentro das dimens\~oes 
	m\'inimas estipuladas no \autoref{ap:Campo}, e tem que incluir todas as caracter\'isticas mostradas. 
	
	\item  \'E aconselh\'avel que o campo de jogo tenha uma zona de advert\^encia. Caso seja 
	usada uma zona de advert\^encia, ela tem que ser uma \'area dentro do campo de jogo e 
	pr\'oxima (adjacente) a qualquer cerca permanente ao longo do campo externo e das 
	linhas laterais. 
	
	\item  N\~ao h\'a nenhuma exig\^encia em providenciar uma zona de advert\^encia na superf\'icie permanente do campo externo (grama ou outro tipo de superf\'icie) quando se usa uma 
	cerca provis\'oria (isto \'e, quando um jogo da modalidade Arremesso R\'apido \'e realizado 
	num campo apropriado para jogo da modalidade Arremesso Lento). 
	
	\item  Uma bola est\'a \textit{fora do campo de jogo} quando ela toca o solo, uma pessoa no 
	campo ou um objeto, fora da \'area de jogo. 
\end{enumerate}

	\subsection{REGRAS DE CAMPO OU REGRAS ESPECIAIS }
Regras de Campo ou Regras Especiais estabelecendo os limites do campo de jogo 
podem ser combinadas antes do in\'icio de um jogo e usadas sempre que \gls{backstop}, 
cercas, arquibancadas, ve\'iculos, espectadores ou outros obst\'aculos est\~ao dentro da 
\'area prescrita. 

\begin{enumerate}[label=\alph*)]
	\item Qualquer obst\'aculo em territ\'orio \gls{fair} que se encontre a uma dist\^ancia menor do 	que as dist\^ancias m\'inimas das cercas demonstradas na Tabela de Dist\^ancias (\autoref{ap:Campo}) tem de estar claramente marcado para que os \'arbitros possam se orientar. 
	
	\item  Se usar um campo de beisebol, o mont\'iculo do arremessador ter\'a de ser removido, 
	e a dist\^ancia entre o \gls{homeplate} e o \gls{backstop}, ajustada \`a medida prescrita. 
\end{enumerate}
	
%\end{multicols}

\chapter{EQUIPAMENTO}
	\section{DEFINI\c{C}\~OES}

	
	\subsection{EQUIPAMENTO OFICIAL}
	Equipamento oficial \'e qualquer material de jogo (\glspl{bat}, luvas, capacetes etc.) que 
	est\'a sendo usado pela equipe na defensiva ou ofensiva durante o andamento do jogo. 
	
	Equipamento da defensiva (luvas, por exemplo) deixado no campo pela equipe que 
	est\'a atuando na ofensiva n\~ao \'e equipamento oficial. 


%	\se	ction{EQUIPAMENTO DE JOGO} 
	
	\subsection{\Gls{bat} OFICIAL }\label{bat-oficial}
	Somente um \gls{bat} oficial deve ser usado em uma Competi\c{c}\~ao da WBSC-SD ou ISF. 
	Para se identificar se um \gls{bat} oficial est\'a dentro dos padr\~oes da WBSC-SD ou ISF \textit{Equipment Standards Commission} basta verificar se est\'a estampado o logo da WBSC-SD ou ISF adotado e aprovado por \textit{Equipment Standards Commission} 
	\footnote{A Lista de \Gls{bat} Aprovado da WBSC-SD e o Logo Aprovado podem ser encontrados no Website da WBSC www.wbsc.org [Vide \autoref{sec:bat} (Especifica\c{c}\~oes do \Gls{bat}) para padr\~oes de \gls{bat} aprovado]}. 

	\subsection{BOLA OFICIAL}
	Somente uma bola oficial deve ser usada em Competi\c{c}\~ao da WBSC. 
	As bolas dentro dos padr\~oes da WBSC \textit{Equipment Standards Commission} t\^em estampada a marca –adotada e aprovada– da WBSC ou ISF \textit{Equipment Standards Commission}
	\footnote{Vide \autoref{ap:Bola} para padr\~oes de bola aprovada}. 
	
	\subsection{\textit{WARM-UP BAT} (\textit{BAT} PARA FAZER AQUECIMENTO)} 
	Somente um \gls{warm-up bat} que satisfaz as especifica\c{c}\~oes estabelecidas no \autoref{sec:bat} (Espe\-ci\-fi\-ca\-\c{c}\~oes do \Gls{bat}) para padr\~oes de \gls{warm-up bat} aprovado pode ser usado. 
	
	\subsection{EQUIPAMENTOS INADEQUADOS}
	\begin{enumerate}
		\item CAPACETE INADEQUADO:
			Um capacete que est\'a rachado, quebrado, amassado ou adulterado \'e um capacete ilegal, e deve ser removido do jogo. 
		
		\begin{enumerate}[label=\alph*)]
			\item O capacete para batedor prevenido, batedor, batedor-corredor e corredor deve ter duas orelheiras (uma em cada lado) e tem de ser do tipo que ofere\c{c}a seguran\c{c}a igual ou maior do que a proporcionada por capacete inteiramente de pl\'astico, com estofamento na parte interna. Um forro que cobre somente as orelhas n\~ao atende \`as especifica\c{c}\~oes de um capacete legal. 
			
			\item  O capacete para receptor ou jogador da defensiva pode ser do tipo \textit{coquinho}, sem orelheiras. 
		\end{enumerate}

	\item \Gls{bat} ILEGAL: 
	\'E um \gls{bat} que n\~ao atende \`as exig\^encias da Regra \ref{bat-oficial}. 

		\item \Gls{bat} ADULTERADO: 
		Um \gls{bat} est\'a Adulterado quando a estrutura f\'isica de um \gls{bat} legal foi modificada.
		
		\item LUVA ILEGAL:
		{\color{red!80}\'E uma luva que n\~ao atende \`as especifica\c{c}\~oes de uma luva legal, ou o uso de um \gls{mitt} por um defensor que n\~ao seja um receptor ou defensor da primeira base. }
		\footnote{
			Exemplos de altera\c{c}\~ao de um \gls{bat}: 
			\begin{itemize}
				\item substituir o cabo de um \gls{bat} de metal por um cabo de madeira ou outro tipo de cabo; 
				\item inserir material dentro do \gls{bat}; 
				\item aplicar fita adesiva em excesso (mais de duas camadas) no cabo do \gls{bat}; 
				\item pintar a superf\'icie superior ou inferior do \gls{bat} com outro prop\'osito que n\~ao o de identifica\c{c}\~ao;
				\item gravar uma marca de identifica\c{c}\~ao \gls{ID} na parte mais grossa de um \gls{bat} de metal; ou 
				\item colocar uma empunhadura com o formato de um sino ou cone no \gls{bat}. 
		\end{itemize}}
		A substitui\c{c}\~ao de uma empunhadura por outra legal n\~ao \'e considerada uma adultera\c{c}\~ao de \gls{bat}. 
		
		Uma marca de identifica\c{c}\~ao \gls{ID} gravada somente na sali\^encia arredondada de seguran\c{c}a de um \gls{bat} de metal ou uma marca a laser para os prop\'ositos de identifica\c{c}\~ao em qualquer parte de um \gls{bat} n\~ao \'e uma adultera\c{c}\~ao.
	\end{enumerate}




 
%\end{minipage}
%\end{center}

	
	\section{EQUIPAMENTO DE JOGADORES}
	\begin{multicols}{2}
	\subsection{LUVAS E \Glspl{mitt}} 
	
	\begin{enumerate}[label=\alph*)]
		\item Qualquer jogador pode usar uma luva, mas somente o receptor e o defensor da primeira base podem usar um \gls{mitt}. 
		
		\item  Nenhum cord\~ao da parte superior, bem como a tira ou outro dispositivo que fica 
		entre o polegar e o corpo da luva ou \gls{mitt} usado por um defensor da primeira base 
		ou receptor, ou da luva usada por qualquer defensor, podem ter mais de 12,70cm (5 
		polegadas) de comprimento. 
		
		\item  A luva do arremessador pode ser de qualquer cor ou combina\c{c}\~ao de cores, contanto 
		que nenhuma cor (inclusive do tran\c{c}ado) seja igual \`a cor da bola. Uma luva usada por 
		qualquer jogador, exceto o arremessador, pode ser de qualquer cor ou combina\c{c}\~ao de 
		cores. 
		
		\item  Luvas com c\'irculos brancos, cinzentos ou com tom amarelado na parte externa, que 
		deem a apar\^encia de uma bola, n\~ao s\~ao equipamento oficial; e n\~ao devem ser usadas. 
	\end{enumerate}
	(Vide \autoref{chap:Luva}: Especifica\c{c}\~oes da Luva, Desenho e Dimens\~oes) 
	
	\subsection{SAPATOS} \label{sec:Sapatos}
	
	\begin{enumerate}[label=\alph*)]
		\item Todos os membros da equipe t\^em de usar sapatos. Um sapato deve ter sua parte 
		superior feita de lona, couro ou materiais similares, e deve ser totalmente fechado. 
		
		\item  A sola e o salto do sapato podem ser lisas ou ter travas de borracha mole ou dura. 
		
		\item  Podem ser usadas placas de metal comum na sola e no salto, desde que suas pontas 
		n\~ao sejam arredondadas e n\~ao se estendam mais de 1,90cm (3/4 de polegadas) da sola ou do salto do sapato. 
		
		\item  Placas de pl\'astico duro, nylon ou poliuretano, similares \`as placas de metal para sola 
		e salto n\~ao s\~ao permitidas em qualquer categoria, em nenhum n\'ivel de jogo. 
		
		\item  O uso de sapatos com travas destac\'aveis atarraxadas sobre a sola e salto n\~ao \'e 
		permitido; entretanto, sapatos com travas destac\'aveis atarraxadas dentro do sapato 
		podem ser usados. 
		
		\item  Somente nas categorias menores e na modalidade Arremesso R\'apido Modificado 
		n\~ao \'e permitido o uso de travas de metal em nenhum n\'ivel de jogo. 
	\end{enumerate}
	
	\subsection{EQUIPAMENTO DE PROTE\c{C}\~AO}\label{ssec:EquipProtecao} 
	
	\begin{enumerate}[label=\alph*)]
		\item \label{item:Mascara} M\'ASCARAS. Todos os receptores t\^em de usar uma m\'ascara, protetor de garganta e capacete. Receptores (ou outros membros da equipe na defensiva) t\^em de usar m\'ascara, protetor de garganta e capacete enquanto recebem os arremessos de aquecimento feitos do \gls{pitching plate} (placa do arremessador), ou quando est\~ao na \'area de aquecimento. Se a pessoa que est\'a recebendo os arremessos se recusar a usar a m\'ascara, ter\'a de ser substitu\'ida por outra pessoa que se disponha a faz\^e-lo. 
		
		O receptor n\~ao precisa usar o protetor de garganta quando sua m\'ascara j\'a vem dotada de um dispositivo fixo com essa finalidade. Receptores est\~ao autorizados a usar m\'ascara do tipo utilizado por goleiro de hockey sobre gelo. Se a m\'ascara n\~ao tiver um dispositivo fixo para proteger a garganta, o receptor ter\'a de anexar um acess\'orio que atenda a essa finalidade, antes de us\'a-la. 
		
		\item  PROTETORES DE ROSTO. Qualquer jogador da defensiva ou ofensiva pode usar um protetor de rosto feito de pl\'astico que esteja aprovado. Protetores de rosto que estejam rachados ou deformados, ou que tenham o acolchoado deteriorado, ou que n\~ao tenham acolchoado, n\~ao podem ser usados, e t\^em de ser retirados do jogo. 
		Receptores n\~ao podem usar o protetor de rosto feito de pl\'astico no lugar da m\'ascara normal com protetor de garganta. 
		
		\item  PROTETORES DE TÓRAX. Todos os receptores (adultos e de categorias menores) t\^em de usar um protetor de t\'orax.
		\item CANELEIRAS. Receptores adultos e de categorias menores t\^em de usar caneleiras que ofere\c{c}am prote\c{c}\~ao \`a r\'otula quando est\~ao atuando na defensiva. 
		
		\item  PROTETOR DE PERNAS E BRA\c{C}OS. Este equipamento pode ser usado por batedor e batedor-corredor. 
	\end{enumerate}
	\end{multicols}
	\section{UNIFORMES} 
	\begin{multicols}{2}
	\subsection{UNIFORMES DE JOGADOR} \label{sec:Uniformes}
	Todos os jogadores de uma equipe t\^em de usar uniformes semelhantes em cor, estado (bom estado) e estilo. Um membro da equipe de uniforme pode, por motivos religiosos, usar cobertura para cabe\c{c}a e traje espec\'ificos que n\~ao est\~ao de acordo com estas Regras, sem penalidade. 
	
	\begin{enumerate}[label=\alph*)]
		\item BON\'ES \label{item:Bone}
		\begin{enumerate}[label=\roman* -]
			\item Os bon\'es t\^em de ser semelhantes, s\~ao obrigat\'orios para todos os jogadores masculinos, e t\^em de ser usados corretamente. 
			\item Bon\'es, viseiras e fitas para cabe\c{c}a s\~ao opcionais para jogadoras (femininas) e podem ser usadas de forma mista. 
			
			Em caso de usar mais de um tipo, todas as pe\c{c}as t\^em de ser da mesma cor, e cada pe\c{c}a do mesmo tipo tem de ser da mesma cor e estilo. 
			
			O uso de viseiras de pl\'astico ou material duro n\~ao \'e permitido. 
			
			\item Se um jogador da defensiva usar um capacete aprovado de cor similar \`a do bon\'e do uniforme da equipe, n\~ao ser\'a exigido que ele use um bon\'e. 
		\end{enumerate}
		\item \gls{undershirt}\footnote{N.T.: Camisetas internas} 
		\begin{enumerate}[label=\roman* -]
			\item Um jogador pode usar uma camiseta interna colorida (pode ser branca). N\~ao \'e necess\'ario que todos os jogadores usem camiseta interna, por\'em se um deles usar, aqueles que est\~ao usando, as camisas internas t\^em de ser semelhantes. Nenhum jogador pode usar camiseta que tenha as mangas expostas \`a vista desgastadas, desfiadas ou rasgadas. 
			\item Pode ser usada uma manga para aquecimento (compress\~ao), mas ela ser\'a tratada da mesma maneira que uma camiseta de mangas compridas. Ambos os bra\c{c}os t\^em de ser cobertos, e as duas mangas t\^em de ser da mesma cor das camisetas internas de mangas compridas usadas por outros jogadores. 
		\end{enumerate}
		\item  CAL\c{C}AS/CAL\c{C}AS PARA \gls{sliding}\footnote{\gls{sliding} \'e o ato de deslizar a uma base.}. Todas as cal\c{c}as de jogadores devem ser de um s\'o estilo, ou todas longas ou todas curtas. Os jogadores podem usar cal\c{c}as para \gls{sliding} de cor firme e uniforme. O uso de cal\c{c}as para \gls{sliding} n\~ao \'e obrigat\'orio para todos os jogadores, por\'em, se mais de um jogador estiver usando tal pe\c{c}a, todas elas t\^em de ser semelhantes em cor e estilo (excetuam-se as almofadas para \gls{sliding} de uso tempor\'ario, com bot\~ao de press\~ao ou velcro). Nenhum jogador pode usar cal\c{c}as para \gls{sliding} que tenham as partes expostas \`a vista (partes que cobrem as pernas) desgastadas, desfiadas ou rasgadas. 
		
		\item  NÚMEROS. Nas costas de todas as camisas do uniforme tem de ser usado um n\'umero ar\'abico de cor contrastante, com pelo menos 15,20cm (6 polegadas) de altura. 
		
		Nenhum \gls{coach} ou jogador da mesma equipe pode usar n\'umeros id\^enticos \footnote{1 e 01 s\~ao exemplos de n\'umeros id\^enticos}. 
		
		Somente n\'umeros redondos (01 a 99) devem ser usados. 
		
		Jogadores sem n\'umero n\~ao ter\~ao permiss\~ao para jogar. 
		
		\item  NOMES. Acima do n\'umero nas costas da camisa do uniforme pode ser inscrito o nome individual do jogador. 
		
		\item  PE\c{C}AS MOLDADAS. Pe\c{c}as moldadas (gesso, metal ou outras subst\^ancias duras em sua forma fina  n\~ao podem ser usadas num jogo. Qualquer metal exposto (exceto uma pe\c{c}a moldada) tem de ser adequadamente coberto com um material macio, preso com fita adesiva e aprovado pelo \'arbitro. 
		
		\item  ADORNOS QUE PODEM CAUSAR DISTRA\c{C}\~AO. Nenhuma pe\c{c}a exposta, incluindo joia que, na opini\~ao do \'arbitro, pode causar distra\c{c}\~ao a jogadores da equipe advers\'aria pode ser usada ou ostentada. O \'arbitro tem de mandar retir\'a-la ou cobri-la. 
		
		Braceletes e/ou colares com fins medicinais que, na opini\~ao do \'arbitro, podem causar distra\c{c}\~ao t\^em de estar presos ao corpo com fita adesiva, de tal maneira que a informa\c{c}\~ao sobre a finalidade dessas pe\c{c}as fique vis\'ivel. 
		
	\end{enumerate}
	\end{multicols}
	\subsection{UNIFORMES DE \Glspl{coach}}\label{sec:UniformesCoach}
	\begin{multicols}{2}
	Um \gls{coach} tem de estar devidamente trajado –inclusive usar cal\c{c}ados apropriados -- ou vestido com o uniforme da equipe, que deve seguir o padr\~ao de cor (es) do clube.
	
	 {\color{red!80}Se um \gls{coach} decide usar um bon\'e, este tem de estar de acordo com a Regra \ref{sec:Uniformes} \ref{item:Bone}}. 
	 
	\subsection{EQUIPAMENTO} 
	N\~ao obstante qualquer disposi\c{c}\~ao destas Regras, a WBSC-SD ou ISF Equipment Standards Commission se reserva o direito de negar ou revogar a aprova\c{c}\~ao de qualquer equipamento que, na sua opini\~ao \'unica, mude significativamente a caracter\'istica do jogo, afete a seguran\c{c}a de participantes ou espectadores ou torne a performance dos jogadores um produto de seu equipamento e n\~ao de sua habilidade individual. 
	\end{multicols}

\subsection*{EFEITOS} 

{\footnotesize
	\begin{tabular}{p{.10\columnwidth}p{.25\columnwidth}|p{.55\columnwidth}}
		\multicolumn{2}{c|}{Regra} & Efeito \\\hline\hline
		\ref{sec:Sapatos}& Uso de sapatos impr\'oprios.&O membro da equipe que continuar cometendo a infra\c{c}\~ao, ap\'os uma advert\^encia do \'arbitro, dever\'a ser expulso do jogo.\\\hline
		\ref{ssec:EquipProtecao}\ref{item:Mascara}&Receptor deixa de usar um capacete& Se o jogador continuar cometendo a infra\c{c}\~ao, ap\'os uma advert\^encia do \'arbitro, dever\'a ser expulso do jogo.\\\hline
		\ref{ssec:EquipProtecao} (b-d) &- Jogador n\~ao usa equipamento obrigat\'orio&O jogador \'e removido do jogo. Se continuar participando, ser\'a expulso do jogo.\\\hline
		\ref{sec:Uniformes}& Uniforme impr\'oprio ou uso incorreto do uniforme por um jogador&
		Se o jogador se recusar a cumprir o que est\'a determinado, dever\'a ser expulso do jogo. 
		\\\hline
		\ref{sec:UniformesCoach} & \Gls{coach} usa roupa impr\'opria. &
		Ap\'os uma advert\^encia do \'arbitro, qualquer infra\c{c}\~ao subsequente cometida por um \gls{coach} ou t\'ecnico da mesma equipe resultar\'a na expuls\~ao do \Gls{coach} principal.\\\hline
\end{tabular}}


\chapter{PARTICIPANTES}

\section{DEFINI\c{C}\~OES PARA \Glspl{coach}} 

\begin{description}
	\item[\Gls{coach}]: \'e uma pessoa que \'e respons\'avel pelas a\c{c}\~oes de sua equipe no campo e pela comunica\c{c}\~ao com o \'arbitro e com a equipe contr\'aria. Um jogador pode ser um \gls{coach}, como substituto de um \gls{coach} ausente ou como um jogador-\gls{coach}. 
\end{description}
	
\begin{description}	
	\item[\Gls{coach} principal]:  	Um t\'ecnico de uma equipe ou o \gls{coach} que assume as principais responsabilidades de um \gls{coach} \'e considerado o \gls{coach} principal. 
	\item[\Gls{coach} de base]: \'E um membro da equipe na ofensiva, que fica posicionado no campo e dentro do \gls{coachsbox} enquanto sua equipe est\'a atacando. 

\end{description}

\section{DEFINI\c{C}\~OES PARA JOGADORAS} 

\begin{description}	
	\item[JOGADOR SOMENTE DA OFENSIVA (OPO)]: \'E um jogador que est\'a no \gls{batting order} (ordem de batedores), exceto o JOGADOR FLEX, por quem o JD est\'a jogando na defesa. 
	\item[JOGADOR DESIGNADO (JD)]: \'E um jogador abridor da equipe na ofensiva, que est\'a escalado para bater no lugar do Jogador Flex. 
	
	\item[JOGADOR FLEX]: \'E o jogador abridor que tem um Jogador Designado (JD) para bater no seu lugar, e cujo nome est\'a relacionado no 10º lugar no Formul\'ario de Escala\c{c}\~ao da Equipe (Line up). O Flex pode jogar em qualquer posi\c{c}\~ao defensiva e pode entrar no jogo na ofensiva somente para bater no lugar do JD. 

	\item[JOGADOR ILEGAL]: \'E um jogador que: 
	
	\begin{enumerate}[label=\alph*)]
		\item assume uma posi\c{c}\~ao na Escala\c{c}\~ao da Equipe, tanto na ofensiva como na defensiva, sem ter sido anunciado como um substituto ao \'arbitro de \gls{home}; ou
		\item assume uma posi\c{c}\~ao na ofensiva ou defensiva, sem ter condi\c{c}\~ao legal para isso. 
	\end{enumerate}

	\item[EXPULSÃO]: \'E o ato mediante o qual um \'arbitro ordena que um jogador, oficiais ou qualquer membro da equipe deixe o jogo e o campo, por viola\c{c}\~ao de uma regra, pelo resto do jogo. 
	\item[REINGRESSO ILEGAL]: Um Reingresso Ilegal ocorre quando: 
	
	\begin{enumerate}[label=\alph*)]
		\item um jogador, incluindo o JD e FLEX, retorna ao jogo numa posi\c{c}\~ao na ordem de batedores para a qual n\~ao est\'a legalmente habilitado, isto \'e, numa posi\c{c}\~ao que n\~ao \'e a sua posi\c{c}\~ao inicial original; ou 
		
		\item  um jogador retorna ao jogo na defensiva ou ofensiva e n\~ao est\'a legalmente habilitado para entrar naquela posi\c{c}\~ao. 
	\end{enumerate}
	\item[SUBSTITUTO ILEGAL]: \'E um jogador que entrou no jogo sem ser anunciado (n\~ao reportado), como um substituto, ao \'arbitro. Esse jogador pode ser: 
	
	\begin{enumerate}[label=\alph*)]
		\item um substituto que n\~ao tenha atuado no jogo anteriormente; 
		\item  um Jogador Ilegal; 
		\item  um jogador que fora declarado ineleg\'ivel; 
		\item  um jogador que tenha reingressado ilegalmente; 
		\item  um JD ou JOGADOR FLEX ilegal; 
		\item  um Jogador de Emerg\^encia que permanece no jogo como um substituto n\~ao anunciado de um Jogador Removido que n\~ao tenha retornado (ao jogo) dentro do tempo permitido de acordo com as prescri\c{c}\~oes da Regra de Jogador de Emerg\^encia. 
	\end{enumerate}
	
	\item[JOGADOR INELEGÍVEL]: \'E aquele que n\~ao pode mais participar do jogo como um jogador, por ter sido removido pelo \'arbitro. Um jogador Ineleg\'ivel pode continuar no jogo como um \gls{coach}. 
	
	\item[JOGADOR DE EMERGÊNCIA INELEGÍVEL]: Um Jogador de Emerg\^encia Ineleg\'ivel \'e aquele que n\~ao pode entrar no jogo para substituir um Jogador Removido (jogador que tem de deixar o jogo para cuidar de um ferimento que tenha causado hemorragia). \'E um jogador que: 
	
	\begin{enumerate}[label=\alph*)]
		\item foi removido do jogo pelo \'arbitro por ter infringido uma Regra; 
		
		\item  est\'a atuando no jogo naquele momento;
		\item n\~ao est\'a atuando no jogo naquele momento, mas est\'a eleg\'ivel para reingressar no jogo. 
	\end{enumerate}
	
	\item[DEFENSOR]: \'E qualquer jogador da defensiva da equipe que est\'a no campo. 
		
	\item[DEFENSOR DO CAMPO INTERNO]: \'E um jogador da defensiva –incluindo o arremessador e o receptor–, que geralmente est\'a posicionado em qualquer lugar perto ou dentro das linhas do caminho da base que formam o territ\'orio \gls{fair}. Um jogador que normalmente joga no campo externo pode ser considerado um defensor do campo interno se ele se move para uma \'area normalmente coberta por defensores do campo interno. 
 
		
	\item[FORMUL\'ARIO DE ESCALA\c{C}ÃO DA EQUIPE (LINE UP CARD)]: \'E a lista de jogadores abridores, substitutos e \glspl{coach}, que \'e entregue ao Chefe dos \'Arbitros e/ou \'Arbitro de \gls{home} e ao Anotador Oficial, antes do in\'icio do jogo. O \'Arbitro de \gls{home} ret\'em o Formul\'ario de Escala\c{c}\~ao da Equipe durante o jogo. 
	
	\item[ESCALA\c{C}ÃO DA EQUIPE]: Jogadores –incluindo o JD e o Jogador Flex– que, no dado momento, est\~ao jogando na ofensiva e na defensiva.
	

	
	\item[REINGRESSO]: 	Ocorre quando um jogador abridor retorna ao jogo ap\'os ser substitu\'ido. 
	
	\item[REMO\c{C}ÃO DO JOGO]: Quando um \'arbitro declara que um jogador n\~ao est\'a eleg\'ivel para nova participa\c{c}\~ao no jogo em raz\~ao de transgress\~ao de Regra. Qualquer pessoa assim removida pode continuar no \gls{bench}, mas n\~ao pode mais participar do jogo, exceto como um \gls{coach}. 
	
	\item[JOGADOR DE EMERGÊNCIA]: \'E aquele autorizado a entrar no jogo para substituir um jogador que tem de sair (do jogo) para tratar de um ferimento com hemorragia. 
	
	\item[JOGADORES ABRIDORES ]: S\~ao aqueles relacionados no Formul\'ario de Escala\c{c}\~ao da Equipe (Line up), que iniciam o jogo na defensiva ou na ofensiva. 
	
	\item[SUBSTITUTO]: 
	\begin{enumerate}[label=\alph*)]
		\item \'E um jogador (n\~ao um Jogador Abridor) que n\~ao tenha atuado no jogo, a n\~ao ser como Jogador de Emerg\^encia. 
		
		\item  \'E um jogador abridor que deixara o jogo uma vez e que pode retornar a esse jogo. 
	\end{enumerate}
	
	\item[MEMBRO DA EQUIPE]: 
	\'E uma pessoa autorizada a sentar no \gls{bench} da equipe. 
	
	\item[CORREDOR TEMPOR\'ARIO]:  
	\'E um jogador que pode correr no lugar do receptor quando este, com dois \glspl{out}, est\'a ocupando uma base. 
	
	\item[JOGADOR REMOVIDO]:  \'E um jogador que tem de deixar o jogo devido a um ferimento com hemorragia que n\~ao pode ser estancada num tempo razo\'avel, ou quando o uniforme do jogador fica coberto de sangue. 
\end{description}


\section{ESCALA\c{C}ÃO DA EQUIPE E LISTAS DE JOGADORES (LINE UP \& ROSTERS)}
\begin{multicols}{2} 
	\subsection{FORMUL\'ARIOS DE ESCALA\c{C}ÃO DA EQUIPE (LINE UP CARDS)}
	
	\begin{enumerate}[label=\alph*)]
		\item O Formul\'ario de Escala\c{c}\~ao cont\'em: 
		\begin{enumerate}[label=\roman* -]
			\item  o \'ultimo sobrenome, o primeiro nome, a posi\c{c}\~ao e o n\'umero de uniforme dos jogadores abridores; 
			\item  o \'ultimo sobrenome, o primeiro nome e o n\'umero de uniforme dos substitutos dispon\'iveis; e 
			\item o \'ultimo sobrenome e o primeiro nome do \gls{coach} principal. 
		\end{enumerate}
		
		\item  O nome de um jogador abridor n\~ao pode estar no Formul\'ario de Escala\c{c}\~ao da Equipe se ele n\~ao estiver presente –uniformizado–na \'area de sua equipe. 
		
		\item  Um jogador Eleg\'ivel que consta da rela\c{c}\~ao de jogadores pode ser acrescentado \`a lista de reservas, a qualquer momento durante o jogo. 
		
		\item  A lista masculina deve conter somente os nomes de jogadores, e a lista feminina, somente os nomes de jogadoras. 
		
		\item  Se um n\'umero de uniforme est\'a inscrito incorretamente no Formul\'ario de Escala\c{c}\~ao da Equipe, a mudan\c{c}a pode ser feita sem penalidade. Se um jogador que est\'a usando uniforme com n\'umero incorreto infringe qualquer Regra, a viola\c{c}\~ao de Regra tem preced\^encia (a Regra tem de ser aplicada). Se o jogador permanece no jogo ap\'os cometer a infra\c{c}\~ao, o n\'umero do uniforme tem de ser corrigido antes de dar prosseguimento \`a partida. 
	\end{enumerate}
	
	\subsection{JOGADORES} 
	
	\begin{enumerate}[label=\alph*)]
		\item Cada equipe tem que ter no m\'inimo nove (9) jogadores escalados em todos os momentos. Se estiver usando Jogador Designado (JD), uma equipe tem de ter dez (10) jogadores relacionados na escala\c{c}\~ao da equipe. O nome do JD tem de estar inscrito no Formul\'ario de Escala\c{c}\~ao da Equipe inicial. 
		
		\begin{enumerate}[label=\roman* -]
			\item As posi\c{c}\~oes da equipe na defensiva s\~ao:
			\begin{itemize}\small
				\item  arremessador (D1), 
				\item  receptor (D2), 
				\item defensor da primeira base (D3), 
				\item defensor da segunda base (D4), 
				\item defensor da terceira base (D5), 
				\item interbases (D6), 
				\item defensor externo esquerdo (D7), 
				\item defensor externo central (D8) e 
				\item defensor externo direito (D9).
			\end{itemize} 
			\item As posi\c{c}\~oes da equipe na defensiva com dez (10) jogadores s\~ao as mesmas de uma equipe com nove (9) jogadores, mais o JD. 
		\end{enumerate}
		\item  Jogadores da equipe que est\'a dentro do campo podem posicionar-se em qualquer lugar do territ\'orio \gls{fair}, no in\'icio de cada arremesso, exceto o receptor, que tem de estar no \gls{catcher's box}, e o arremessador, que tem de estar numa posi\c{c}\~ao legal de arremesso, ou dentro do C\'irculo do Arremessador, quando o \'arbitro p\~oe a bola em jogo. 
		
		\item  Uma equipe tem de ter a quantidade exigida de jogadores Eleg\'iveis na escala\c{c}\~ao, em todos os momentos, para continuar um jogo. 
	\end{enumerate}
	
	\subsection{JOGADORES ABRIDORES} 
	
	\begin{enumerate}[label=\alph*)]
		\item Um jogador abridor \'e oficializado uma vez que o Formul\'ario de Escala\c{c}\~ao da Equipe 
		\'e confirmado pelo representante da equipe e pelo \'arbitro de \gls{home} na reuni\~ao pr\'e-jogo no \gls{homeplate}. 
		
		\item  Nomes, n\'umeros de uniformes e posi\c{c}\~oes podem estar relacionados no Formul\'ario de Escala\c{c}\~ao da Equipe antes da reuni\~ao pr\'e-jogo. 
		
		\item  Em caso de ferimento ou doen\c{c}a, o representante da equipe pode fazer mudan\c{c}as no Formul\'ario de Escala\c{c}\~ao da Equipe na reuni\~ao no \gls{homeplate} antes de as escala\c{c}\~oes serem declaradas oficiais. Um substituto listado pode ocupar o lugar de um jogador abridor doente ou machucado cujo nome est\'a na escala\c{c}\~ao inicial de sua equipe. Ele deve ser considerado o jogador abridor, e o outro jogador pode ser um substituto. 
		
		\item  O jogador abridor assim substitu\'ido na reuni\~ao no \gls{homeplate} pode entrar no jogo, mais tarde, como um substituto, a qualquer momento.
		\item Todos os jogadores abridores, incluindo o JD e o Jogador Flex, podem ser substitu\'idos e reingressar no jogo uma vez, e t\^em de permanecer na mesma posi\c{c}\~ao na ordem de batedores sempre que estiverem atuando. 
	\end{enumerate}
	\subsection{JOGADOR DESIGNADO (JD)} 
	
	\begin{enumerate}[label=\alph*)]
		\item Um JD pode bater no lugar de qualquer jogador da defensiva. 
		
		\item  O JD pode jogar na defensiva no lugar de qualquer jogador, incluindo o Jogador Flex. 
		\begin{enumerate}[label=\roman* -]
			\item Se o JD joga na defensiva no lugar de um jogador que n\~ao seja o Jogador Flex, esse jogador continua batendo e \'e identificado como o \gls{OPO} (Jogador Somente da Ofensiva). O OPO n\~ao \'e considerado ter deixado o jogo e continua a bater, mas n\~ao joga na defensiva. 
			\item Quando o JD joga na defensiva no lugar do Jogador Flex, isso \'e tratado como uma substitui\c{c}\~ao e tem de ser comunicado ao \'arbitro. 
			\item Quando o JD joga na defensiva no lugar do Jogador Flex, a quantidade de jogadores fica reduzida a nove (9), e o jogo pode terminar legalmente com nove (9) jogadores. 
		\end{enumerate}
		\item  O JD e o Jogador Flex n\~ao podem estar no jogo atuando na ofensiva ao mesmo tempo. 
	\end{enumerate}
	
	\subsection{JOGADOR FLEX (FLEX) }
	
	\begin{enumerate}[label=\alph*)]
		\item Se uma equipe anuncia o uso de um JD, ela tem de mencionar o nome de um 
		Jogador Flex no Formul\'ario de Escala\c{c}\~ao da Equipe. 
		
		\item  O Jogador Flex \'e colocado na d\'ecima (10ª) posi\c{c}\~ao no \gls{line-up} (escala\c{c}\~ao da equipe) inicial, em seguida aos nomes dos nove (9) batedores, e pode jogar em 
		qualquer posi\c{c}\~ao como defensor. 
		
		\item  O Jogador Flex pode entrar no jogo para atuar na ofensiva somente no lugar do JD. 
		i. Quando o Jogador Flex atua como batedor, a quantidade de jogadores fica reduzida a 
		nove (9). A equipe pode terminar o jogo com nove (9) jogadores. 
		ii. O Jogador Flex pode atuar como batedor no lugar do JD quantas vezes for 
		necess\'ario. Cada vez que o Jogador Flex atua na ofensiva no lugar do JD, o \'arbitro de 
		\gls{home} deve ser anunciado, j\'a que essa altera\c{c}\~ao \'e tratada como uma substitui\c{c}\~ao. 
	\end{enumerate}
	\subsection{JOGADOR DE EMERGÊNCIA} 
	
	\begin{enumerate}[label=\alph*)]
		\item Um Jogador de Emerg\^encia pode entrar no jogo no lugar de um Jogador Removido.	\item O Jogador Removido n\~ao deve retornar ao jogo at\'e que a hemorragia tenha cessado, o ferimento esteja limpo e coberto e, se necess\'ario, o uniforme seja substitu\'ido, mesmo que a camisa do uniforme tenha um n\'umero diferente. N\~ao haver\'a penalidade se usar uma camisa com n\'umero diferente; entretanto, o \'arbitro tem de ser informado sobre a mudan\c{c}a de n\'umero. 
		
		\item  Um Jogador de Emerg\^encia pode jogar no lugar do Jogador Removido pelo resto do \gls{inning} em andamento e pelo \gls{inning} seguinte completo. O Jogador Removido pode retornar ao jogo a qualquer momento durante esse per\'iodo, sem ser tratado como uma substitui\c{c}\~ao. Um Jogador de Emerg\^encia n\~ao \'e considerado um substituto. Se o Jogador Removido n\~ao retorna, depois de terminado o restante de \gls{inning} e depois de completado o \gls{inning} seguinte completo, tem de ser feita uma substitui\c{c}\~ao legal. 
		
		\item  Um representante da equipe tem de comunicar ao \'arbitro de \gls{home} todas as mudan\c{c}as. Se n\~ao o fizer, o jogador que entra no jogo sem ser anunciado estar\'a sujeito a ser declarado um substituto ilegal. 
		
		\item  Um Jogador de Emerg\^encia pode ser:
		\begin{enumerate}[label=\roman* -]
			\item um substituto legal que n\~ao tenha ainda atuado no jogo; 
			\item um substituto legal que tenha atuado no jogo, mas que fora substitu\'ido depois; ou 
			\item um jogador abridor que n\~ao est\'a mais atuando e n\~ao tem mais condi\c{c}\~ao de retornar ao jogo. 
		\end{enumerate}
	\end{enumerate}
	
	\subsection{CORREDOR TEMPOR\'ARIO}
	\'E legal o uso de Corredor Tempor\'ario para um receptor que est\'a atuando no jogo, quando ele, com dois (2) \glspl{out}, est\'a ocupando uma base. Deve-se aplicar as seguintes normas: 
	
	\begin{enumerate}[label=\alph*)]
		\item o uso de Corredor Tempor\'ario \'e opcional para o t\'ecnico da equipe na ofensiva; 
		
		\item  o Corredor Tempor\'ario pode ser usado a qualquer momento depois que ocorre a segunda elimina\c{c}\~ao; 
		\item o Corredor Tempor\'ario \'e o jogador que, no momento em que o t\'ecnico opta pelo seu uso, \'e o \'ultimo da ordem de batedores e n\~ao est\'a ocupando uma base. Se for usado um jogador incorreto como Corredor Tempor\'ario, o erro dever\'a ser corrigido, sem penalidade, quando for percebido. 
	\end{enumerate}
	
	\section{SUBSTITUI\c{C}\~OES} 
	
	\begin{enumerate}[label=\alph*)]
		\item Um substituto pode ocupar o lugar de qualquer jogador cujo nome est\'a no \gls{line-up} (escala\c{c}\~ao da equipe). Podem ser feitas substitui\c{c}\~oes m\'ultiplas para o jogador relacionado na escala\c{c}\~ao inicial, mas nenhum substituto pode retornar ao jogo depois de ter sido retirado do \gls{line-up}, exceto como um Jogador de Emerg\^encia ou \gls{coach}. 
		
		\item  Um jogador abridor e seu (s) substituto (s) n\~ao podem estar no jogo ao mesmo tempo. 
		
		\item  Uma substitui\c{c}\~ao tem de ser feita somente quando a bola est\'a morta. O \gls{coach} ou o representante da equipe tem de comunicar imediatamente ao \'arbitro de \gls{home} antes de fazer a substitui\c{c}\~ao. O substituto n\~ao \'e considerado legalmente no jogo at\'e que um arremesso tenha sido efetuado ou uma jogada tenha sido executada. O \'arbitro de \gls{home} anunciar\'a a mudan\c{c}a ao anotador. 
		
		\item  Qualquer substituto que est\'a legalmente no jogo, mas n\~ao tenha sido anunciado ao \'arbitro, torna-se um Substituto Ilegal. 
		
		\item  N\~ao h\'a qualquer infra\c{c}\~ao se o t\'ecnico, \gls{coach}, representante da equipe ou o jogador que comete falta avisa o \'arbitro antes da apela\c{c}\~ao da equipe prejudicada. 
		
		\item  Um substituto que retorna ao jogo depois de ser substitu\'ido \'e um jogador ilegal (Reingresso Ilegal), a menos que ele esteja sendo usado como um Jogador de Emerg\^encia ou \gls{coach}. 
		
		\item  Se um ferimento impede que um batedor-corredor ou um corredor que est\'a avan\c{c}ando a uma base que lhe fora concedida enquanto a bola est\'a morta, esse batedor-corredor ou corredor pode ser substitu\'ido. O substituto ser\'a autorizado a ir at\'e a base concedida. O substituto tem de tocar a (s) base (s) concedida (s) ou a (s) que deixou de tocar anteriormente. 
	\end{enumerate}
\end{multicols}

\section{APELA\c{C}\~OES}
\begin{multicols}{2} 
	\begin{enumerate}[label=\alph*)]
		\item Apela\c{c}\~oes t\^em de ser feitas por um t\'ecnico, \gls{coach} ou jogador antes que um 
		\'arbitro d\^e uma decis\~ao sobre: 
		\begin{enumerate}[label=\roman* -]
			\item Substitui\c{c}\~oes ilegais; 
			\item O uso de um jogador n\~ao anunciado quando est\'a sendo aplicada a Regra de Jogador de Emerg\^encia; 
			\item Reingresso Ilegal; ou 
			\item O uso de um jogador n\~ao anunciado quando est\'a sendo aplicada a Regra de Jogador Designado.
			\item Uma apela\c{c}\~ao pelos motivos acima pode ser feita a qualquer momento enquanto o jogador est\'a no jogo. 
		\end{enumerate}
	\end{enumerate}
\end{multicols}
\vfill

\section*{EFEITOS} 

{\footnotesize
	\begin{tabular}{p{.08\columnwidth}p{.20\columnwidth}|p{.55\columnwidth}}
		\multicolumn{2}{c|}{Regra} & Efeito \\\hline\hline
		3.2.2 (a), 3.2.3 (c) 3.2.6 (c) & 
		Quando n\~ao completa um jogo com a quantidade necess\'aria de jogadores. 
		&
		
		O jogo \'e confiscado a favor da equipe n\~ao infratora. 	\\\hline
		%\end{tabular}
		%
		%\vspace{4mm}
		%\begin{tabular}{p{.05\columnwidth}p{.20\columnwidth}|p{.60\columnwidth}}
		%	\multicolumn{2}{c|}{Regra} & Efeito \\\hline\hline
		3.2.8& Efeitos Substituto N\~ao Anunciado/Jogador Ilegal:
		
		\begin{enumerate}[label=\alph*)]
			\item Substituto Ilegal; 	
			\item  Jogador de Emerg\^encia n\~ao anunciado; ou 
			\item  Retorno de Jogador Removido n\~ao comunicado.
		\end{enumerate} 
		&
		
		\begin{enumerate}[label=\alph*)]
			\item Substituto N\~ao Anunciado ou Jogador Ilegal: \'e uma Jogada de Apela\c{c}\~ao. 
			
			\item  A apela\c{c}\~ao tem de ser levada \`a aten\c{c}\~ao do \'arbitro enquanto o jogador ilegal ou o substituto n\~ao anunciado est\'a no jogo. 
			
			\item  Uma vez que tenha sido efetuado um arremesso, ou tenha sido executada uma jogada, e o substituto n\~ao anunciado tenha sido descoberto, esse substituto \'e declarado um Jogador Ineleg\'ivel. 
			
			\item  Um substituto legal tem de ocupar o lugar do Jogador Ineleg\'ivel. 
			\begin{enumerate}[label=\roman* -]
				\item Se a equipe infratora n\~ao tem um substituto legal, o jogo \'e confiscado. 
				\item Se o jogador ilegal sofre apela\c{c}\~ao enquanto est\'a no \gls{batter's box}, um substituto legal deve assumir a contagem de arremessos (\gls{ball} e \gls{strike}). 
				\item Toda a\c{c}\~ao anterior \`a descoberta \'e legal, exceto se o substituto n\~ao anunciado bate o arremesso e alcan\c{c}a uma base, e depois, ap\'os ser descoberto, sofre apela\c{c}\~ao antes de um arremesso ao Batedor Prevenido, ou no fim do jogo e antes de o \'arbitro deixar o campo; todos os corredores (incluindo o batedor) devem retornar \`a base que estavam ocupando no momento do arremesso; o substituto n\~ao anunciado \'e \textit{Declarado Ineleg\'ivel} e \'e declarado \gls{out}. 
				\item Todas as elimina\c{c}\~oes, com exce\c{c}\~ao dos casos mencionados no item (d) acima permanecer\~ao. 
				\item Se o substituto \'e um Jogador Ilegal, ele deve estar tamb\'em sujeito \`a penalidade por essa infra\c{c}\~ao. 
			\end{enumerate}
			\item  Se o Jogador Ilegal \'e descoberto atuando na defensiva e, ap\'os executar uma jogada, \'e feita uma apela\c{c}\~ao correta, esse jogador \'e declarado Ineleg\'ivel, e a equipe na ofensiva pode: 
			\begin{enumerate}[label= \arabic*)]
				\item  aceitar o resultado da jogada; ou 
				\item  optar pelo retorno do batedor, assumindo a contagem de arremessos (\gls{ball} e \gls{strike}) que tinha antes de o Jogador Ilegal ser descoberto. Cada corredor deve retornar \`a base que estava ocupando antes da jogada. 
			\end{enumerate}
			\item  Se um Jogador Ineleg\'ivel retorna ao jogo, \'e declarado um confisco de jogo a favor da equipe n\~ao infratora. 
			
			\item  Ap\'os ser comunicada uma apela\c{c}\~ao sobre substitui\c{c}\~ao n\~ao anunciada ou reingresso n\~ao anunciado, o jogador original ou seu substituto \'e considerado ter deixado o jogo. 
		\end{enumerate}
		\\\hline
	\end{tabular}
	
	\begin{tabular}{p{.07\columnwidth}p{.25\columnwidth}|p{.55\columnwidth}}
		\multicolumn{2}{c|}{Regra} & Efeito \\\hline\hline
		3.2.8 & Reingresso Ilegal.
		\begin{enumerate}[label=\alph*)]
			\item Jogador abridor retornando ao \gls{line-up} numa posi\c{c}\~ao diferente no \gls{line-up} da ofensiva. 
			
			\item  Substituto reingressando no jogo, n\~ao como um Jogador de Emerg\^encia.
			\item Jogador abridor reingressando no jogo pela segunda vez, n\~ao como um Jogador de Emerg\^encia. 
			\item  Um Jogador de Emerg\^encia Ineleg\'ivel. 
			\item  Flex entrando no jogo para atuar na ofensiva por um jogador que n\~ao seja o Jogador Designado. 
		\end{enumerate}
		
		&
		
		\begin{enumerate}[label=\alph*)]
			\item Esta \'e uma jogada de apela\c{c}\~ao. 
			\begin{enumerate}[label=\roman* -]
				\item A apela\c{c}\~ao pode ser feita sempre que um jogador que reingressara ilegalmente est\'a no jogo. 
				\item A apela\c{c}\~ao n\~ao precisa ser feita antes do pr\'oximo arremesso. 
			\end{enumerate}
			
			\item  O t\'ecnico/\gls{coach} mencionado no Formul\'ario de Escala\c{c}\~ao da Equipe e o jogador que reingressara ilegalmente no jogo \'e expulso. 
			
			\item  Um substituto legal tem de ocupar o lugar do jogador expulso por ter reingressado no jogo ilegalmente, antes de o jogo ter prosseguimento. 
			
			\item  Se o t\'ecnico/\gls{coach} principal \'e expulso, ele tem de nomear um novo t\'ecnico/\gls{coach} principal. 
			
			\item  Em a\c{c}\~oes que ocorrem enquanto o jogador que reingressara ilegalmente est\'a no jogo devem ser aplicadas as penalidades impostas a Substituto Ilegal/Jogador ilegal. 
		\end{enumerate}
		\\\hline
	\end{tabular}
}

\section{\Glspl{coach} }
	\begin{multicols}{2} 
		\subsection{EM GERAL}
		
		\begin{enumerate}[label=\alph*)]
			\item Um \gls{coach} ou representante da equipe tem a responsabilidade de avisar o \'arbitro de \gls{home} quando ocorre mudan\c{c}a na escala\c{c}\~ao da equipe. 
			
			\item  Um \gls{coach} n\~ao pode usar linguagem que possa repercutir negativamente sobre 
			jogadores, \'arbitros ou espectadores. 
			
			\item  Nenhum equipamento de comunica\c{c}\~ao deve ser usado entre: 
			\begin{enumerate}[label=\roman* -]
				\item Os \glspl{coach} no campo. 
				\item Um \gls{coach} e o \Gls{dugout}. 
				\item Um \gls{coach} e qualquer jogador. 
				\item A \'area do espectador e o campo, incluindo o \Gls{dugout}, um \gls{coach} e um jogador. 
			\end{enumerate}
			\item  Um \gls{coach} ou t\'ecnico da equipe na defensiva pode ser um membro da equipe (n\~ao um jogador), que permanece no \Gls{dugout}, ou um \gls{coach} que est\'a atuando como jogador (\gls{coach}-jogador). 
			
			\item  Um \gls{coach}-jogador num jogo pode orientar a sua equipe e lhe dar assist\^encia durante o jogo. 
		\end{enumerate}
		
		\subsection{COACH PRINCIPAL} 
		\begin{enumerate}[label=\alph*)]
			\item O \gls{coach} principal tem a responsabilidade de assinar o Formul\'ario de Escala\c{c}\~ao da Equipe. 
			
			\item  Caso o \gls{coach} principal seja expulso do jogo, ele deve submeter \`a aprecia\c{c}\~ao do 
			\'arbitro de \gls{home} o nome da pessoa que vai assumir as suas obriga\c{c}\~oes pelo resto do 
			jogo. 
		\end{enumerate}
		\subsection{COACHES DE BASE}
		
		\begin{enumerate}[label=\alph*)]
			\item S\~ao permitidos dois \glspl{coach} de base, cuja fun\c{c}\~ao \'e dar assist\^encia e passar instru\c{c}\~oes aos membros de sua equipe enquanto est\~ao no ataque. 
			
			\begin{enumerate}[label=\roman* -]
				\item Cada \gls{base coach} tem de permanecer com ambos os p\'es dentro dos limites de seus respectivos \gls{coachsbox}. Um tem de estar posicionado perto da primeira base, e outro, perto da terceira base. 
				
				\item Um \gls{coach} de base pode deixar o \gls{coachsbox} para esquivar-se de um defensor, ou para mandar um corredor deslizar, avan\c{c}ar ou retornar a uma base, desde que n\~ao interfira na jogada. 
			\end{enumerate}
			\item  Um \gls{coach} de base pode dirigir-se somente a membros de sua equipe. 
			
			\item  Um \gls{coach} de base pode levar consigo, ao \gls{coachsbox}, um Livro de Anota\c{c}\~oes (\gls{scorebook}), caneta ou l\'apis e um \gls{indicator} (marcador de \gls{ball}, \gls{strike} e \gls{out}), que ser\~ao usados somente para anotar pontos ou registrar outros dados do jogo. 
			
			\item  Um jogador jovem que atua como \gls{coach} nos \glspl{coachsbox} da primeira ou terceira base, e um jovem que participa do jogo como \gls{bat boy} ou \gls{bat girl} (recolhedor/recolhedora de \gls{bat}), t\^em de usar um capacete aprovado enquanto est\~ao no campo ou dentro do \Gls{dugout}. 
			
		\end{enumerate}
	\end{multicols}
\section*{EFEITOS}

{\footnotesize
	\begin{tabular}{p{.08\columnwidth}p{.20\columnwidth}|p{.55\columnwidth}}
		\multicolumn{2}{c|}{Regra} & Efeito \\\hline\hline 
		
		3.4 & Transgress\~ao da regra sobre responsabilidades dos \glspl{coach}. &
		Na primeira infra\c{c}\~ao, deve ser feita uma advert\^encia. Qualquer infra\c{c}\~ao subsequente cometida por um \gls{coach}/t\'ecnico da mesma equipe resultar\'a na expuls\~ao do \Gls{coach} Principal. 
		\\\hline
		3.4.3 (d) & Jogador jovem que atua como \gls{coach} no \gls{coachsbox}, sem usar um capacete. &
		Se repetir a infra\c{c}\~ao ap\'os ser advertido, o jogador jovem tem que ser expulso. \\\hline
\end{tabular}}

\section{MEMBROS DA EQUIPE} 

\subsection{EM GERAL} 
	\begin{multicols}{2}
		\begin{enumerate}[label=\alph*)]
			\item Nenhum membro da equipe pode contestar qualquer decis\~ao do \'arbitro que implique aprecia\c{c}\~ao. 
			
			\item  Durante um jogo, uma pessoa relacionada no Formul\'ario de Escala\c{c}\~ao da Equipe, ou com permiss\~ao para ficar no \Gls{dugout}, tem de permanecer dentro dele, a menos que as Regras permitam, ou quando o \'arbitro considera justific\'avel, sua perman\^encia fora dessa \'area. Isso inclui jogadores, exceto o Batedor Prevenido (que tem de permanecer dentro do C\'irculo do Batedor Prevenido), no in\'icio do jogo, entre \glspl{inning}, ou quando um arremessador est\'a se aquecendo. Nessa \'area, \'e proibido fumar, consumir \'alcool ou usar fumo de mascar. 
			
			\item  Um membro da equipe n\~ao deve: 
			\begin{enumerate}[label=\roman* -]
				\item fazer, ou permitir que outras pessoas fa\c{c}am ou mandem fazer, coment\'ario depreciativo ou insultuoso para/sobre jogadores advers\'arios, dirigentes, \'arbitros ou espectadores; 
				\item cometer qualquer ato que possa ser considerado uma conduta antidesportiva. 
			\end{enumerate}
		\end{enumerate}
		
	\end{multicols}
\subsection*{EFEITOS}

{\footnotesize
	\begin{tabular}{p{.08\columnwidth}p{.15\columnwidth}|p{.70\columnwidth}}
		\multicolumn{2}{c|}{Regra} & Efeito \\\hline\hline 
		
		3.5.1 (a) e (b) & Contestando decis\~ao do \'arbitro e conduta do \Gls{dugout}. &
		Efeito: 
		
		\begin{enumerate}[label=\alph*)]
			\item Na primeira falta, a equipe ser\'a advertida. 
			
			\item  Na reincid\^encia, o infrator ser\'a expulso. 
		\end{enumerate}
		\\\hline
		3.51 (c) & Conduta antidesportiva. &
		
		\begin{enumerate}[label=\alph*)]
			\item Na primeira infra\c{c}\~ao, o infrator pode ser advertido. 
			\begin{enumerate}[label=\roman* -]
				\item Se a primeira infra\c{c}\~ao \'e grave, o \'arbitro deve expulsar o infrator. 
				\item Na segunda infra\c{c}\~ao, o infrator \'e expulso. 
			\end{enumerate}
			\item  Um membro da equipe expulso do jogo deve ir diretamente para o vesti\'ario pelo 
			resto do jogo, ou deixar o campo. 
			
			\item  Se uma pessoa expulsa dessa maneira n\~ao deixar o jogo, imediatamente, justificar\'a 
			um confisco do jogo. 
			
			\item  Um \'arbitro encarregado pode denunciar um membro da equipe por conduta ofensiva, abuso verbal ou f\'isico, a qualquer momento ap\'os um jogo ter sido encerrado, caso em que o membro da equipe denunciado deve comparecer perante a organiza\c{c}\~ao sob a qual o jogo ou torneio est\'a sendo realizado.
		\end{enumerate}
\end{tabular}}
\begin{multicols}{2}
	\subsection{\'ARBITROS} 
	\subsubsection{DIREITOS E OBRIGA\c{C}\~OES} 
%\begin{multicols}{2}
	Os \'arbitros s\~ao os representantes da Liga ou Organiza\c{c}\~ao pela qual tenham sido escalados para um jogo espec\'ifico e, como tais, est\~ao autorizados a, e incumbidos de fazer cumprir estas Regras. Eles t\^em autoridade para mandar um jogador, \gls{coach}, capit\~ao ou t\'ecnico fazer ou deixar de fazer qualquer coisa que, na sua opini\~ao, seja necess\'aria para garantir o cumprimento de uma ou todas estas regras, e para impor as penalidades prescritas nestas regras. O \'arbitro de \gls{home} tem autoridade para dar decis\~oes sobre quaisquer situa\c{c}\~oes que n\~ao estejam especificamente cobertas por estas Regras. 
	
	\subsubsection{\'ARBITRO DE \textit{HOME} }
	O \'arbitro de \gls{home} \'e respons\'avel por: 
	
	\begin{enumerate}[label=\alph*)]
		\item Decidir sobre a condi\c{c}\~ao do campo de jogo para um jogo. 
		
		\item  Posicionar-se atr\'as do \gls{homeplate} e atr\'as do receptor. 
		
		\item  Ter o comando total do jogo e ser o respons\'avel pela correta condu\c{c}\~ao da partida. 
		
		\item  Declarar \glspl{ball} e \glspl{strike}. 
		
		\item  Em harmonia e em colabora\c{c}\~ao com os \'arbitros de base, tomar decis\~oes sobre jogadas, bolas batidas, \gls{fair} ou \gls{foul}, bolas pegas legalmente ou ilegalmente. Em jogadas nas quais o \'arbitro de base tem de deixar o campo interno, o \'arbitro de \gls{home} deve assumir as obriga\c{c}\~oes normalmente exigidas do \'arbitro de base. 
		
		\item  Determinar e declarar se 
		\begin{enumerate}[label=\roman* -]
			\item um batedor bate a bola por meio de \gls{bunt}; ou 
			\item uma bola batida toca o corpo ou a roupa do batedor. 
		\end{enumerate}
		
		\item  Tomar decis\~oes nas bases quando solicitado a faz\^e-lo. 
		
		\item  Determinar quando um jogo \'e confiscado. 
		
		\item  Assumir todas as obriga\c{c}\~oes quando \'e designado como um \'arbitro \'unico para um jogo. 
	\end{enumerate}
	
	
	\subsubsection{\'ARBITRO DE BASE} 
	
	\begin{enumerate}[label=\alph*)]
		\item Um \'arbitro de base deve posicionar-se devidamente no campo de jogo, de acordo com o sistema de arbitragem adotado. 
		
		\item  Um \'arbitro de base deve auxiliar o \'arbitro de \gls{home}, de todas as maneiras, para aplicar estas Regras. 
	\end{enumerate}
	
	\subsubsection{RESPONSABILIDADES DE UM \'ARBITRO ÚNICO }
	Se for designado somente um \'arbitro, suas obriga\c{c}\~oes e a jurisdi\c{c}\~ao devem estender-se para todos os casos protegidos por estas Regras. A posi\c{c}\~ao inicial do \'arbitro para cada arremesso deve ser atr\'as do \gls{homeplate} e atr\'as do receptor. Em cada bola batida ou jogada que se desenvolve, o \'arbitro deve sair de tr\'as do \gls{homeplate} e mover-se para dentro do campo interno, a fim de obter a melhor posi\c{c}\~ao para observar qualquer jogada em andamento. 
	\subsubsection{TROCA DE \'ARBITROS }
	Um \'arbitro n\~ao pode ser trocado durante um jogo com a anu\^encia das equipes oponentes, a menos que ele se machuque ou adoe\c{c}a e n\~ao consiga continuar atuando. 
	
	\subsubsection{APRECIA\c{C}ÃO DO \'ARBITRO }
	
	\begin{enumerate}[label=\alph*)]
		\item N\~ao deve haver reclama\c{c}\~ao alguma sobre qualquer decis\~ao de um \'arbitro no campo, com o pretexto de que ele n\~ao fora correto em sua decis\~ao sobre: 
		se uma bola batida foi \gls{fair} ou \gls{foul}, se um corredor foi \gls{safe} ou \gls{out}, se uma bola arremessada foi \gls{strike} ou \gls{ball}; 
		ou sobre qualquer jogada cuja decis\~ao implique uma aprecia\c{c}\~ao. 
		
		Nenhuma decis\~ao dada por um \'arbitro deve ser mudada, a menos que ele seja convencido de que tal decis\~ao infringe uma destas Regras. 
		Em caso de o t\'ecnico, capit\~ao ou qualquer das duas equipes solicitar a mudan\c{c}a de uma decis\~ao baseada unicamente em um ponto das Regras, o \'arbitro cuja decis\~ao esteja sendo questionada deve, se estiver em d\'uvida, consultar seus companheiros, antes de tomar qualquer atitude. Somente o t\'ecnico ou o capit\~ao de uma equipe est\'a legalmente autorizado para reclamar sobre uma decis\~ao e pedir a sua mudan\c{c}a, com a alega\c{c}\~ao de que ela est\'a em conflito com estas Regras. 
		
		\item  Em nenhuma circunst\^ancia um \'arbitro deve pedir para mudar uma decis\~ao dada por seus companheiros, ou criticar ou interferir nas suas atribui\c{c}\~oes, a n\~ao ser que seja por eles requisitado para isso. 
		\item  Os \'arbitros consultados podem retificar qualquer situa\c{c}\~ao em que a mudan\c{c}a de decis\~ao de um \'arbitro, ou uma decis\~ao demorada de um \'arbitro, coloca um batedor-corredor ou corredor em risco, ou coloca a equipe na defensiva em desvantagem. Essa corre\c{c}\~ao n\~ao ser\'a poss\'ivel depois que um arremesso, legal ou ilegal, tiver sido efetuado, ou se todos os jogadores da equipe na defensiva tiverem abandonado o territ\'orio \gls{fair}. 
	\end{enumerate}
	
	\subsubsection{PARALISA\c{C}ÃO DA PARTIDA} 
	
	\begin{enumerate}[label=\alph*)]
		\item Um \'arbitro deve paralisar a partida quando, na sua opini\~ao, as circunst\^ancias justificam tal medida. 
		
		\item  A partida deve ser paralisada quando o \'arbitro de \gls{home} deixa sua posi\c{c}\~ao para limpar o \gls{homeplate}, ou para cumprir outras obriga\c{c}\~oes que n\~ao estejam diretamente relacionadas com decis\~ao de jogadas.
		\item O \'arbitro deve paralisar a partida sempre que um batedor ou um arremessador sai da posi\c{c}\~ao por um motivo justo. 
		
		\item  Um \'arbitro n\~ao deve declarar \gls{time} depois que o arremessador inicia o movimento de arremesso. 
		
		\item  Um \'arbitro n\~ao deve declarar \gls{time} enquanto alguma jogada est\'a em andamento. 
		\item  Em caso de ferimento, exceto quando, na opini\~ao do \'arbitro, esse ferimento \'e muito grave (que pode pôr o jogador em risco), n\~ao deve ser declarado \gls{time} at\'e que todas as jogadas em andamento tenham sido completadas, ou os corredores tenham sido mantidos em suas bases. 
		
		\item  Os \'arbitros n\~ao devem paralisar a partida, a pedido de jogadores, \glspl{coach} ou 	t\'ecnicos, at\'e que todas as a\c{c}\~oes em andamento –de ambas as equipes– tenham sido 
		completadas. 
	\end{enumerate}
	
%\end{multicols}
	\subsection*{EFEITOS}
\end{multicols}
{\footnotesize
	\begin{tabular}{p{.07\columnwidth}p{.25\columnwidth}|p{.60\columnwidth}}
		\multicolumn{2}{c|}{Regra} & Efeito \\\hline\hline 
		
		3.6.7 & Paralisa\c{c}\~ao da partida. 
		
		\gls{time} declarado devido a um ferimento grave que p\~oe um jogador em risco. 
		&
		
		
		Quando \'e declarado \gls{time} em caso de ferimento, a bola torna-se morta, e o(s) corredor (es) pode (m) ser autorizado (s) a avan\c{c}ar uma ou mais bases que ele (s) teria (m) conquistado, na opini\~ao do \'arbitro, se n\~ao tivesse (m) se machucado. 
\end{tabular}}

\subsection{ANOTADORES}
\begin{multicols}{2}
	\subsubsection{RESPONSABILIDADES DO ANOTADOR OFICIAL }
	O Anotador Oficial: 
	
	\begin{enumerate}[label=\alph*)]
		\item Deve preparar ou mandar preparar e guardar os dados de um jogo, conforme est\'a estabelecido nestas Regras. 
		
		\item  Deve ser a \'unica autoridade para anotar todas as decis\~oes envolvendo aprecia\c{c}\~ao. 
		\item  Deve determinar se o avan\c{c}o de um batedor \`a primeira base \'e o resultado de uma batida indefens\'avel (hit) ou de um erro. 
		\item  N\~ao deve tomar uma decis\~ao sobre anota\c{c}\~ao que seja contradit\'oria ou que conflite com estas Regras ou com uma decis\~ao do \'arbitro. 
	\end{enumerate}
	
\end{multicols}

\chapter{ARERMESSADOR}
%\begin{multicols}{2}  
	
	\section{DEFINI\c{C}\~OES} 
	\begin{description}
		\item[\textit{CHARGED DEFENSIVE CONFERENCE}]\footnote{N.T.: Confer\^encia Defensiva } Ocorre quando um \'arbitro concede tempo \`a equipe na defensiva ou paralisa a partida para permitir que: 
		
		\begin{enumerate}[label=\alph*)]
			\item um representante da equipe na defensiva entre no campo de jogo para comunicar-
			se com qualquer defensor; ou 
			
			\item  um defensor vá ao \Gls{dugout} e d\'a ao \'arbitro raz\~ao para acreditar que ele (defensor) tenha recebido instru\c{c}\~oes. 
		\end{enumerate}
	\item[\Gls{crow hop}]\footnote{N.T.: pulo do corvo/salto alavanca} \'E o ato de um arremessador que: 

	\begin{enumerate}[label=\alph*)]
		\item d\'a o impulso de um lugar que n\~ao o \gls{pitcher's plate} para soltar a bola; ou 
		
		\item  d\'a um passo para fora do \gls{pitcher's plate}, estabelecendo um segundo impulso (ou 		ponto de partida), e depois, iniciando desse novo ponto de partida, completa o 
		arremesso. 
	\end{enumerate}

	\item[ARREMESSADOR ILEGAL]: \'E um jogador que est\'a legalmente no jogo, mas n\~ao pode arremessar por ter sido removido da posi\c{c}\~ao de arremessador pelo \'arbitro;
	
		\item[SALTO]
			
	\begin{enumerate}[label=\alph*)]
		\item O arremessador se eleva do solo em seu primeiro movimento, com um impulso \`a 
		frente do \gls{pitcher's plate}, sem estabelecer um segundo ponto de partida como no 
		\gls{crow hop}. 
		
		\item  O p\'e de apoio pode se desprender e/ou arrastar numa a\c{c}\~ao cont\'inua e com o 
		\'impeto criado pelo movimento (\`a frente) do arremessador faz com que todo o corpo e 
		ambos os p\'es (o p\'e de apoio e o p\'e livre com o qual d\'a o passo) estejam no ar ao 
		mesmo tempo e se direcionem ao \gls{homeplate}. 
		
		\item  O arremesso \'e completado quando o arremessador toca o solo e, com um 
		movimento cont\'inuo, joga a bola em dire\c{c}\~ao ao \gls{homeplate}. 
	\end{enumerate} 	
	\item[\Gls{passedball}]\footnote{N.T.: Bola passada}: bola arremessada defens\'avel que passa para tr\'as do receptor. 
	
	\'E um arremesso que deveria ter sido agarrado ou controlado pelo receptor, com um esfor\c{c}o normal. 
	
	\item[Arremesso]: \'E o ato executado pelo arremessador, que consiste em jogar a bola ao batedor. 
	
	\item[P\'e de apoio]: \'E o p\'e com o qual o arremessador d\'a o impulso a partir do \gls{pitcher's plate}.
	
	\item[Arremesso de retorno r\'apido]: Arremesso efetuado com o claro prop\'osito de pegar o batedor desprevenido, ou seja, quando ele n\~ao est\'a devidamente posicionado no \gls{batter's box}, ou enquanto ele ainda est\'a fora de equil\'ibrio por causa do arremesso anterior. 

	\end{description}
	
	
	


	
	
	\section{REUNIÃO DA DEFENSIVA} 
	\subsection{CONFERÊNCIA DEFENSIVA }
	Uma reuni\~ao inclui jogadores posicionados no campo que deixam seus postos e se dirigem ao \Gls{dugout} para receber instru\c{c}\~oes, independentemente de ter havido pedido de \gls{time} ou n\~ao. 
	
	Uma Confer\^encia Defensiva  termina quando o membro da equipe na defensiva 
	cruza a linha de \gls{foul} para retornar ao \Gls{dugout}, ou um defensor retorna ao campo.
	
	\begin{enumerate}[label=\alph*)]
		\item A equipe na defensiva tem direito a somente tr\^es (3) Confer\^encias Defensivas num jogo de sete \glspl{inning}. 
		
		\item  Confer\^encias Defensivas  s\~ao cumulativas, ou seja, a contagem de confer\^encias 
		n\~ao recome\c{c}a com a entrada de um novo arremessador. 
		
		\item  Confer\^encias Defensivas   n\~ao realizadas em sete \glspl{inning} n\~ao podem ser usadas em jogos com \glspl{inning} extras. 
		
		\item  Num jogo com \glspl{inning} extras, \'e permitida somente uma Confer\^encia Defensiva em cada \gls{inning} extra. 

		Uma Confer\^encia Defensiva  n\~ao realizada num \gls{inning} extra de um jogo n\~ao pode ser usada em \gls{inning} extra subsequente. 
		
	\end{enumerate}
%\end{multicols}
\section*{EFEITOS}

{\footnotesize
	\begin{tabular}{p{.08\columnwidth}p{.15\columnwidth}|p{.70\columnwidth}}
		\multicolumn{2}{c|}{Regra} & Efeito \\\hline\hline 
		
		4.2.1 (a) & &Na quarta reuni\~ao, e em cada Confer\^encia Defensiva  adicional num jogo de sete 
		\glspl{inning}, ou em qualquer Confer\^encia Defensiva  que exceda o limite de uma 
		reuni\~ao por \gls{inning} num jogo com \glspl{inning} extras, o arremessador que est\'a no jogo 
		durante a reuni\~ao \'e declarado um Arremessador Ilegal, e ele n\~ao pode voltar a arremessar pelo resto do jogo, mas pode jogar em outra posi\c{c}\~ao defensiva. \\\hline
\end{tabular}}

\begin{multicols}{2}
	
	\subsection{O QUE NÃO \'E UMA CONFERÊNCIA DEFENSIVA }
	
	
	N\~ao \'e uma Confer\^encia Defensiva  quando: 
	
	\begin{enumerate}[label=\alph*)]
		\item um t\'ecnico, \gls{coach} ou membro da equipe na defensiva comunica uma mudan\c{c}a de arremessador ao \'arbitro de \gls{home}, antes ou depois de conversar com o arremessador; 
		
		\item  um t\'ecnico ou \gls{coach} comunica, do \Gls{dugout}, uma mudan\c{c}a ao \'arbitro e, depois de fazer a mudan\c{c}a, cruza a linha de \gls{foul} para falar com algum defensor; 
		
		\item  um ou mais membros da equipe na defensiva e pelo menos um defensor se re\'unem durante uma CONFERÊNCIA ofensiva, contanto que todos os defensores estejam posicionados e prontos para reiniciar a partida quando a ofensiva terminar a reuni\~ao; 
		
		\item  instru\c{c}\~oes s\~ao passadas do \Gls{dugout}; 
		
		\item  um t\'ecnico/\gls{coach} que est\'a atuando como jogador se re\'une com um defensor. 
		
		O \'arbitro pode controlar as reuni\~oes entre o t\'ecnico-jogador/\gls{coach}-jogador e um arremessador; primeiramente, deve adverti-lo, e se isso continuar, dever\'a expuls\'a-lo; 
		ou 
		
		\item  um \'arbitro tiver paralisado o jogo. 
	\end{enumerate}
	
\end{multicols}
	\section{ARREMESSO LEGAL -- REQUISITOS} 

\begin{multicols}{2}	
	
	\subsection{A\c{C}ÃO PRELIMINAR ANTES DE EFETUAR UM ARREMESSO }
	Antes de efetuar um arremesso, as a\c{c}\~oes abaixo devem ocorrer: 
	
	\begin{enumerate}[label=\alph*)]
		\item Todos os jogadores t\^em de estar posicionados em territ\'orio \gls{fair}, e o receptor tem de estar dentro do \gls{catcher's box} e numa posi\c{c}\~ao para receber o arremesso. 
		
		\item  O arremessador, de posse da bola, tem de estar sobre ou perto do \gls{pitcher's plate}. 
		
		\item  O arremessador tem de ter o p\'e de apoio em contato com o \gls{pitcher's plate} e 
		ambos os p\'es dentro dos 61,00cm (24 polegadas) de extens\~ao do \gls{pitcher's plate}. 
		
		Os quadris t\^em de estar alinhados com a primeira e a terceira bases. 
		\footnote{Somente na modalidade Arremesso Modificado -- O arremessador tem de ter ambos 
		os p\'es em contato com o \gls{pitcher's plate} e dentro dos 61,00cm (24 polegadas) de 
		extens\~ao do \gls{pitcher's plate}. 
	
		Os ombros t\^em de estar alinhados com a primeira e a terceira bases.}
		\item O arremessador tem de receber -- ou aparentar estar recebendo -- uma senha do 
		receptor enquanto est\'a sobre o \gls{pitcher's plate}, com as m\~aos separadas e a bola na 
		luva ou na m\~ao com a qual efetua os arremessos. 
		
		\item  Ap\'os receber o sinal, o arremessador deve levar o seu corpo inteiro a uma 
		imobilidade total e completa, com a bola em ambas as m\~aos na frente do corpo. O p\'e 
		com o qual d\'a o passo (p\'e livre) tem de estar im\'ovel no in\'icio e durante a pausa. 
		
		O p\'e livre pode mover-se para a frente somente com o in\'icio do arremesso. Qualquer 
		movimento para tr\'as feito com o p\'e livre durante ou depois da pausa \'e uma a\c{c}\~ao 
		ilegal. 
		
		Essa posi\c{c}\~ao tem de ser mantida por n\~ao menos de dois (2) segundos e n\~ao 
		mais de cinco (5) segundos antes de soltar a bola. 
		
		Quando o arremessador segura a bola com ambas as m\~aos ao lado do corpo, deve-se considerar que ele o fez na frente do corpo. 
		\footnote{Somente na modalidade Arremesso Modificado -- Essa posi\c{c}\~ao tem de ser mantida por n\~ao menos de dois (2) segundos e n\~ao mais de dez (10) segundos antes de soltar a bola}.
	\end{enumerate}
	
	\subsection{INÍCIO DO ARREMESSO }
	\label{sssec:InicioArremesso}
	\begin{enumerate}[label=\alph*)]
		\item O arremesso inicia quando o arremessador tira uma m\~ao da bola ou faz qualquer 
		movimento relacionado com sua maneira de arremessar. 
		
		O arremessador n\~ao pode usar um movimento de arremesso em que, ap\'os assumir a posi\c{c}\~ao de arremesso com a bola em ambas as m\~aos, faz um movimento para tr\'as e para a frente e retorna a bola para ambas m\~aos na frente do corpo. 
		
		\item  O p\'e de apoio tem de permanecer em contato com o \gls{pitcher's plate} antes do in\'icio do arremesso. 
		
		Levantar o p\'e de apoio do \gls{pitcher's plate} e retorn\'a-lo ao \gls{plate}, criando um movimento de impulso -- balan\c{c}ando o corpo para cima e para baixo -- \'e um ato ilegal. 
		
		\footnote{Somente na modalidade Arremesso Modificado -- Ambos os p\'es t\^em de permanecer em contato com o \gls{pitcher's plate} antes do in\'icio do arremesso. 
		
		Levantar o p\'e de apoio do \gls{pitcher's plate} e retorn\'a-lo ao \gls{plate}, criando um movimento de impulso – balan\c{c}ando o corpo para cima e para baixo -- \'e um ato ilegal.} 
	\end{enumerate}
	
	\subsection{ARREMESSO LEGAL ARREMESSO R\'APIDO}
	Para um arremesso ser considerado legal, tem de ser observado o seguinte: 
	
	\begin{enumerate}[label=\alph*)]
		\item O arremessador tem de jogar a bola ao batedor, imediatamente, ap\'os fazer os movimentos de arremesso;
		
		\item O arremessador pode apenas fazer uma rota\c{c}\~ao quando est\'a adotando o estilo \gls{windmill} (estilo ‘molinete'). 
		
		Entretanto, ele pode deixar o bra\c{c}o cair para o lado e para tr\'as antes de iniciar o movimento de ‘molinete'. Isso permite que o bra\c{c}o passe legalmente pelo quadril duas vezes; 
		
		\item  O arremesso tem de ser feito com um movimento em que a m\~ao do arremessador fique em n\'ivel inferior ao do cotovelo (\gls{underhanded motion}) e abaixo do quadril, e o punho n\~ao pode estar mais afastado do corpo do que o cotovelo; n\~ao pode haver uma parada ou revers\~ao do movimento para a frente. 
		
		\item  O arremessador tem de soltar a bola com um movimento cont\'inuo da m\~ao e do punho para a frente, passando em linha reta e paralelamente ao corpo. 
		
		\item  No momento de completar o arremesso, o arremessador pode dar um passo com o p\'e livre (p\'e com o qual d\'a o passo) simultaneamente ao ato de soltar a bola. 
		
		O passo tem de ser para a frente, na dire\c{c}\~ao do batedor e dentro do espa\c{c}o de 61,00cm (24 polegadas), que corresponde \`a extens\~ao do \gls{pitcher's plate}. N\~ao \'e um passo se o arremessador desliza qualquer dos p\'es atrav\'es do \gls{pitcher's plate}, desde que seja mantido o contato com a placa e n\~ao haja um movimento para tr\'as da placa. 
		
		\item  O p\'e de apoio tem de permanecer em contato com o \gls{pitcher's plate}, ou pode desprender-se e arrastar-se para fora da placa, ou estar no ar, antes que o p\'e livre toque o solo. 
		
		O arremessador pode saltar do \gls{pitcher's plate}, aterrissar e, com um movimento cont\'inuo, arremessar ao batedor. O p\'e de apoio pode acompanhar o movimento cont\'inuo do arremessador. 
		
		\item  Todo movimento do bra\c{c}o com o qual faz os arremessos tem de ser cont\'inuo enquanto o arremessador d\'a o passo, impulsiona ou salta do \gls{pitcher's plate}. 
		
		\item  O impulso do arremessador para arrastar, pular ou saltar tem de iniciar do \gls{pitcher's plate}. O arremessador n\~ao deve dar aquele salto conhecido por \gls{crow hop} (pulo do corvo/salto-alavanca) ou impulsionar de qualquer lugar que n\~ao seja o \gls{pitcher's plate}. 
		
		\item  O bra\c{c}o do arremesso pode continuar o movimento desde que a rota\c{c}\~ao (wind up) n\~ao continue. 
		
		\item  O arremessador n\~ao deve derrubar a bola, ou faz\^e-la rolar ou saltar, 
		intencionalmente, a fim de evitar que ela seja batida. 
		
		\item  O arremessador tem vinte (20) segundos ap\'os receber a bola, ou depois que o 
		\'arbitro declara \gls{play}, para efetuar o arremesso seguinte. 
	\end{enumerate}
	
	\subsection{ARREMESSO MODIFICADO} 
	Para um arremesso ser considerado legal, deve ser observado o seguinte: 
	\begin{enumerate}[label=\alph*)]
		\item O arremessador tem de jogar a bola ao batedor, imediatamente, ap\'os fazer os 
		movimentos de arremesso. 
		
		\item  O arremessador pode levar a bola para tr\'as do dorso quando movimenta o bra\c{c}o 
		para tr\'as, contanto que n\~ao haja uma parada ou revers\~ao do movimento para a frente 
		e ele n\~ao use um arremesso do tipo \gls{windmill} (‘molinete') ou \gls{slingshot} (‘p\^endulo'); 
		n\~ao \'e permitido que ele fa\c{c}a uma rota\c{c}\~ao completa do bra\c{c}o quando executa o 
		arremesso. 
		
		\item  A bola tem de estar na parte interna do pulso do arremessador quando ele 
		movimenta o bra\c{c}o para baixo e durante a conclus\~ao do arremesso. 
		
		\item  O arremesso tem de ser feito com um movimento em que a m\~ao do arremessador 
		fique em n\'ivel inferior ao do cotovelo (\gls{underhanded motion}) e abaixo do quadril, e a 
		palma da m\~ao pode estar virada para baixo. 
		
		\item  Ao movimentar para a frente o bra\c{c}o com o qual faz os arremessos: 
		\begin{enumerate}[label=\roman* -]
			\item  o cotovelo tem de estar ‘fechado' (direcionado) para o ponto de onde vai soltar a 
			bola; e 
			\item  os ombros e o quadril que impulsiona e conduz o arremesso t\^em de formar um \^angulo reto com o \gls{homeplate} quando a bola sai da m\~ao do arremessador. 
		\end{enumerate}
		\item  A bola tem de ser solta no primeiro impulso para a frente com o bra\c{c}o que utiliza para arremessar, e este tem de passar o quadril. O movimento para soltar a bola tem de ser completo e suave, sem parada brusca do bra\c{c}o perto do quadril. 
		
		\item  Impulsionar com o p\'e de apoio de um lugar que n\~ao seja o \gls{pitcher's plate}, antes que o p\'e livre tenha deixado a placa, \'e um \gls{crow hop} (pulo do corvo/salto-alavanca), e \'e ilegal. 
		
		\item  No momento de completar o arremesso, o arremessador tem de dar um passo simultaneamente ao ato de soltar a bola. O passo tem de ser para a frente, na dire\c{c}\~ao do batedor e dentro do espa\c{c}o de 61,00cm (24 polegadas), que corresponde \`a extens\~ao do \gls{pitcher's plate} projetada para a frente. O p\'e livre tem de ser direcionado ao \gls{homeplate}, e n\~ao precisa tocar o solo \`a frente de, ou ao longo de uma linha reta entre o p\'e de apoio e o \gls{homeplate}. N\~ao \'e um passo se o arremessador desliza qualquer dos p\'es sobre o \gls{pitcher's plate}, desde que o contato com a placa seja mantido. Levantar o p\'e de apoio do \gls{pitcher's plate} e retorn\'a-lo \`a placa, criando um movimento de impulso, \'e um ato ilegal que infringe a \autoref{sssec:InicioArremesso}.
		
		\item  O bra\c{c}o do arremessador pode acompanhar o movimento para soltar a bola, desde, por\'em, que esse movimento n\~ao continue depois de soltar a bola.	\item O arremessador tem de jogar a bola ao batedor, sem inten\c{c}\~ao de faz\^e-la rolar ou saltar ap\'os tocar o solo, a fim de evitar que ela seja batida. 
		
		\item  O arremessador tem vinte (20) segundos ap\'os receber a bola, ou depois que o \'arbitro declara \gls{play}, para efetuar o arremesso seguinte. 
	\end{enumerate}
	
	\subsection{POSICIONAMENTO DA DEFESA} 
	
	\begin{enumerate}[label=\alph*)]
		\item Um defensor n\~ao deve agir, ou ocupar uma posi\c{c}\~ao, com inten\c{c}\~ao antidesportiva de distrair um batedor. 
		
		\item  Quando o corredor da 3ª base est\'a tentando anotar ponto por meio de um \gls{squeeze play} (jogada de press\~ao) ou \gls{steal} (roubo de base), nenhum defensor pode ficar sobre no, ou na frente do \gls{homeplate}, sem estar de posse da bola, ou tocar o batedor ou o seu \gls{bat}. 
	\end{enumerate}
	
	\subsection{SUBST\^aNCIAS ESTRANHAS} 
	
	\begin{enumerate}[label=\alph*)]
		\item Nenhum membro da equipe na defensiva pode, em nenhum momento durante o jogo, aplicar uma subst\^ancia estranha na bola. Um arremessador que lambe os dedos da m\~ao com a qual efetua os arremessos tem de sec\'a-los antes de ter contato com a bola. 
		
		\item  Sob a supervis\~ao e controle do \'arbitro, pode ser usada resina em p\'o para secar as m\~aos; esse material tem de ser deixado sobre o solo (atr\'as do \gls{pitcher's plate}), dentro do C\'irculo do Arremessador, quando n\~ao est\'a em uso. 
		
		\item  \'E permitido o uso de pano aprovado impregnado somente com resina, para secar a m\~ao, e esse pano tem de ser guardado no bolso traseiro ou mantido preso no cinto. 
		
		\item  Nenhum defensor pode aplicar resina na bola, ou na luva para depois pôr a bola em contato com a resina. 
		
		\item  O arremessador n\~ao pode usar fita em seus dedos, ou um \gls{sweatband} (faixa para enxugar suor), bracelete, ou outros objetos similares no pulso ou antebra\c{c}o (do bra\c{c}o com o qual faz os arremessos). 
		
		Se um arremessador necessita usar \gls{sweatband} ou fita em seu bra\c{c}o com o qual faz os arremessos em raz\~ao de um ferimento, ambos os bra\c{c}os t\^em de ser cobertos com uma camiseta. 
	\end{enumerate}
	
	\subsection{O RECEPTOR}
	
	\begin{enumerate}[label=\alph*)]
		\item O receptor tem de permanecer no \gls{catcher's box} at\'e que seja completado o arremesso. 
		
		\item  O receptor tem de devolver a bola -- imediatamente e diretamente -- ao arremessador, ap\'os cada arremesso, inclusive depois de um \gls{foulball}, exceto: 
		
		\begin{enumerate}[label=\roman* -]
			\item depois de uma elimina\c{c}\~ao por \gls{strike} (\gls{strikeout}); 
			\item quando o batedor se torna um batedor-corredor; 
			\item quando h\'a algum corredor em base; 
			\item quando ele pega uma bola \gls{foul} perto da linha de \gls{foul} e lan\c{c}a a qualquer base para tentar eliminar um corredor; ou 
			\item quando, em situa\c{c}\~ao de \gls{check swing} em terceiro \gls{strike} n\~ao agarrado, ele lan\c{c}a \`a 1ª base para eliminar o batedor-corredor. 
		\end{enumerate}
	\end{enumerate}
	
	\subsubsection{LAN\c{C}AMENTO A UMA BASE}
	Ap\'os ter assumido a posi\c{c}\~ao de arremesso, o arremessador n\~ao deve lan\c{c}ar ou simular um lan\c{c}amento a uma base durante uma situa\c{c}\~ao de bola viva enquanto seu p\'e est\'a em contato com o \gls{pitcher's plate}. Se isso ocorrer durante uma Jogada de Apela\c{c}\~ao com bola viva, a apela\c{c}\~ao ser\'a cancelada. O arremessador pode parar ou sair da posi\c{c}\~ao de arremesso, dando um passo para tr\'as do \gls{pitcher's plate} antes de separar as m\~aos. Se o passo for dado para a frente ou para os lados, ser\'a aplicada a penalidade de um Arremesso Ilegal. 
	
\end{multicols}

\section*{EFEITOS}

(4.3.1 a 4.3.7) 

\resizebox{.95\textwidth}{!}{
	\begin{tabular}{p{.08\columnwidth}p{.50\columnwidth}|p{.45\columnwidth}}
		\multicolumn{2}{c|}{Regra} & Efeito \\\hline\hline 
		4.3.3 (k) & Um arremessador n\~ao solta a bola em 20 segundos. &
		
		\'E concedido um \gls{ball} ao batedor. \\\hline
		
		4.3.4 (a) & Um defensor age de maneira antidesportiva ou se posiciona para distrair o batedor. N\~ao \'e necess\'ario que seja efetuado arremesso. &
		
		O jogador \'e expulso do jogo. 
		\\\hline
		4.3.4 (b) & Um defensor fica parado na frente do \gls{homeplate}, sem estar de posse da bola, ou toca o batedor ou o \gls{bat} num poss\'ivel \gls{squeeze play} (jogada de press\~ao). 
		& A bola torna-se morta. O batedor \'e autorizado a ir \`a primeira base por Obstru\c{c}\~ao, \'e declarado um Arremesso Ilegal, e todos os corredores avan\c{c}ar\~ao uma base.\\\hline 
		
		4.3.5 & Um membro da equipe na defensiva continua aplicando uma subst\^ancia estranha na bola ou continua infringindo qualquer dispositivo da Regra 4.3.5. 
		&O arremessador \'e expulso do jogo. \\\hline
		
		4.3.6 (b) & Um receptor n\~ao devolve a bola diretamente ao arremessador quando n\~ao h\'a corredor (es) em base. 
		& \'E concedido um \gls{ball} ao batedor.\\\hline 
	\end{tabular}
}


\vspace{10mm}

\resizebox{.95\textwidth}{!}{
	\begin{tabular}{p{.08\columnwidth}p{.15\columnwidth}|p{.90\columnwidth}}
		\multicolumn{2}{c|}{Regra} & Efeito \\\hline\hline 
		4.3.1 a 4.3.7 & Por uma infra\c{c}\~ao das Regras 4.3.1 a 4.3.7 –Movimentos de arremesso impr\'oprios [exce\c{c}\~ao para os efeitos expostos acima pelas Regras 4.3.3 (k), 4.3.5 e 4.3.6 (b) ] resultam em um Arremesso Ilegal sendo declarado. 
		& Isto \'e uma Bola Morta Demorada, e devem ser aplicadas as seguintes normas e penalidades: 
		
		\begin{enumerate}[label=\alph*)]
			\item Se o Arremesso Ilegal n\~ao for batido, ser\'a concedido um \gls{ball} extra ao batedor (ser\'a concedida a primeira base se for o quarto \gls{ball}), e cada corredor avan\c{c}ar\'a uma	base.
			
			Se um corredor avan\c{c}ar legalmente num Arremesso Ilegal, \Gls{passedball} (arremesso que poderia ter sido agarrado ou controlado pelo receptor, com um esfor\c{c}o normal) ou lan\c{c}amento descontrolado do receptor, qualquer base extra obtida poder\'a ser mantida. 
			
			Se o corredor for declarado \gls{out} depois de avan\c{c}ar uma base, a elimina\c{c}\~ao ser\'a mantida. 
			
			\item  Se o batedor bater o Arremesso Ilegal, a equipe na ofensiva ter\'a o direito de optar pela aplica\c{c}\~ao da penalidade de um Arremesso Ilegal ou pelo resultado da jogada. Se o batedor bater o Arremesso Ilegal e alcan\c{c}ar a primeira base, e se todos os outros corredores avan\c{c}arem pelo menos uma base na jogada, o Arremesso Ilegal ser\'a anulado, e todas as a\c{c}\~oes resultantes da jogada ser\~ao mantidas; nenhuma op\c{c}\~ao ser\'a dada. 
			
			\item  Se o batedor erra ao tentar bater um Arremesso Ilegal depois de dois \glspl{strike}, o receptor derruba a bola mas consegue eliminar o batedor-corredor na primeira base, e os outros corredores avan\c{c}am pelo menos uma base, a equipe na ofensiva tem o direito de optar pela aplica\c{c}\~ao da penalidade de um Arremesso Ilegal ou pelo resultado da jogada. Se o batedor-corredor alcan\c{c}ar a primeira base em consequ\^encia de um terceiro \gls{strike} n\~ao agarrado e todos os outros corredores avan\c{c}arem pelo menos uma base, o Arremesso Ilegal ser\'a anulado, e todas as a\c{c}\~oes resultantes da jogada ser\~ao mantidas; nenhuma op\c{c}\~ao ser\'a dada. 
			
			\item  Se o t\'ecnico da equipe na ofensiva n\~ao aceitar o resultado da jogada, a bola ficar\'a morta. O \'arbitro conceder\'a um \gls{ball} ao batedor (a primeira base se for o quarto \gls{ball}) e todos os corredores ser\~ao autorizados a avan\c{c}ar uma base. 
			
			\item  Se um Arremesso Ilegal atinge o batedor, a bola torna-se morta; o batedor \'e autorizado a ir \`a 1ª base e todos os corredores avan\c{c}am uma base. Nenhuma op\c{c}\~ao ser\'a dada. 
		\end{enumerate}
		\\\hline
\end{tabular}}


\subsection{ARREMESSOS DE AQUECIMENTO}
\begin{multicols}{2}
	\begin{enumerate}[label=\alph*)]
		\item No in\'icio do primeiro \gls{inning} de ambas as equipes, ou quando um arremessador substitui outro, s\~ao permitidos no m\'aximo cinco (5) arremessos de aquecimento, que devem ser feitos ao receptor ou a outro membro da equipe na defensiva, dentro de um (1) minuto. No in\'icio de cada metade de \gls{inning} (depois do primeiro \gls{inning}), o arremessador do \gls{inning} anterior tem um (1) minuto para efetuar tr\^es (3) arremessos de aquecimento. Se tiver passado, ou estiver prestes a passar, um minuto, sem que o arremessador tenha iniciado os arremessos de aquecimento, o \'arbitro lhe permitir\'a somente um (1) arremesso. 
		\begin{enumerate}[label=\roman* -]
			\item  Se a equipe na defensiva n\~ao utiliza um jogador do \gls{bench} para receber os arremessos de aquecimento enquanto o receptor (que estava ocupando uma base, ou estava no \gls{batter's box} ou no c\'irculo do Batedor Prevenido) se prepara para ocupar a sua posi\c{c}\~ao, o \'arbitro permitir\'a somente um (1) arremesso, exceto quando um novo arremessador entra no jogo. 
			\item Isso n\~ao se aplica se o \'arbitro retarda o in\'icio ou o rein\'icio da partida por causa de substitui\c{c}\~ao, reuni\~ao, ferimentos ou outras raz\~oes citadas pelo \'arbitro de \gls{home}. 
		\end{enumerate}
		\item  A partida fica paralisada enquanto s\~ao efetuados os arremessos de aquecimento. 
		
		\item  Um arremessador que retorna para arremessar na mesma metade de \gls{inning} n\~ao tem direito a arremessos de aquecimento. 
	\end{enumerate}
	
\end{multicols}

\section*{EFEITOS}

{\footnotesize
	\begin{tabular}{p{.08\columnwidth}p{.30\columnwidth}|p{.45\columnwidth}}
		
		R4.4 & Arremessos de aquecimento excedentes. &
		Ser\'a concedido um \gls{ball} ao batedor por cada arremesso excedente. 
\end{tabular}}


\subsection{ARREMESSO NULO}
%\begin{multicols}{2}
	O arremesso \'e anulado, a bola torna-se morta e todas as a\c{c}\~oes subsequentes a esse arremesso s\~ao canceladas por um \'arbitro quando: 
	
	\begin{enumerate}[label=\alph*)]
		\item O arremessador efetua o arremesso enquanto a partida est\'a paralisada. 
		
		\item  O arremessador tenta um retorno r\'apido da bola: 
		
		\begin{enumerate}[label=\roman* -]
			\item  antes que o batedor tenha se posicionado no \gls{batter's box}; ou 
			\item quando o batedor est\'a fora de equil\'ibrio em consequ\^encia de um arremesso 
			anterior. 
		\end{enumerate}
		
		\item  Um corredor \'e declarado \gls{out} por ter deixado uma base antes de o arremessador soltar a bola de sua m\~ao. 
		
		\item  O arremessador inicia o arremesso antes que um corredor tenha retornado sua base ap\'os um \gls{foulball} ter sido declarado. 
		
		\item  Um t\'ecnico, \gls{coach} ou jogador declara ou pede \gls{time}, usa qualquer outra palavra ou frase, ou pratica qualquer outro ato enquanto a bola est\'a viva e em jogo, com o evidente prop\'osito de tentar fazer o arremessador cometer um Arremesso Ilegal. 	Nesse caso, o \'arbitro deve advertir a equipe infratora, e se qualquer membro da equipe advertida repetir esse tipo de ato, ser\'a expulso do jogo. 
	\end{enumerate}
%\end{multicols}

\section{BOLA DERRUBADA} 
\begin{multicols}{2}
	Se a bola escapa ou cai da m\~ao do arremessador durante o arremesso: 
	
	\begin{enumerate}[label=\alph*)]
		\item O \'arbitro de \gls{home} declara um \gls{ball} ao batedor. 
		
		\item  A bola permanece em jogo. 
		
		\item  Um corredor pode avan\c{c}ar a seu pr\'oprio risco. 
	\end{enumerate}
	
\end{multicols}

\section{RETORNO DE ARREMESSADOR}
\begin{multicols}{2}
	N\~ao h\'a limite quanto ao n\'umero de vezes que um jogador pode retornar \`a posi\c{c}\~ao de arremessador, contanto que esse jogador n\~ao tenha sa\'ido do jogo ou n\~ao tenha sido declarado um Arremessador Ilegal por um \'arbitro. 
	
	
\end{multicols}

\section{ARREMESSADOR ILEGAL}
\begin{multicols}{2}
	Um jogador que tenha sido declarado um Arremessador Ilegal em raz\~ao de excesso de Confer\^encias Defensivas n\~ao pode retornar \`a posi\c{c}\~ao de arremessador, em nenhum momento, pelo resto do jogo. 
	
\end{multicols}

\subsection*{EFEITOS}

{\footnotesize
	\begin{tabular}{p{.08\columnwidth}p{.25\columnwidth}|p{.60\columnwidth}}
		
		4.8 & Arremessador Ilegal. Arremessador declarado ilegal retorna \`a posi\c{c}\~ao de arremessador e efetua um arremesso, legal ou ilegal. &
		
		\begin{enumerate}[label=\alph*)]
			\item O Arremessador Ilegal \'e expulso. 
			
			\item  Se o Arremessador Ilegal \'e descoberto antes do arremesso seguinte, o t\'ecnico da equipe na ofensiva tem a op\c{c}\~ao de: 
			
			\begin{enumerate}[label=\roman* -]
				\item  aceitar o resultado da jogada, ou 
				
				\item ter a jogada anulada, e nesse caso o batedor bate novamente, assumindo a contagem de \gls{ball} e \gls{strike} que tinha antes da descoberta do Arremessador Ilegal, e 
				
				\item cada corredor retorna \`a base que estava ocupando no momento do arremesso. 
			\end{enumerate}
		\end{enumerate}
		\\\hline
\end{tabular}}


\chapter{BATEDOR E BATEDOR-CORREDOR}

 
	\section{DEFINI\c{C}\~OES - Atletas}

	\begin{description}
	\item[Batedor]: \'E um jogador da ofensiva que entra no \gls{batter's box} com a inten\c{c}\~ao de ajudar sua equipe a anotar pontos. Ele continua sendo um batedor at\'e que seja declarado \gls{out} pelo \'arbitro ou se torne um batedor-corredor. 
	\item[Batedor prevenido]: \'E o jogador da ofensiva cujo nome segue o do batedor do turno na ordem de batedores. 	
	\item[Batedor-corredor]: \'E um jogador que, ap\'os terminar a sua vez de bater, n\~ao foi ainda declarado \gls{out}, nem tocou a primeira base. 
	
	\item[Ordem de batedores]: \'E a lista oficial de jogadores da ofensiva, cujos nomes devem estar relacionados na escala\c{c}\~ao da equipe na sequ\^encia em que devem bater. 
	
		\item[Corredor]: \'E um jogador da equipe na ofensiva que, ao terminar a sua vez de bater, conseguiu alcan\c{c}ar a primeira base e ainda n\~ao est\'a declarado \gls{out}.

	\item[Confer\^encia ofensiva]: Ocorre quando a equipe na ofensiva pede a paralisa\c{c}\~ao da partida para permitir que o t\'ecnico ou outro representante da equipe se re\'una com qualquer membro de sua equipe. Isso inclui o batedor, o corredor, o Batedor Prevenido e os \glspl{coach} entre si. 
	
	N\~ao \'e uma CONFERÊNCIA ofensiva quando:
	\begin{enumerate}[label=\alph*)]
		\item  um arremessador est\'a vestindo um agasalho enquanto est\'a sobre uma base, ou
		\item jogadores da ofensiva se re\'unem enquanto a equipe na defensiva est\'a reunida, ou 
		\item o jogo est\'a paralisado, desde que estejam prontos para jogar
		\item a reuni\~ao da defensiva termina, ou 
		\item o \'arbitro reinicia o jogo. 
	\end{enumerate}
		\'E permitida somente uma confer\^encia por \gls{inning}. 
	\end{description}

	\section{DEFINI\c{C}\~OES - Batidas}
	\begin{description}

	\item[Bola batida]: \'E qualquer bola arremessada que atinge o \gls{bat}, ou \'e por ele atingida, e cai em territ\'orio \gls{fair} ou \gls{foul}.
	
	N\~ao \'e necess\'ario que tenha havido inten\c{c}\~ao de bater a bola. 
	
	\item[\Gls{bunt}]\footnote{N.T.: Toque}: \'E uma bola batida por meio de um toque intencional com o \gls{bat} (sem girar o \gls{bat}), a fim de faz\^e-la rolar lentamente dentro do campo de jogo. 
	
		\item[\gls{slap hit}]\footnote{N.T.: batida colocada}: 	\'E uma bola batida –exceto um \gls{bunt}– com um movimento curto e controlado com o \gls{bat} (o \gls{bat} \'e movimentado de cima para baixo), e n\~ao com um \gls{swing} completo. Os dois tipos mais comuns de \gls{slap hit} s\~ao: 
	
	\begin{enumerate}[label=\alph*)]
		\item aquele em que o batedor assume uma posi\c{c}\~ao como se fosse executar um \gls{bunt}, 	mas depois, ou impulsiona a bola contra o solo fazendo um \gls{swing} r\'apido e curto, ou 	empurra (com o \gls{bat}) a bola para o campo interno; ou
		\item aquele em que o batedor d\'a passos acelerados (dentro do \gls{batter's box}) na dire\c{c}\~ao 	do arremessador, antes de bater o arremesso fazendo um \gls{swing} r\'apido e curto, ou 	antes de empurrar (com o \gls{bat}) a bola para o campo interno. 
	\end{enumerate}
	
	\item[Turno de bater]: Inicia quando um batedor entra no \gls{batter's box} e continua at\'e que ele seja declarado \gls{out} ou se torne um batedor-corredor. 
	
	\end{description}

	\section{DEFINI\c{C}\~OES - Bolas}
	\begin{description}	
	\item[Bola bloqueada]: \'E uma bola batida, lan\c{c}ada ou arremessada que: 
	
	\begin{enumerate}[label=\alph*)]
		\item fica alojada na cerca ou na roupa/equipamento do \'arbitro; 
		
		\item  \'e tocada, parada ou manuseada por uma pessoa que n\~ao est\'a atuando no jogo; 
		
		\item  toca qualquer objeto que n\~ao \'e parte do equipamento oficial ou da \'area de jogo; 
		
		\item  \'e tocada por um jogador da defensiva que est\'a em contato com o solo que n\~ao \'e parte da \'area de jogo (as linhas s\~ao consideradas parte da \'area de jogo).
		
	\end{enumerate}
	Uma bola lan\c{c}ada que toca acidentalmente um \gls{coach} de base (dentro ou fora do \gls{coachsbox}) n\~ao \'e uma Bola Bloqueada, e permanece em jogo. 
	
	\item[Bola morta]: \'E uma bola que n\~ao est\'a em jogo. Nenhuma jogada pode ocorrer. 
	
	\item[\Gls{delayed dead ball}]\footnote{N.T.: Bola morta demorada}: \'E uma situa\c{c}\~ao de jogo em que a bola permanece viva at\'e a conclus\~ao de uma jogada; quando a jogada estiver totalmente conclu\'ida, e se for necess\'ario, um \'arbitro declarar\'a que a bola est\'a morta e aplicar\'a a regra apropriada. 
	
	\item[\gls{line drive} (Bola batida que vai em linha reta)]: \'E uma bola batida em voo que vai em linha reta, com for\c{c}a e diretamente, para dentro do campo de jogo. 
	
	\item[Bola \gls{fair}]: \'E uma bola batida legalmente e viva que: 
	
	\begin{enumerate}[label=\alph*)]
		\item Permanece, ou \'e tocada, em ou sobre territ\'orio \gls{fair} entre o \gls{homeplate} e a primeira base ou entre o \gls{homeplate} e a terceira base. 
		
		\item  Passa pela primeira ou terceira base (rolando ou pulando), em ou sobre territ\'orio \gls{fair}, independentemente de onde ela toque depois de passar sobre a base. 
		
		\item  Toca a primeira, segunda ou terceira base. 
		
		\item  Enquanto est\'a em ou sobre territ\'orio \gls{fair}, toca o corpo ou a roupa de um \'arbitro ou jogador. 
		
		\item  Cai primeiro em territ\'orio \gls{fair} al\'em da primeira e terceira base. 
		
		\item  Enquanto est\'a sobre territ\'orio \gls{fair}, sai do campo de jogo, passando por cima da cerca do campo externo (\gls{outfield}). 
		
		\item  Enquanto em voo, atinge o poste da linha de \gls{foul}. 
		
		\item  \'E julgada \gls{fair fly} (\gls{fly} em territ\'orio \gls{fair}) de acordo com a posi\c{c}\~ao da bola em rela\c{c}\~ao \`a linha de \gls{foul}, incluindo o poste de \gls{foul}, e n\~ao pela posi\c{c}\~ao do defensor – se ele se encontra em territ\'orio \gls{fair} ou \gls{foul}– no momento em que toca a bola. N\~ao importa se a bola toca primeiro o territ\'orio \gls{fair} ou \gls{foul}, desde que ela n\~ao toque qualquer coisa estranha ao terreno natural, em territ\'orio \gls{foul}, e preencha todos os outros requisitos de uma bola \gls{fair}. A posi\c{c}\~ao da bola no momento em que tem contato com o defensor determina se a batida \'e \gls{fair} ou \gls{foul}, sem levar em considera\c{c}\~ao se ela poderia rolar, sem ser tocada, para o territ\'orio \gls{foul} ou \gls{fair}. 
	\end{enumerate}
	
	\item[Bola \gls{fly}]: \'E uma bola batida para o ar. 
	
	\item[bola em v\^oo]:
	\'E qualquer bola batida, lan\c{c}ada ou arremessada que n\~ao tenha ainda tocado o solo ou algum objeto, exceto um defensor. 
	
	\item[\Gls{homeplate}]: \'E uma bola \gls{fair fly} (exceto um \gls{line drive} ou um \gls{fly} resultante de \gls{bunt}) que pode ser pega por um defensor do campo interno mediante um esfor\c{c}o normal, na seguinte situa\c{c}\~ao: a primeira e segunda base, ou a primeira, segunda e terceira base, est\~ao ocupadas e h\'a menos de dois \glspl{out}. 
	
	O arremessador, o receptor e qualquer defensor do campo externo posicionado no campo interno, na jogada, ser\~ao considerados defensores do campo interno para os prop\'ositos desta regra. 
	
	\item[Bola em risco]:
	\'E um termo que indica que a bola est\'a em jogo e um jogador da ofensiva pode ser declarado \gls{out}. 

	\item[Bola \gls{foul}]: \'E uma bola batida legalmente que: 

	\begin{enumerate}[label=\alph*)]
		\item Permanece em territ\'orio \gls{foul} entre o \gls{homeplate} e a primeira base ou entre o \gls{homeplate} e a terceira base. 
		
		\item  Passa pela primeira ou terceira base (rolando ou pulando), em ou sobre territ\'orio \gls{foul}. 
		
		\item  Cai primeiro em territ\'orio \gls{foul} al\'em da primeira ou terceira base. 
		
		\item  Enquanto est\'a em ou sobre territ\'orio \gls{foul}, toca o corpo, o equipamento de jogo – em uso ou removido do lugar onde normalmente \'e usado–, a roupa de um \'arbitro ou jogador, ou qualquer objeto estranho ao terreno natural. 
		
		\item  Toca o batedor, ou o \gls{bat} em suas m\~aos, pela segunda vez, enquanto ele est\'a dentro do \gls{batter's box}.
		\item Vai diretamente do \gls{bat} –sem subir al\'em da cabe\c{c}a do batedor– para qualquer parte do corpo ou equipamento do receptor e \'e pega por outro defensor. 
		
		\item  Atinge o \gls{pitcher's plate} e rola, sem ser tocada, para o territ\'orio \gls{foul}, antes de alcan\c{c}ar a primeira ou terceira base. 
		
		\item  \'E julgada \gls{foul fly} (\gls{fly} em territ\'orio \gls{foul}) de acordo com a posi\c{c}\~ao da bola em rela\c{c}\~ao \`a linha de \gls{foul}, incluindo o poste de \gls{foul}, e n\~ao pela posi\c{c}\~ao do defensor –se ele se encontra em territ\'orio \gls{fair} ou \gls{foul}– no momento em que toca a bola. 
		
		A posi\c{c}\~ao da bola no momento em que tem contato com o defensor determina se a batida \'e \gls{fair} ou \gls{foul}, sem levar em considera\c{c}\~ao se ela poderia rolar, sem ser tocada, para o territ\'orio \gls{foul} ou \gls{fair}. 
	\end{enumerate}

	\item[\Gls{foul tip}]:
	
	\begin{enumerate}[label=\alph*)]
		\item \'E uma bola batida que:
		\begin{enumerate}[label=\roman*.]
			\item vai diretamente do \gls{bat} \`as m\~aos ou \`a luva do receptor; 
			\item n\~ao sobe al\'em da cabe\c{c}a do batedor; e 
			\item \'e agarrada legalmente pelo receptor. 
		\end{enumerate}
		\item  A cada \Gls{foul tip} \'e contado um \gls{strike}, e a bola permanece em jogo. 
		
		N\~ao \'e uma pegada legal se a bola \'e agarrada no rebote, a menos que ela tenha tocado primeiro a m\~ao ou a luva do receptor. 
	\end{enumerate}


	\item[Bola batida ilegalmente]: Ocorre uma batida ilegal quando o batedor bate a bola para o territ\'orio \gls{fair} ou \gls{foul}: 

	\begin{enumerate}[label=\alph*)]
		\item enquanto um p\'e est\'a completamente fora do \gls{batter's box} e sobre o solo no momento em que o \gls{bat} tem contato com a bola; 
		
		\item  enquanto qualquer parte de um p\'e est\'a tocando o \gls{homeplate} no momento em que o \gls{bat} tem contato com a bola; 
		
		\item  com um \gls{bat} ilegal, n\~ao aprovado ou Adulterado;
		
		\item procedendo da seguinte forma: depois de ter dado um passo com qualquer dos p\'es 	inteiramente para fora do \gls{batter's box}, retorna e toca a bola com o \gls{bat} enquanto est\'a com ambos os p\'es dentro do \gls{batter's box}. 
	\end{enumerate}

	\item[Bola pega ilegalmente]: Ocorre uma pegada ilegal quando um defensor pega uma bola batida, lan\c{c}ada ou arremessada, usando seu bon\'e, m\'ascara, luva ou qualquer parte do seu uniforme enquanto tais itens est\~ao fora do lugar apropriado. 

	\item[\Gls{trapped ball}] \'e: 

	\begin{enumerate}[label=\alph*)]
		\item Uma bola batida legalmente para o ar (\gls{fly}) ou que vai em linha reta (\gls{line drive}), que toca o solo ou uma cerca antes de ser pega; 
		
		\item  Uma bola batida legalmente para o ar (\gls{fly}) que \'e pega contra a cerca, com a luva ou com a m\~ao. 
		
		\item  Uma bola lan\c{c}ada a qualquer base para efetivar um \gls{out} for\c{c}ado (\gls{forced out}), que o defensor consegue segurar, colocando a luva SOBRE a bola que est\'a tocando o solo e n\~ao SOB a bola. 
		
		N\~ao ser\'a um \gls{trapped ball} se a bola for pega com a luva sob a bola. 
		
		\item  Uma bola arremessada \gls{strike} que toca o solo antes de o receptor peg\'a-la. 
	\end{enumerate}


	\end{description}

	\section{DEFINI\c{C}\~OES - Lan\c{c}amento}

	\begin{description}	
	\item[Lan\c{c}amento]: \'E o ato de um defensor jogar a bola a outro defensor.

	\item[Lan\c{c}amento descontrolado]: \'E um lan\c{c}amento em que a bola atirada de um defensor a outro n\~ao pode ser pega ou controlada e permanece em jogo. 

	\item[\Gls{overthrow}]\footnote{N.T.:Mau lan\c{c}amento}: \'E uma bola lan\c{c}ada de um defensor a outro, que vai al\'em das linhas que delimitam o campo de jogo ou se torna uma bola bloqueada. 

	\item[\gls{wild pitch}]\footnote{N.T.:Arremesso descontrolado}: \'E um arremesso t\~ao alto, t\~ao baixo, ou t\~ao fora do \gls{homeplate} que o receptor n\~ao consegue parar ou controlar com um esfor\c{c}o normal. 
	
	\item[base por \glspl{ball} ou \gls{walk} ]: Ocorre quando quatro arremessos s\~ao julgados \gls{ball} pelo \'arbitro de \gls{home}, incluindo arremessos ilegais. Ao batedor \'e concedido a primeira base. A bola \'e viva. 


	\item[\Gls{hit by pitch}]\footnote{N.T: Atingido por arremesso}:	\'E quando uma bola arremessada toca qualquer parte do corpo do batedor (incluindo suas m\~aos ou sua roupa) que est\'a dentro do \gls{batter's box}, desde, por\'em, que ele n\~ao tenha girado o \gls{bat} para tentar bater essa bola, ou o arremesso n\~ao tenha sido declarado \gls{strike}. 
	
	N\~ao importa se a bola toca o solo antes de atingir o batedor. 
	\end{description}

\section{DEFINI\c{C}\~OES - Pegada}

	\begin{description}		
		\item[\Gls{catch}]\footnote{N.T.: Pegada legal}: Ocorre uma pegada legal quando um defensor, com sua(s) m\~ao(s) ou luva, pega uma bola batida ou lan\c{c}ada. 
	\end{description}
	
	Para a pegada ser v\'alida:
	\begin{enumerate}[label=\alph*)]
	\item o defensor tem de segurar a bola por um tempo suficiente para provar que ela foi agarrada firmemente, e/ou que seu ato de liberar foi volunt\'ario e intencional. 
	
	\item se o jogador derruba a bola no momento de retir\'a-la da luva –depois de t\^e-la segurado dentro dela– ou no ato do lan\c{c}amento. 
	
	\item  Se a bola est\'a apenas sustentada no (s) bra\c{c}o (s) do defensor, ou se sua queda ao solo \'e evitada por alguma parte do seu corpo, equipamento ou roupa. 
	
	Deve-se considerar que a pegada n\~ao est\'a completa at\'e que a bola esteja dominada com sua(s) m\~ao (s) ou luva. 
	
	\item  os p\'es do defensor t\^em de estar dentro do campo de jogo, tocando a linha demarcat\'oria da \'area fora de jogo, ou estar no ar depois de deixar a \'area de bola viva. 
	
	\item Se o jogador tem o controle da bola quando retorna ao solo na \'area de bola morta, a pegada \'e legal. 
	
	Quando um jogador que est\'a na \'area de bola morta retorna \`a \'area de bola viva, tem de ter ambos os p\'es tocando a \'area de jogo, antes de ter contato com a bola, para a pegada ser v\'alida. 
\end{enumerate}

	N\~ao \'e uma pegada legal se um defensor (enquanto est\'a tentando controlar a bola) colide com outro jogador, \'arbitro ou uma cerca, ou cai ao solo, e em raz\~ao disso derruba a bola. 
	
	Uma bola batida que, enquanto em v\^oo, tem contato com qualquer coisa, exceto um jogador da defensiva, \'e tratada da mesma forma como se ela tivesse tocado o solo. 



	\section{DEFINI\c{C}\~OES - Lance/Jogada}
	\begin{description}	
	\item[Roubo de base]:\'E uma jogada em que um corredor tenta avan\c{c}ar \`a base seguinte ou ao \gls{homeplate} durante ou depois de um arremesso ao batedor. 

	\item[\Gls{double play}]\footnote{N.T.: Jogada dupla}: \'E uma jogada executada pela equipe na defensiva, na qual dois jogadores da equipe na ofensiva s\~ao declarados \gls{out} em consequ\^encia de a\c{c}\~ao cont\'inua. 		
	
	\item[Jogada tripla]: \'E uma jogada de a\c{c}\~ao cont\'inua da defensiva, na qual tr\^es jogadores da ofensiva s\~ao declarados \gls{out}s. rapped ball
	\item[BASE POR \glspl{ball} INTENCIONAL OU \gls{walk} INTENCIONAL]:
	\'E um lance em que a equipe na defensiva concede a primeira base ao batedor, sem arremessar quatro \glspl{ball}. A bola \'e morta. 
	
	\item[BOLA \gls{fly} DERRUBADA INTENCIONALMENTE]: \'E um lance em que, com menos de dois \glspl{out} e corredor na primeira base, um defensor derruba, intencionalmente, uma bola \gls{fair fly} (depois de t\^e-la controlado com a m\~ao ou luva), incluindo um \gls{line drive} ou um \gls{fly} resultante de \gls{bunt}, que pode ser pega por um defensor do campo interno mediante um esfor\c{c}o normal. Um \gls{trapped ball}, ou um \gls{fly} que o defensor deixa cair ao solo sem tocar a bola, n\~ao \'e considerado como uma bola derrubada intencionalmente. Se for declarado um \gls{homeplate}, a Regra de \gls{homeplate} ter\'a prioridade. 
	
	\item[\gls{squeeze play} (JOGADA DE PRESS\~AO)]: \'E uma jogada em que a equipe na ofensiva, com um corredor na terceira base, tenta anotar ponto com esse corredor por meio de um toque na bola dado pelo batedor. 

	
	\item[Interfer\^encia]:	ocorre uma Interfer\^encia quando: 
	\begin{enumerate}[label=\alph*)]
		\item um jogador ou membro da equipe na ofensiva impede, estorva ou confunde um jogador da defensiva que est\'a tentando executar uma jogada; 
		
		\item  um \'arbitro impede que o receptor tente fazer um lan\c{c}amento para eliminar um corredor que est\'a fora da base; 
		
		\item  um \'arbitro ou corredor \'e atingido por uma bola batida \gls{fair}: 
		\begin{enumerate}[label=\roman*.]
			\item  antes de ela tocar um defensor, incluindo o arremessador; 
			\item antes de ela passar um defensor do campo interno, exceto o arremessador, sem ser tocada; ou 
			\item depois de ela passar um defensor, exceto o arremessador, e na opini\~ao do \'arbitro outro defensor teria chance de realizar um \gls{out}; 
		\end{enumerate}
		\item  um espectador entra no campo de jogo ou alcan\c{c}a o campo de jogo e impede que um defensor apanhe a bola, ou tem contato com a bola sobre a qual um defensor est\'a tentando fazer uma jogada. 
	\end{enumerate}
	
	\item[Obstru\c{c}\~ao]: ocorre uma Obstru\c{c}\~ao quando: 
	
	\begin{enumerate}[label=\alph*)]
		\item um jogador ou membro da equipe na defensiva estorva ou impede que um batedor gire o \gls{bat} ou bata uma bola arremessada; 
		
		\item  um defensor impede o avan\c{c}o de um batedor-corredor ou corredor que est\'a correndo as bases legalmente enquanto: 
		\begin{enumerate}[label= \arabic*)]
			\item   n\~ao est\'a de posse da bola; 
			\item   n\~ao est\'a em a\c{c}\~ao para defender uma bola batida; 
			\item   simula um toque (\gls{fake tag}), sem estar de posse da bola; 
			\item   tendo a posse da bola, empurra um corredor para fora da base; ou 
			\item   tendo a posse da bola, n\~ao est\'a em a\c{c}\~ao para fazer uma jogada sobre o batedor-
			corredor ou corredor. 
		\end{enumerate}
	\end{enumerate}
	\item[JOGADA OPCIONAL]: \'E uma jogada em que o t\'ecnico/\gls{coach} da equipe na ofensiva tem a alternativa de aceitar a penalidade da a\c{c}\~ao ilegal ou o resultado da jogada. Tal escolha pode ser feita nos seguintes casos: 

	\begin{enumerate}[label=\roman*.]
		\item  obstru\c{c}\~ao cometida pelo receptor; 
		\item uso de uma luva ilegal; 
		\item uma substitui\c{c}\~ao ilegal; 
		\item um arremesso ilegal; 
		\item um arremessador ilegal retorna ao jogo e efetua arremessos. 
	\end{enumerate}
	
	\item[\Gls{fake tag}]\footnote{N.T.:Simula\c{c}\~ao de toque}:	\'E uma forma de Obstru\c{c}\~ao em que um defensor, sem estar de posse da bola, impede a a\c{c}\~ao de um corredor que est\'a avan\c{c}ando ou retornando a uma base. N\~ao \'e necess\'ario que o corredor pare ou deslize; a simples diminui\c{c}\~ao da velocidade quando ocorre uma simula\c{c}\~ao de toque caracteriza uma Obstru\c{c}\~ao. 	
		
	\item[\gls{out} FOR\c{C}ADO]: \'E aquela que pode ser feita somente quando um corredor perde o direito \`a base que est\'a ocupando porque o batedor se torna um batedor-corredor, e antes que esse batedor-corredor (ou um corredor subsequente) tenha sido declarado \gls{out}. 
	
	Numa jogada de apela\c{c}\~ao, a elimina\c{c}\~ao for\c{c}ada \'e determinada pela situa\c{c}\~ao for\c{c}ada no momento em que \'e feita a apela\c{c}\~ao, e n\~ao no momento da infra\c{c}\~ao. 
	Se a situa\c{c}\~ao for\c{c}ada tiver deixado de existir em raz\~ao da elimina\c{c}\~ao de um corredor subsequente, antes da apela\c{c}\~ao, a jogada n\~ao ser\'a mais de elimina\c{c}\~ao for\c{c}ada. 
	Se um corredor for\c{c}ado a avan\c{c}ar, ap\'os tocar a base seguinte, recuar por qualquer raz\~ao em dire\c{c}\~ao \`a base que havia ocupado anteriormente, a Jogada For\c{c}ada ser\'a restabelecida. 
	
	\item[\gls{over-slide}]\footnote{N.T.: Ultrapassagem da base}: Quando um batedor-corredor ou corredor desliza para uma base que est\'a tentando alcan\c{c}ar, ultrapassa-a e perde o contato com ela; nesse caso, o batedor-corredor/corredor corre o risco de ser declarado \gls{out}. 
	
	Um batedor-corredor pode ultrapassar a primeira base, sem correr o risco de ser declarado \gls{out}, desde que regresse imediatamente \`a primeira base. 	
\end{description}

\section{DEFINI\c{C}\~OES - Campo e equipamento}
\begin{description}
	\item[Equipamento ou uniforme fora do lugar apropriado]: Quando um defensor toca intencionalmente, ou pega, uma bola batida \gls{fair}, uma bola lan\c{c}ada, ou uma bola arremessada, com seu bon\'e, capacete, m\'ascara, protetor, bolso, luva destacada da m\~ao, ou qualquer parte do seu uniforme que esteja fora do lugar apropriado em seu corpo. 
	
	\item[Base deslocada]:	\'E uma base que foi tirada de sua posi\c{c}\~ao correta.
	
	\item[Caminho da base]: \'E a linha reta entre uma base e a posi\c{c}\~ao do corredor no momento em que um jogador da defensiva est\'a tentando (ou est\'a prestes a tentar) toc\'a-lo com a bola. 	
	 
	\item[Zona de \gls{strike}]: \'E o espa\c{c}o –sobre qualquer parte do \gls{homeplate}– entre a {\color{red!80}parte inferior do esterno (caixa tor\'acica	e a parte inferior da r\'otula dos joelhos do batedor quando ele assume a sua postura habitual para bater a bola arremessada. 
		
	Somente Arremesso Modificado -- O espa\c{c}o sobre o \gls{homeplate} entre as axilas do batedor e a parte superior de seus joelhos quando ele assume a sua postura habitual para bater a bola arremessada.) 
		
	A postura habitual para bater a bola \'e aquela que o batedor assume depois que a bola \'e arremessada, quando ele decide se tenta ou n\~ao bat\^e-la. }
\end{description}

\clearpage
\begin{description}	

	
	\item[\gls{tag}]\footnote{N.T.:Toque}: Um \gls{tag} legal \'e o ato de um defensor tocar: 
	
	\begin{enumerate}[label=\alph*)]
		\item Um batedor-corredor ou corredor que est\'a fora da base, com a bola firmemente segura em sua m\~ao ou luva. A bola n\~ao ser\'a considerada firmemente segura se ela estiver ‘pipocando', ou for derrubada pelo defensor, depois de tocar o batedor-corredor ou corredor, a menos que esse jogador bata intencionalmente na bola que o defensor est\'a segurando em sua m\~ao ou luva. 
		
		O corredor tem de ser tocado com a m\~ao ou luva que est\'a segurando a bola. 
		
		\item  Uma base com a bola firmemente segura em sua m\~ao ou luva. A base pode ser tocada com qualquer parte do corpo para ser um \gls{tag} legal \footnote{Exemplo, o defensor pode tocar a base com um p\'e, com uma m\~ao, sentar sobre a base etc.). Isso deve ser aplicado a qualquer elimina\c{c}\~ao for\c{c}ada ou jogada de apela\c{c}\~ao.}. 
	\end{enumerate}
	
	\item[\gls{tagging up}]\footnote{N.T.: Ato de retocar uma base}: \'E o ato de um corredor que volta para sua base, ou nela permanece, antes de avan\c{c}ar legalmente numa bola batida \gls{fly} ap\'os o primeiro contato do defensor com a bola. 
	
	\item[Regra do terceiro \gls{strike}]: Esta regra deve ser aplicada quando o receptor n\~ao agarra o terceiro \gls{strike} antes que a bola toque o solo e: 
	
	\begin{enumerate}[label=\alph*)]
		\item h\'a menos de dois \glspl{out} e a primeira base n\~ao est\'a ocupada; ou 
		
		\item  h\'a dois \glspl{out}. 
	\end{enumerate}


\end{description}

\subsection*{EFEITOS}

{\footnotesize
	\begin{tabular}{p{.08\columnwidth}p{.25\columnwidth}|p{.55	\columnwidth}}\hline\hline
		5.2 (b) & Segunda Confer\^encia. &Expuls\~ao do \gls{coach} ou t\'ecnico que insistir por uma segunda confer\^encia. \\\hline
		
\end{tabular}}


\section{BATEDOR}

	\subsection{BATEDOR PREVENIDO}
		\begin{multicols}{2}
			\begin{enumerate}[label=\alph*)]
				\item No in\'icio de um \gls{inning}, \'e o jogador que deve come\c{c}ar batendo; ele tem de permanecer no C\'irculo do Batedor Prevenido at\'e ser chamado ao \gls{batter's box}. 
				
				\item  Uma vez iniciado o \gls{inning}, \'e o jogador da ofensiva que, na escala\c{c}\~ao de batedores, \'e o pr\'oximo a entrar no \gls{batter's box}. 
				
				\item  O Batedor Prevenido: 
				
				\begin{enumerate}[label=\roman*.]
					\item   Pode permanecer dentro de quaisquer dos C\'irculos do Batedor Prevenido, de 
					modo que fique atr\'as do batedor, e n\~ao do lado aberto do batedor. 
					\item   Tem de usar um capacete. 
					\item   Pode exercitar com dois \glspl{bat} oficiais de softbol (no m\'aximo), um \gls{warm-up bat} (\gls{bat} para fazer aquecimento) aprovado, ou com combina\c{c}\~ao que n\~ao exceda o limite de dois \glspl{bat}. Um \gls{bat} com o qual o Batedor Prevenido est\'a exercitando n\~ao pode ter nada anexado a ele, a n\~ao ser um acess\'orio aprovado pela WBSC-SD ou ISF. 
					\item   Pode deixar o c\'irculo do Batedor Prevenido: 
					\begin{enumerate}[label=\arabic*)]
						\item  quando chega a sua vez de bater; 
						\item para orientar os corredores que avan\c{c}am da terceira base para \gls{homeplate}; ou 
						\item para evitar uma poss\'ivel Interfer\^encia numa bola \gls{fly} ou numa bola lan\c{c}ada. 
					\end{enumerate}	
				\end{enumerate}
				\item  N\~ao deve interferir na a\c{c}\~ao de um jogador da defensiva com chance de fazer uma jogada. 
			\end{enumerate}
			
		\end{multicols}
	
	\subsection*{EFEITOS}

{\footnotesize
	\begin{tabular}{p{.08\columnwidth}p{.20\columnwidth}|p{.60\columnwidth}}\hline\hline
		
		5.3 (d) & Atrapalha um jogador da defensiva que tem oportunidade de fazer uma jogada. 
		& a bola \'e morta e, 
		\begin{enumerate}[label=\arabic*)]
			\item  Se a Interfer\^encia \'e cometida quando um jogador da defensiva est\'a tentando eliminar um corredor, 
			
			\begin{enumerate}[label=\alph*)]
				\item o corredor que est\'a mais perto do \gls{homeplate} no momento da Interfer\^encia \'e declarado \gls{out}; e 
				\item  outros corredores devem retornar \`a \'ultima base tocada no momento da Interfer\^encia, a menos que sejam for\c{c}ados a avan\c{c}ar porque o batedor-corredor se tornara um corredor.
			\end{enumerate}	
			\item  Se a Interfer\^encia \'e cometida quando um jogador da defensiva est\'a tentando pegar uma bola \gls{fly}, ou sobre uma bola \gls{fly} que um defensor est\'a tentando pegar, 
			
			\begin{enumerate}[label=\alph*)]
				\item o batedor-corredor deve ser declarado \gls{out}, e 
				
				\item  os corredores devem retornar \`a base que estavam ocupando no momento do arremesso. 
			\end{enumerate}
		\end{enumerate}\\\hline
		5.3 (c) (ii) & O batedor se recusa a usar capacete quando ordenado a faz\^e-lo.&
		Ap\'os uma advert\^encia, o jogador deve ser expulso do jogo. \\\hline
		5.3 (e) & O batedor usa equipamento de aquecimento ilegal. &O equipamento de aquecimento ilegal deve ser removido do jogo. O jogador que continuar usando o equipamento que fora removido ser\'a expulso do jogo. \\\hline
\end{tabular}}


\subsection{ORDEM DE BATEDORES }
\begin{multicols}{2}
	
	\begin{enumerate}[label=\alph*)]
		\item A ordem de batedores tem de ser seguida durante todo o jogo, a menos que um jogador seja substitu\'ido por outro, e nesse caso o substituto tem de ocupar o lugar do jogador substitu\'ido na ordem de batedores. 
		
		\item  O primeiro batedor em cada \gls{inning} deve ser o batedor cujo nome vem em seguida ao do \'ultimo batedor que completou o turno de bater no \gls{inning} anterior. 
		
		\item  Quando o terceiro \gls{out} num \gls{inning} ocorre antes de o batedor completar o turno de bater, esse batedor tem de ser o primeiro batedor no pr\'oximo \gls{inning}. A contagem de \gls{ball} e \gls{strike} \'e cancelada. 
		
		\item  Um jogador bate fora de ordem quando o batedor correto deixa de bater na sequ\^encia em que est\'a relacionado no formul\'ario de escala\c{c}\~ao da equipe. 
	\end{enumerate}
	
\end{multicols}

\subsection*{EFEITOS}

\resizebox{\textwidth}{!}{\footnotesize
	\begin{tabular}{p{.08\columnwidth}p{.07\columnwidth}|p{.90\columnwidth}}
		
		5.4.1 & Batedor fora de ordem. &
		Isto \'e uma jogada de apela\c{c}\~ao que pode ser feita pelo t\'ecnico, \gls{coach} ou jogador da equipe na defensiva. A equipe na defensiva perde o seu direito de apelar sobre um batedor fora de ordem quando todos os jogadores da defensiva tiverem deixado o territ\'orio \gls{fair}, a caminho do \gls{bench} ou \Gls{dugout}. 
		
		\begin{enumerate}[label=\alph*)]
			\item Quando o erro \'e descoberto enquanto o batedor incorreto est\'a no \gls{batter's box}: 
			\begin{enumerate}[label=\roman*.]
				\item O batedor correto pode ocupar legalmente o seu lugar e assumir a contagem de \gls{ball} e \gls{strike} do batedor incorreto. 
				\item Qualquer ponto anotado ou avan\c{c}o nas bases enquanto o batedor incorreto estava no \gls{batter's box} \'e legal. 
			\end{enumerate}
			\item  Quando o erro \'e descoberto ap\'os o batedor incorreto ter completado a sua vez de bater, e antes de ser efetuado um arremesso, legal ou ilegal, a outro batedor: 
			
			\begin{enumerate}[label=\roman*.]
				\item O jogador que deveria ter batido \'e declarado \gls{out}. 
				\item Qualquer avan\c{c}o ou ponto anotado em consequ\^encia da a\c{c}\~ao do batedor impr\'oprio que se tornara um batedor-corredor ser\'a anulado. Qualquer \gls{out} feito antes de descobrir esta infra\c{c}\~ao ser\'a mantida. 
				\item O pr\'oximo batedor \'e o jogador cujo nome vem em seguida ao do jogador declarado \gls{out} por n\~ao ter batido na sua vez. Se o pr\'oximo jogador for o batedor incorreto declarado \gls{out}, dever\'a bater aquele que o segue na ordem de batedores. 
				\item Se o batedor impr\'oprio for declarado \gls{out}, ele n\~ao poder\'a bater no mesmo \gls{inning} at\'e que todos os outros batedores na ordem de batedores tenham completado a sua vez de bater. Se a sua vez de bater chegar antes disso, deve bater o Batedor Prevenido. 
				\item Se o batedor declarado \gls{out} nessas circunst\^ancias completar o terceiro \gls{out}, o batedor correto no \gls{inning} seguinte ser\'a o jogador que iria bater em seguida caso a elimina\c{c}\~ao tivesse ocorrido em uma jogada normal. 
				\item Se o terceiro \gls{out} \'e feito sobre um batedor-corredor ou corredor antes da descoberta da infra\c{c}\~ao, ainda pode ser feita uma apela\c{c}\~ao para restabelecer a ordem de batedores correta. 
			\end{enumerate}
			\item  Se o erro \'e descoberto depois do primeiro arremesso, legal ou ilegal, ao pr\'oximo 
			Batedor: 
			\begin{enumerate}[label=\roman*.]
				\item A situa\c{c}\~ao do batedor incorreto fica legalizada. 
				\item Todos os pontos anotados e os avan\c{c}os de corredores s\~ao legais. 
				\item O pr\'oximo batedor na ordem de batedores \'e aquele cujo nome vem em seguida ao do batedor incorreto. 
				\item Ningu\'em \'e declarado \gls{out} por ter deixado de bater. 
				\item Jogadores que tenham deixado de bater e n\~ao tenham sido declarados \gls{out}s perdem a sua oportunidade de bater, e t\^em de aguardar at\'e chegar novamente a sua vez na ordem normal de batedores. 
			\end{enumerate}
			\item  Nenhum corredor ser\'a removido da base que est\'a ocupando, para bater no seu turno correto. Ele simplesmente perde a sua vez de bater, sem qualquer penalidade. O batedor cujo nome vem em seguida ao seu, na ordem de batedores, torna-se o batedor legal. Isso n\~ao se aplica a um batedor-corredor que tenha sido retirado da base pelo \'arbitro, de acordo com a Se\c{c}\~ao (b) (ii) acima. 
		\end{enumerate}
\end{tabular}}

\begin{multicols}{2}
	\subsection{EXIGÊNCIAS PARA BATEDOR }
	
	\begin{enumerate}[label=\alph*)]
		\item Um batedor tem de usar um capacete aprovado
		\item Um batedor tem de ocupar a sua posi\c{c}\~ao no \gls{batter's box} dentro de 10 segundos depois que o \'arbitro declara \gls{play ball}. 
		
		\item  Nenhum membro da equipe na ofensiva pode apagar as linhas do \gls{batter's box}, em nenhum momento na reuni\~ao pr\'e-jogo ou durante um jogo. 
		
		\item  O batedor tem de ter ambos os p\'es completamente dentro do \gls{batter's box} antes do in\'icio do arremesso. Os p\'es do batedor podem tocar as linhas, mas nenhuma parte de um p\'e pode estar fora das linhas antes ao arremesso. 
		
		\item  Ap\'os entrar no \gls{batter's box}, o batedor tem de manter pelo menos um p\'e inteiramente dentro do \gls{box} entre arremessos, a menos que: 
		
		\begin{enumerate}[label=\roman*.]
			\item a bola seja batida para o territ\'orio \gls{fair} ou \gls{foul}; 
			\item o \'impeto de um \gls{swing} ou tentativa de \gls{swing}, que inclui um \gls{slap} (movimento curto e controlado do \gls{bat}) ou \gls{swing} interrompido, leve o 
			batedor para fora do \gls{batter's box}; 
			\item seja for\c{c}ado a sair do \gls{batter's box} por um arremesso; 
			\item ocorra um \gls{wild pitch} ou \Gls{passedball}; 
			\item  haja uma tentativa de jogada em qualquer base; 
			\item seja declarado \gls{time}; 
			\item  o arremessador deixe o c\'irculo do arremessador ou o receptor deixe o 
			\gls{catcher's box}; ou 
			\item o \'arbitro decida que o arremesso efetuado depois de contados tr\^es \glspl{ball} \'e um \gls{strike}, e o batedor ache que foi um \gls{ball}. 
		\end{enumerate}
	\end{enumerate}
	
	\subsection{\glspl{ball} E \glspl{strike}}
	Cada bola arremessada legalmente que n\~ao \'e batida pelo batedor \'e declarada \gls{ball} ou \gls{strike} pelo \'arbitro de \gls{home}. 
	
	\begin{enumerate}[label=\alph*)]
		\item \'E declarado um \gls{ball}, e a bola permanece viva, a menos que ela se torne morta por qualquer outra raz\~ao: 
		\begin{enumerate}[label=\roman*.]
			\item quando um batedor n\~ao gira o \gls{bat} numa bola arremessada que n\~ao entra na zona de \gls{strike}, toca o \gls{homeplate}, ou toca o solo, antes de chegar ao \gls{homeplate}; 
			\item quando o receptor n\~ao devolve a bola diretamente ao arremessador conforme exige a regra; ou 
			\item  quando o arremessador n\~ao efetua o arremesso dentro de 20 segundos. 
		\end{enumerate}
		\item  \'E declarado um \gls{ball}, e a bola fica morta: 
		
		\begin{enumerate}[label=\roman*.]
			\item por cada bola arremessada ilegalmente que n\~ao \'e batida pelo batedor; 
			
			\item quando o t\'ecnico prefere n\~ao aceitar o resultado da jogada depois que a bola \'e batida; ou 
			\item por cada arremesso de aquecimento excedente. 
		\end{enumerate}
		
		\item  \'E declarado um \gls{strike}, a bola permanece viva, e os corredores podem avan\c{c}ar correndo o risco de serem declarado \gls{out}s: 
		
		\begin{enumerate}[label=\roman*.]
			\item quando qualquer parte de uma bola arremessada entra na zona de \gls{strike}, sem tocar o solo, e o batedor n\~ao gira o \gls{bat}, (
			{\color{red!80}Somente Arremesso R\'apido -- desde que o topo da bola esteja no –ou abaixo do– esterno, ou a parte inferior da bola esteja na –ou acima da– parte mais baixa da r\'otula dos joelhos}); 
			\item por cada bola arremessada legalmente que o batedor tenta bater e erra (seu \gls{bat} n\~ao tem contato com a bola); 
			\item  por cada \gls{foul tip}. 
		\end{enumerate}
		\item  \'E declarado um \gls{strike}, a bola torna-se morta, e os corredores t\^em de retornar a suas bases, sem o risco de serem declarado \gls{out}s, mas n\~ao precisam tocar as bases intermedi\'arias: 
		
		\begin{enumerate}[label=\roman*.]
			\item quando uma bola arremessada atinge o batedor enquanto a bola est\'a na zona de \gls{strike}; 
			\item  por cada bola arremessada que o batedor tenta bater e erra (o \gls{bat} n\~ao tem contato com a bola), e essa bola toca qualquer parte do batedor; 
			\item por cada bola \gls{foul} quando o batedor tem menos de dois \glspl{strike}; 
			\item quando qualquer parte do corpo ou roupa do batedor \'e atingida pela bola batida enquanto ele est\'a dentro \gls{batter's box}, e a contagem de arremessos \'e menos de dois \glspl{strike}; 
			\item quando o batedor n\~ao entra no \gls{batter's box} dentro de 10 segundos depois que o \'arbitro declara \gls{play ball} (n\~ao \'e necess\'ario que o arremessador tenha efetuado um arremesso); 
			\item quando um membro da equipe na ofensiva apaga, intencionalmente, as linhas do \gls{batter's box}; 
			\begin{enumerate}[label=\arabic*)]
				\item  se um batedor apagar as linhas, o \'arbitro declarar\'a um \gls{strike} (n\~ao \'e necess\'ario que o arremessador tenha efetuado um arremesso); 
				
				\item quando o \gls{coach} ou um membro da equipe que n\~ao seja um jogador apaga as linhas, deve ser declarado um \gls{strike} ao pr\'oximo batedor relacionado (ou seu substituto) no formul\'ario de escala\c{c}\~ao da equipe; 
				
				\item se qualquer pessoa continuar apagando as linhas, intencionalmente, ap\'os a primeira infra\c{c}\~ao, essa pessoa ser\'a expulsa do jogo; 
			\end{enumerate}
			\item quando o batedor sai do \gls{batter's box} colocando ambos os p\'es para fora das linhas e retarda o jogo, e nenhuma das exce\c{c}\~oes \'e aplicada (n\~ao \'e necess\'ario que o arremessador tenha efetuado um arremesso). \end{enumerate}
	\end{enumerate}
	
	\subsection{O BATEDOR \'E DECLARADO \gls{out}}
	
	\begin{enumerate}[label=\alph*)]
		\item A bola permanece viva, e os corredores podem avan\c{c}ar correndo o risco de serem declarado \gls{out}s quando: 
		\begin{enumerate}[label=\roman*.]
			\item  O receptor agarra um terceiro \gls{strike} (um arremesso que o batedor deixa passar sem girar o \gls{bat} ou tenta bater, ou um \gls{foul tip}). 
			\item Com menos de dois \glspl{out} e a primeira base ocupada, \'e declarado o terceiro \gls{strike}. 
		\end{enumerate}	
		\item  A bola \'e declarada morta, e os corredores t\^em de retornar \`a base que estavam ocupando no momento do arremesso, mas n\~ao precisam tocar as bases intermedi\'arias quando o batedor: 
		
		\begin{enumerate}[label=\roman*.]
			\item  Tenta bater um terceiro \gls{strike} e erra (o \gls{bat} n\~ao tem contato com a bola), e a bola toca qualquer parte do corpo do batedor; ou n\~ao tenta bat\^e-lo, e a bola arremessada atinge o batedor enquanto o arremesso est\'a na zona de \gls{strike}. 
			\item N\~ao usa um capacete apropriado para batedor quando ordenado a faz\^e-lo pelo \'arbitro. 
			\item Entra no \gls{batter's box} com um \gls{bat} Adulterado ou ilegal, ou \'e descoberto usando um \gls{bat} Adulterado ou ilegal. Nesse caso, o \gls{bat} \'e retirado do jogo. Se o \gls{bat} tiver sido Adulterado, o batedor ser\'a expulso do jogo. 
			
			\item Um p\'e est\'a completamente fora das linhas do \gls{batter's box} e tocando o solo, ou qualquer parte de um p\'e est\'a tocando o \gls{homeplate}, quando ele bate a bola, independentemente de a bola batida ser \gls{fair} ou \gls{foul}. v. Deixa o \gls{batter's box} para ganhar um impulso na corrida, mas est\'a dentro do \gls{batter's box} quando seu \gls{bat} tem contato com a bola; se n\~ao tiver contato com a bola arremessada, n\~ao haver\'a penalidade. 
			\item Muda de um \gls{batter's box} para outro, passando na frente do receptor, enquanto o arremessador est\'a recebendo a senha ou aparenta estar recebendo uma senha, com os p\'es em contato com o \gls{pitcher's plate}, ou a qualquer momento depois disso antes de o arremesso ser completado. 
			\item Toca uma bola \gls{fair} com o \gls{bat}, pela segunda vez, em territ\'orio \gls{fair}, a menos que: 
			
			\begin{enumerate}[label=\arabic*)]
				\item Ele esteja dentro do \gls{batter's box}, e o contato \'e feito enquanto o \gls{bat} est\'a em suas m\~aos. \'E declarado um \gls{foulball}. 
				\item Ele derrube o \gls{bat} e a bola role contra esse \gls{bat} em territ\'orio \gls{fair}; e, na opini\~ao do \'arbitro, n\~ao houve inten\c{c}\~ao alguma de interferir no curso dessa bola; a bola batida deve ser julgada, se \'e \gls{fair} ou \gls{foul}, dependendo de onde ela para ou \'e tocada primeiro por um defensor. 
			\end{enumerate}
		\end{enumerate}
		\item  A bola \'e declarada morta, e os corredores t\^em de retornar \`a \'ultima base que, na opini\~ao do \'arbitro, estavam ocupando no momento em que o batedor cometeu a Interfer\^encia, ou seja: 
		
		\begin{enumerate}[label=\roman*.]
			\item  Saiu do \gls{batter's box} e atrapalhou o receptor que estava pegando ou lan\c{c}ando a bola. 
			\item Estorvou, intencionalmente, o receptor enquanto estava dentro do \gls{batter's box}. 
			\item  Interferiu numa jogada no \gls{homeplate}. 
			\item Interferiu, intencionalmente, numa bola lan\c{c}ada enquanto estava dentro ou fora do \gls{batter's box}. 
		\end{enumerate}
		
	\end{enumerate}
\end{multicols}
	\section{BATEDOR-CORREDOR} 
\begin{multicols}{2}
	
	\subsection{O BATEDOR TORNA-SE UM BATEDOR-CORREDOR}
	
	\begin{enumerate}[label=\alph*)]
		\item Quando acerta uma batida \gls{fair} ou \gls{foul} legalmente. A bola \'e viva em batida \gls{fair} 
		ou em \gls{foul fly} pego no ar. A bola \'e morta em batida \gls{ground} declarada \gls{foul}. 
		
		\item  De acordo com a Regra do Terceiro \gls{strike}. A bola \'e viva. 
		
		\item  Tem de avan\c{c}ar \`a primeira base e toc\'a-la: 
		
		\begin{enumerate}[label=\roman*.]
			\item Quando o \'arbitro de \gls{home} declara o quarto \gls{ball} e a bola est\'a viva. 
			
			\item Quando a equipe na defensiva resolve conceder quatro \glspl{ball}, intencionalmente, ao batedor. A comunica\c{c}\~ao dessa inten\c{c}\~ao ao \'arbitro de \gls{home} pode ser feita pelo arremessador, receptor ou \gls{coach} principal. A bola torna-se morta. 
			
			\begin{enumerate}[label=\arabic*)]
				\item  A comunica\c{c}\~ao ao \'arbitro deve ser entendida como quatro \glspl{ball} efetuados pelo arremessador. A comunica\c{c}\~ao pode ser feita a qualquer momento antes de o batedor iniciar e completar a sua vez de bater, independentemente da contagem de arremessos. 
				
				\item Se tal concess\~ao for feita a dois batedores, o segundo batedor n\~ao poder\'a ser autorizado a ir \`a primeira base enquanto o primeiro batedor n\~ao a tiver alcan\c{c}ado. Se o \'arbitro, equivocadamente, permite que dois batedores ‘andem' ao mesmo tempo, e o primeiro batedor deixa de tocar a primeira base, nenhuma apela\c{c}\~ao por omiss\~ao de base deve ser aceita sobre o primeiro batedor. 
				
				\item A bola torna-se morta e os corredores n\~ao podem avan\c{c}ar, a menos que sejam for\c{c}ados. 
			\end{enumerate}
		\end{enumerate}
		\item  Quando o receptor ou qualquer outro jogador da defensiva obstrui/estorva o batedor, ou impede que ele gire o \gls{bat} ou bata uma bola arremessada. 
		
		\item  Quando uma bola \gls{fair} atinge o corpo, equipamento que est\'a usando ou a roupa do \'arbitro, ou um corredor. 
		
		\item  Quando \'e atingido por um arremesso. As m\~aos do batedor n\~ao s\~ao consideradas uma parte do \gls{bat}. A bola torna-se morta e o batedor adquire o direito de ir \`a primeira base, sem o risco de ser declarado \gls{out}. Se o batedor n\~ao tentar evitar ser atingido pela bola arremessada, o \'arbitro declarar\'a um \gls{ball} e n\~ao conceder\'a uma base. A bola ficar\'a fora de jogo (bola morta). 
		
		\item  \'E declarado um \gls{homerun} quando uma bola batida \gls{fly} que est\'a em territ\'orio \gls{fair}: 
		
		\begin{enumerate}[label=\roman*.]
			\item  Passa sobre a cerca em territ\'orio \gls{fair}. 
			\item  Bate na luva ou corpo do defensor e passa diretamente sobre a cerca em territ\'orio \gls{fair}, ou toca o topo da cerca em territ\'orio \gls{fair} e passa sobre essa cerca. 
			\item  Toca o poste de \gls{foul}, acima do n\'ivel da cerca. 
			\item \'E tocada por um defensor, que est\'a em \'area de bola morta, e essa bola, na  opini\~ao do \'arbitro, teria passado sobre a cerca em territ\'orio \gls{fair}.
			
			N\~ao \'e um \gls{homerun} se uma bola batida \gls{fly} que est\'a em territ\'orio \gls{fair}: 
			
			\begin{enumerate}[label=\arabic*)]
				\item Passa sobre a cerca a uma dist\^ancia menor do que aquela prescrita na Regra *** 2, Anexo 1 (a) e Anexo ***1 (f) –essa dist\^ancia deve ser marcada para orienta\c{c}\~ao do \'arbitro. 
				\item Bate na luva ou no corpo do defensor e passa sobre a cerca em territ\'orio \gls{foul}. 
				\item Toca primeiro a cerca, desvia ap\'os ter conato com um defensor e depois passa sobre a cerca. 
				\item \'E tocada por um defensor, que est\'a em \'area de bola morta, e essa bola, na opini\~ao do \'arbitro, n\~ao teria passado sobre a cerca em territ\'orio \gls{fair}. 
			\end{enumerate}
		\end{enumerate}
		\item Quando qualquer pessoa, exceto um membro da equipe, entra no campo de jogo e interfere: 
		
		\begin{enumerate}[label=\roman*.]
			\item Numa bola batida \gls{ground} que est\'a em territ\'orio \gls{fair}. 
			\item  Na a\c{c}\~ao de um defensor que est\'a prestes a efetuar uma defesa ou pegar uma bola lan\c{c}ada. 
			\item  Na a\c{c}\~ao de um defensor que est\'a prestes a lan\c{c}ar uma bola. 
			\item  Numa bola lan\c{c}ada por um defensor. 
		\end{enumerate}
	\end{enumerate}
\end{multicols}

\subsection*{EFEITOS}

{\footnotesize
	\begin{tabular}{p{.01\columnwidth}p{.15\columnwidth}|p{.75\columnwidth}}
		
		
		& 5.5.1 (d) Um jogador da defensiva impede que o batedor gire o \gls{bat} ou bata uma bola arremessada&. 
		
		
		\begin{enumerate}[label=\arabic*)]
			\item O \'arbitro deve fazer o gesto de Bola Morta Demorada (\gls{delayed dead ball}). A bola permanece viva at\'e a jogada ser conclu\'ida. 
			
			\item  O t\'ecnico da equipe na ofensiva tem a op\c{c}\~ao de: aceitar a concess\~ao pela falta cometida pelo receptor (Obstru\c{c}\~ao), ou aceitar o resultado da jogada. 
			
			\item  Se o batedor bater a bola e chegar a salvo (\gls{safe}) \`a primeira base, e se todos os outros corredores tiverem avan\c{c}ado pelo menos uma base em consequ\^encia da bola batida, a Obstru\c{c}\~ao ser\'a cancelada. Uma vez que um corredor passa uma base, mesmo sem pis\'a-la, ele \'e considerado como se tivesse alcan\c{c}ado essa base. Toda a\c{c}\~ao resultante da bola batida ser\'a mantida. Nenhuma op\c{c}\~ao ser\'a dada. 
			
			\item Se o t\'ecnico n\~ao aceitar o resultado da jogada, ser\'a aplicada a regra de \textsl{Obstru\c{c}\~ao do Receptor}, e nesse caso ser\'a concedida a primeira base ao batedor; os outros corredores ser\~ao autorizados a avan\c{c}ar somente se forem for\c{c}ados. 
		\end{enumerate}
		\\\hline
		& 5.5.1 (e) Uma bola \gls{fair} atinge o corpo, o equipamento que est\'a usando ou a roupa do \'arbitro ou de um corredor. &
		\begin{enumerate}[label=\arabic*)]
			\item Depois de tocar um defensor (incluindo o arremessador). A bola permanece em jogo. 
			\item Depois de passar um defensor, exceto o arremessador, e nenhum outro defensor tinha chance de eliminar um corredor. A bola permanece em jogo. 
			\item Antes de passar um defensor, exceto o arremessador, sem ter sido tocada. A bola torna-se morta. 
		\end{enumerate}\\\hline
\end{tabular}}

\vspace{5mm}
%	%\begin{multicols}{2}
\subsection{O BATEDOR-CORREDOR \'E DECLARADO \gls{out}} 
\begin{multicols}{2}
	\begin{enumerate}[label=\alph*)]
		\item A bola permanece viva e um corredor pode avan\c{c}ar a seu pr\'oprio risco quando: 
		
		\begin{enumerate}[label=\roman*.]
			\item O receptor derruba o terceiro \gls{strike} e o batedor-corredor \'e tocado legalmente com a bola enquanto est\'a fora da base, ou pisa a primeira base 
			depois que a bola lan\c{c}ada a essa base \'e pega legalmente por um defensor. 
			\item Um defensor pega legalmente uma bola batida \gls{fly} antes que ela toque o solo, algum objeto ou uma pessoa, exceto um jogador da defensiva. 
			\item Ap\'os acertar uma batida \gls{fair}, um corredor \'e tocado enquanto est\'a fora da base, ou um batedor-corredor \'e declarado \gls{out} pela bola lan\c{c}ada antes de chegar \`a primeira base. 
			\item Em vez de ir \`a primeira base, entra na \'area de sua equipe 
			\begin{enumerate}[label=\arabic*)]
				\item ap\'os acertar uma batida \gls{fair}; 
				\item ap\'os ter obtido uma base por \glspl{ball} (\gls{base on balls}); 
				\item quando tem que avan\c{c}ar legalmente \`a primeira base. 
			\end{enumerate}
			\item \'E declarado um \gls{homeplate}. 
			\item Ap\'os acertar uma batida \gls{fair}, toca somente a parte da base dupla que est\'a em territ\'orio \gls{fair}, na sua primeira tentativa de alcan\c{c}ar a primeira base, e ocorre uma jogada nessa base. Isso \'e tratado da mesma forma que uma omiss\~ao de base. Se a equipe na defensiva apelar, o batedor-corredor ser\'a declarado \gls{out} (Jogada de Apela\c{c}\~ao). A equipe na defensiva perde a oportunidade de apelar se, depois que o batedor-corredor ultrapassa a base, n\~ao se manifesta sobre a omiss\~ao de base antes que ele retorne \`a por\c{c}\~ao \gls{fair} da primeira base. 
			\item Desvia mais de um (1) metro [tr\^es (3) p\'es] do caminho da base para evitar ser tocado com a bola na (s) m\~ao (s) de um defensor. 
			\item Quando qualquer pessoa, exceto outro corredor, presta ajuda f\'isica a um corredor numa bola \gls{fly} (o batedor-corredor \'e declarado \gls{out} se essa bola \'e 
			pega). 
		\end{enumerate}
		\item A bola \'e declarada morta, o corredor tem de retornar \`a \'ultima base tocada legalmente no momento do arremesso, mas n\~ao precisa tocar as bases intermedi\'arias quando o batedor-corredor: 
		\begin{enumerate}[label=\roman*.]
			\item N\~ao usa um capacete aprovado quando ordenado a faz\^e-lo pelo \'arbitro. 
			\item Corre fora da faixa de um (1) metro [tr\^es (3) p\'es] e, na opini\~ao do \'arbitro, 
			\begin{enumerate}[label=\arabic*)]
				\item estorva o defensor que est\'a pegando a bola lan\c{c}ada \`a primeira base; 
				ou 
				\item  interfere numa bola lan\c{c}ada e impede que um defensor execute uma jogada na primeira base. Uma bola lan\c{c}ada que atinge um batedor-	corredor n\~ao caracteriza, necessariamente, uma Interfer\^encia. 
			\end{enumerate}	
			\item Interfere na jogada de um defensor que est\'a tentando pegar uma bola batida. Um batedor-corredor pode correr fora da faixa de um metro para se esquivar de um defensor que est\'a tentando pegar a bola batida. 
			
			\item Interfere na jogada de um defensor que est\'a tentando lan\c{c}ar uma bola. 
			
			\item Interfere, intencionalmente, numa bola lan\c{c}ada. 
			
			\item Interfere numa bola batida \gls{fair} (fora do \gls{batter's box}) antes de chegar \`a primeira base. 
			
			\item Interfere na jogada de um terceiro \gls{strike} n\~ao agarrado. 
			
			\item Ap\'os bater a bola, atira seu \gls{bat} de tal forma que pode interferir na jogada de um defensor que tem chance de fazer um \gls{out}. 
			
			\item O Batedor Prevenido estorva um jogador da defensiva que est\'a tentando pegar uma bola \gls{fly}, ou interfere numa bola \gls{fly} que um defensor est\'a 
			tentando pegar. 
			
			\item Um membro da equipe na ofensiva, que n\~ao seja um batedor, batedor-
			corredor, corredor ou Batedor Prevenido, interfere na a\c{c}\~ao de um defensor que est\'a tentando pegar uma bola \gls{foul fly} (bola \gls{fly} que est\'a em territ\'orio \gls{foul}), ou num \gls{foul fly} que um defensor est\'a tentando pegar. 
			Se, na opini\~ao do \'arbitro, a Interfer\^encia \'e cometida com clara inten\c{c}\~ao de evitar uma jogada dupla, o corredor que est\'a mais perto do \gls{homeplate} no momento em que a falta \'e cometida deve tamb\'em ser declarado \gls{out}. 
			\item Interfere, intencionalmente, numa jogada no \gls{homeplate}, para evitar um \gls{out} evidente nessa base; se, na opini\~ao do \'arbitro, a Interfer\^encia foi cometida com o prop\'osito de atrapalhar uma jogada no \gls{homeplate}, o corredor tamb\'em \'e declarado \gls{out}. 
			\item D\'a um passo para tr\'as na dire\c{c}\~ao do \gls{homeplate}, para evitar ou retardar um toque de um defensor. 
			\item Numa situa\c{c}\~ao de Jogada For\c{c}ada, toca somente a por\c{c}\~ao 
			
			\gls{fair} da base dupla e colide com um defensor que, usando tamb\'em a por\c{c}\~ao
			
			\gls{fair} da base, est\'a prestes a pegar uma bola lan\c{c}ada. 

			\item Com menos de dois \glspl{out} e um corredor na primeira base, um defensor derruba, intencionalmente, uma bola \gls{fair fly} (incluindo um \gls{line drive} ou um \gls{fly} resultante de \gls{bunt}), que poderia ser pega por um defensor do campo interno com um esfor\c{c}o normal, depois de t\^e-la controlado com a m\~ao ou luva. 
			
			\item Um \gls{bunt} executado depois do segundo \gls{strike} resulta em \gls{foulball}, exceto se um corredor atrapalha um defensor que est\'a tentando pegar uma bola \gls{fly} resultante de \gls{bunt} em territ\'orio \gls{foul}, ou interfere numa bola \gls{foul fly} que um defensor est\'a tentando pegar; nesse caso, o batedor-corredor volta a bater, com um \gls{strike} adicional pela bola \gls{foul} se a contagem de bolas era menos de dois \glspl{strike} quando ele bateu a bola. Se um \gls{fly} resultante de \gls{bunt} for pego, a bola continuar\'a viva e em jogo. 
		\end{enumerate}
	
		\item Um corredor tem de retornar \`a \'ultima base que, na opini\~ao do \'arbitro, foi tocada no momento da Interfer\^encia, e a bola torna-se morta quando: 
		
		\begin{enumerate}[label = \roman*.]
			\item Na opini\~ao do \'arbitro, o corredor precedente (aquele que est\'a imediatamente \`a frente) que ainda n\~ao est\'a declarado \gls{out} atrapalha, intencionalmente, um defensor que est\'a tentando 
			
			\begin{enumerate}[label = \arabic*)]
				\item pegar uma bola lan\c{c}ada; ou 
				\item lan\c{c}ar a bola para tentar completar a jogada. 
			\end{enumerate}		
			\item Uma pessoa, exceto um membro da equipe, entra no campo de jogo e 
			interfere 
			\begin{enumerate}[label = \arabic*)]
				\item na a\c{c}\~ao de um defensor que est\'a prestes a pegar uma bola \gls{fly}; ou 
				\item numa bola \gls{fly} que, na opini\~ao do \'arbitro, um jogador da defensiva \'e capaz de pegar. 
			\end{enumerate}	
		\end{enumerate}
		
	\end{enumerate}
\end{multicols}
\subsection*{EFEITO} 	
\begin{tabular}{p{.08\columnwidth}p{.15\columnwidth}|p{.65\columnwidth}}
	\multicolumn{2}{c|}{Regra} & Efeito \\\hline\hline 
	
	5.5.2 (a) (v) & \'E declarado um \gls{homeplate}.& A bola est\'a viva e um corredor pode avan\c{c}ar correndo o risco de a bola ser pega, ou retocar a base e avan\c{c}ar ap\'os a bola ser tocada, como em qualquer bola \gls{fly}. Se um \gls{homeplate} declarado se torna um \gls{foulball}, ele \'e tratado da mesma forma que qualquer \gls{foulball}. 
	
	Se num \gls{homeplate} declarado a bola cai ao ch\~ao sem ter contato com um defensor e salta para o territ\'orio \gls{foul}, antes de passar a primeira ou terceira base, \'e um \gls{foulball}. Se num \gls{homeplate} declarado a bola cai ao ch\~ao sem ter contato com um defensor, fora da linha de base, e salta para o territ\'orio \gls{fair}, antes de passar a primeira ou terceira base, \'e um \gls{homeplate}. 
	\\\hline 	
	5.5.2 (b) (ii) a (xi) & Batedor-corredor comete Interfer\^encia. 
	&
	EXCE\c{C}\~AO: Se ocorrer uma jogada sobre um corredor antes da Interfer\^encia e 
	\begin{enumerate}[label = \arabic*)]
		\item ele for declarado \gls{out}, o resultado dessa jogada ser\'a mantido; 
		\item ele n\~ao for declarado \gls{out}, o resultado dessa jogada ser\'a mantido, a menos que a Interfer\^encia cometida pelo batedor-corredor ocasione o terceiro \gls{out}; outros corredores sobre os quais n\~ao tenha havido jogada t\^em de retornar \`a base tocada legalmente no momento do arremesso. 
	\end{enumerate}
	\\\hline
	Regra 5.5.2 (c) (ii) & Corredor precedente (aquele que est\'a imediatamente \`a frente) comete Interfer\^encia. & A bola torna-se morta e o corredor tamb\'em \'e declarado \gls{out}. 	\\\hline 
\end{tabular}


\section{BASE DUPLA} 
\begin{multicols}{2}
	Devem ser aplicadas as seguintes regras quando \'e usada a base dupla: 
	
	\begin{enumerate}[label=\alph*)]
		\item O batedor-corredor est\'a sujeito ao seguinte. 
		\begin{enumerate}[label=\roman*.]
			\item Uma bola batida que atinge a por\c{c}\~ao \gls{fair} \'e declarada \gls{fair}, e aquela que atinge somente a por\c{c}\~ao \gls{foul} \'e declarada \gls{foul}. 
			\item Um jogador da defensiva tem de usar somente a por\c{c}\~ao \gls{fair} da base dupla, exceto em jogada com bola viva feita do territ\'orio \gls{foul} do lado da primeira base; nesse caso, o batedor-corredor e o jogador da defensiva podem usar tanto a por\c{c}\~ao \gls{foul} como a por\c{c}\~ao \gls{fair} da base dupla. Quando o jogador da defensiva usa a por\c{c}\~ao \gls{foul} da base dupla, o batedor-corredor pode correr em territ\'orio \gls{fair}, e se for atingido por um lan\c{c}amento feito do territ\'orio \gls{foul} do lado da primeira base, isso n\~ao caracteriza uma Interfer\^encia. Se for apontada uma Interfer\^encia intencional, o batedor-corredor deve ser declarado \gls{out}. A faixa de um metro (tr\^es p\'es) duplica-se em lan\c{c}amentos feitos do territ\'orio \gls{foul} do lado da primeira base. 
			\item Numa jogada na primeira base em que o batedor-corredor que tenta alcan\c{c}ar a primeira base atrav\'es de uma batida, ou de um terceiro \gls{strike} n\~ao agarrado, toca somente a por\c{c}\~ao \gls{fair} da base dupla. Se a equipe na defensiva apelar antes de o batedor-corredor retornar \`a por\c{c}\~ao \gls{fair}, ele ser\'a declarado \gls{out}. Isso \'e tratado da mesma forma que uma omiss\~ao de base; a equipe na defensiva pode apelar. 
			\item Depois de ultrapassar a base correndo, o batedor-corredor tem de retornar \`a por\c{c}\~ao \gls{fair}. 
			\item Quando, em uma bola batida para o campo externo n\~ao est\'a havendo jogada na base dupla, o batedor-corredor pode tocar tanto a por\c{c}\~ao \gls{foul} como a por\c{c}\~ao \gls{fair} da base. 
		\end{enumerate}
		\item Devem ser aplicadas as seguintes regras a um corredor: 
		\begin{enumerate}[label=\roman*)]
			\item Depois de ultrapassar a base correndo, tem de retornar \`a por\c{c}\~ao \gls{fair}. 
			\item Quando vai retocar a base numa bola \gls{fly}, tem de usar a por\c{c}\~ao \gls{fair}. 
			\item Numa jogada em que se tenta surpreender o corredor fora da base, esse corredor tem de retornar \`a por\c{c}\~ao \gls{fair}. 
			
			\item Se o corredor ficar sobre a por\c{c}\~ao \gls{foul} somente ap\'os retornar \`a por\c{c}\~ao \gls{fair}, ele ser\'a considerado como se n\~ao estivesse em contato com a base, e poder\'a ser declarado \gls{out} se: 
			\begin{enumerate}[label=\arabic*)]
				\item for tocado com a bola; ou 
				\item ficar sobre a por\c{c}\~ao \gls{foul} da base enquanto o arremessador est\'a de posse da bola dentro do c\'irculo do arremessador. 
			\end{enumerate}	
		\end{enumerate}
	\end{enumerate}
\end{multicols}

\section{USO DE LUVA ILEGAL}
\begin{multicols}{2}
	Quando um defensor faz uma jogada sobre um batedor-corredor ou corredor enquanto est\'a usando uma luva ilegal, o t\'ecnico da equipe prejudicada tem a op\c{c}\~ao de: 
	
	\begin{enumerate}[label=\alph*)]
		\item Aceitar o resultado da jogada. 
		
		\item No caso de um batedor-corredor, mandar o jogador bater novamente, assumindo a contagem de \gls{ball} e \gls{strike} anterior ao arremesso; os outros corredores t\^em de retornar \`as bases que estavam ocupando legalmente no momento do arremesso. 
		\item No caso de um corredor, ter toda a jogada anulada; nesse caso, os corredores t\^em de retornar \`as bases que estavam ocupando legalmente no momento da jogada. Se a jogada for o resultado da a\c{c}\~ao de um batedor que termina a sua vez de bater, esse jogador dever\'a bater novamente, assumindo a contagem de \gls{ball} e \gls{strike} que tinha antes de completar o seu turno, e os corredores t\^em de retornar \`as bases que estavam ocupando no momento do arremesso. Um arremesso efetuado pelo arremessador n\~ao \'e considerado uma jogada. 
	\end{enumerate}
\end{multicols}

\section{REMO\c{C}\~AO DO CAPACETE}
\begin{multicols}{2}
	\begin{enumerate}[label=\alph*)]
		\item Quando a bola est\'a viva, um batedor, batedor-corredor ou corredor ser\'a declarado \gls{out} quando usa um capacete incorretamente, de prop\'osito, ou o remove deliberadamente, durante uma jogada, exceto num \gls{homerun} para fora do campo. A declara\c{c}\~ao de \gls{out} de um batedor-corredor ou corredor por ter removido o capacete, deliberadamente, n\~ao cancela qualquer situa\c{c}\~ao de Jogada For\c{c}ada, entretanto, se um capacete usado por um batedor, batedor-corredor ou corredor se soltar acidentalmente de seu lugar apropriado, n\~ao haver\'a penalidade.
		
		\item Quando a bola est\'a morta, um corredor tem de retornar \`a \'ultima base tocada no momento do contato: 
		\begin{enumerate}[label=\roman*.]
			\item Se uma bola lan\c{c}ada ou batida tem contato com o capacete removido propositalmente, ou um defensor tem contato com o capacete removido propositalmente, enquanto est\'a tentando fazer uma jogada. 
			\item Quando uma bola batida ou la\c{c}ada tem contato com o capacete que se soltara acidentalmente, e esse contato interfere na jogada que est\'a sendo executada; ou, quando um jogador da defensiva tem contato com um capacete que est\'a sobre o solo, e esse contato impede que ele fa\c{c}a uma jogada, o batedor-corredor ou corredor que estava usando esse capacete \'e declarado \gls{out}, mesmo que ele tenha anotado ponto. O ponto \'e anulado. 
		\end{enumerate}
	\end{enumerate}
\end{multicols}

\section{TOCAR AS BASES EM ORDEM LEGAL} 
\begin{multicols}{2}
	
	\begin{enumerate}[label=\alph*)]
		\item O batedor-corredor e todos outros corredores t\^em de tocar as bases em ordem legal (isto \'e, primeira, segunda e terceira base e \gls{homeplate}). 
		
		EXCE\c{C}\~AO: Se um corredor for obstru\'ido numa base e n\~ao conseguir tocar essa base, ou for colocado na segunda base de acordo com a Regra de Desempate (\Gls{tie-breaker rule}). 
		
		\item Um corredor que est\'a retornando a uma base enquanto a bola est\'a viva, correndo o risco de ser declarado \gls{out}, tem de retornar 
		
		\begin{enumerate}[label=\roman*.]
			\item \`a base que deixara antes de uma bola \gls{fly} ser pega no ar ou tocada por um defensor; ou 
			\item \`a base que omitira; e tem de tocar as bases em ordem inversa. 
		\end{enumerate}
		\item Quando um corredor est\'a retornando a uma base enquanto a bola est\'a morta, ele n\~ao precisa tocar as bases intermedi\'arias, a menos que tenha omitido uma base; nesse caso, a equipe na defensiva pode apelar se ele n\~ao retocar a base omitida. 
		\item Quando um corredor ou batedor-corredor adquire o direito a uma base, tocando-a antes de ser declarado \gls{out}, ele tem o direito de ocupar essa base at\'e que tenha tocado legalmente a base seguinte, em ordem, ou at\'e que seja for\c{c}ado a desocup\'a-la para um corredor subsequente. A bola est\'a em jogo e os corredores podem avan\c{c}ar, correndo o risco de serem declarado \gls{out}s. 
		
		\item Quando um corredor desloca uma base de sua posi\c{c}\~ao correta, nem ele, nem o(s) corredor(es) que o segue (m) na mesma s\'erie de jogadas s\~ao obrigados a acompanhar uma base que est\'a exageradamente fora de posi\c{c}\~ao. A bola est\'a em jogo e os corredores podem avan\c{c}ar ou retornar, correndo o risco de serem declarado \gls{out}s. 
		
		\item Dois corredores n\~ao podem ocupar a mesma base simultaneamente. O corredor que ocupava legalmente a base primeiro \'e autorizado a nela permanecer, a menos que seja for\c{c}ado a avan\c{c}ar. O outro corredor poder\'a ser declarado \gls{out} se for tocado com a bola. 
		
		\item O fato de um corredor precedente ser declarado \gls{out} por ter omitido uma base, ou por ter deixado uma base antecipadamente numa bola \gls{fly} pega no ar, n\~ao afeta a situa\c{c}\~ao de um corredor subsequente que toca as bases em ordem correta. Se a falta cometida pelo corredor –n\~ao tocar uma base em ordem normal ou deixar uma base antecipadamente numa bola \gls{fly} pega no ar -- causar o terceiro \gls{out} do \gls{inning}, nenhum corredor subsequente poder\'a anotar ponto. 
		
		\item Nenhum corredor poder\'a retornar para tocar uma base que omitira ou deixara ilegalmente, depois que um corredor subsequente tiver anotado ponto, ou depois que ele tiver deixado o campo de jogo. 
		
		\item As bases deixadas antecipadamente numa bola \gls{fly} pega no ar t\^em de ser retocadas antes de avan\c{c}ar \`as bases concedidas. 
		
		\item As bases concedidas t\^em de ser tocadas em ordem legal. 
	\end{enumerate}
	
\end{multicols}

\subsection*{EFEITO} 	
\begin{tabular}{p{.08\columnwidth}p{.15\columnwidth}|p{.65\columnwidth}}
	\multicolumn{2}{c|}{Regra} & Efeito \\\hline\hline 
	5.9 (g) a (i) & Tocar as bases. & O corredor ser\'a declarado \gls{out} se a defensiva apelar legalmente por ele ter omitido uma base ou deixado uma base antes de a bola ser tocada numa bola \gls{fly} pega no ar. 
\end{tabular}



\section{CORREDORES}
\begin{multicols}{2}	
	
	\subsection{CORREDORES PODEM AVAN\c{C}AR CORRENDO O RISCO DE SEREM DECLARADO \gls{out}S ENQUANTO A BOLA EST\'A VIVA} 
	
	\begin{enumerate}[label=\alph*)]
		\item Quando a bola deixa a m\~ao do arremessador em seu arremesso. 
		\item Numa bola lan\c{c}ada ou numa bola batida \gls{fair} que n\~ao \'e bloqueada. 
		
		\item Numa bola lan\c{c}ada que atinge um \'arbitro ou um jogador da ofensiva. 
		
		\item Quando uma bola \gls{fly} pega legalmente tem o primeiro contato com um defensor. 
		
		\item Quando uma bola batida \gls{fair} 
		
		\begin{enumerate}[label=\roman*.]
			\item  atinge um \'arbitro ou corredor depois de passar um defensor, exceto o arremessador, e desde que nenhum outro defensor tenha uma chance de fazer 
			um \gls{out}; 
			
			\item  \'e tocada por um defensor, incluindo o arremessador; 
			
			\item  atinge um fot\'ografo, encarregado da manuten\c{c}\~ao do campo, policial etc. designados para o jogo; a bola permanece viva. 
		\end{enumerate}	
		\item  Quando uma bola viva fica alojada no uniforme ou equipamento de um jogador da defensiva. 
		
		\item  Quando, a qualquer momento, deixa de tocar uma base para a qual est\'a autorizado a avan\c{c}ar, antes de tentar seguir para a base seguinte. 
		
		\item  Quando, ap\'os ultrapassar a primeira base correndo, faz uma tentativa de ir para a segunda base. 
		
		\item  Quando, ap\'os deslocar uma base, tenta ir para a base seguinte. 
		\item  Quando, um arremesso ilegal n\~ao batido -- e que \'e tamb\'em um \gls{wild pitch} ou \Gls{passedball}– tenta avan\c{c}ar al\'em de uma base que lhe \'e concedida (pelo arremesso ilega
		\item  Quando avan\c{c}a al\'em da base que lhe \'e concedida numa jogada em que: 
		\begin{enumerate}[label=\roman*.]
			\item 	um defensor toca, intencionalmente, uma bola lan\c{c}ada, com um equipamento removido do lugar onde normalmente \'e usado; 
			\item um defensor toca, intencionalmente, uma bola batida \gls{fair}, com um equipamento removido do lugar onde normalmente \'e usado. 
		\end{enumerate}
		
		
		\item  Quando avan\c{c}a al\'em da base at\'e a qual est\'a protegido ou que lhe \'e concedida por ter sido obstru\'ido. 
		
		\item  Quando avan\c{c}a al\'em da base para a qual \'e for\c{c}ado a ir por causa de uma base por \glspl{ball} concedida ao batedor.
	\end{enumerate}
%\end{multicols}

	\subsection*{EFEITO} 	
\begin{tabular}{p{.08\columnwidth}p{.40\columnwidth}|p{.40\columnwidth}}
	\multicolumn{2}{c|}{Regra} & Efeito \\\hline\hline 
	5.10.1 (h \& i) & N\~ao toca uma base ou continua avan\c{c}ando \`a base seguinte. 
	
	& O corredor ser\'a declarado \gls{out} se a defensiva apelar legalmente. 
\end{tabular}


%\begin{multicols}{2}	
	\subsection{BASES CONCEDIDAS A CORREDOR (ES) POR OBSTRU\c{C}\~AO }
	
	Quando ocorre uma Obstru\c{c}\~ao, inclusive num \gls{run-down play}: 
	
	\begin{enumerate}[label=\alph*)]
		\item Deve ser sinalizada uma Bola Morta Demorada; a bola permanece viva at\'e que a jogada seja conclu\'ida. 
		
		\item  Ao corredor obstru\'ido, e a cada um dos corredores afetados pela Obstru\c{c}\~ao, ser\'a concedida a base (ou bases) que, na opini\~ao do \'arbitro, teriam alcan\c{c}ado se n\~ao tivesse ocorrido a obstru\c{c}\~ao. Se o \'arbitro achar que h\'a justificativa, um jogador da defensiva que faz uma simula\c{c}\~ao de toque (\gls{fake tag}) poder\'a ser expulso do jogo. 
		
		\item  Se o corredor obstru\'ido for declarado \gls{out} antes de chegar \`a base que teria alcan\c{c}ado se n\~ao tivesse corrido a Obstru\c{c}\~ao, ser\'a declarada uma bola morta. Ao corredor obstru\'ido, e a cada corredor afetado pela Obstru\c{c}\~ao, ser\'a (\~ao) concedida (s) a (s) base (s) que, na opini\~ao do \'arbitro, teriam alcan\c{c}ado se a obstru\c{c}\~ao n\~ao tivesse acontecido. 
		
		\item  Um corredor obstru\'ido nunca pode ser declarado \gls{out} no espa\c{c}o -- entre duas bases -- onde ocorreu a Obstru\c{c}\~ao, a menos que: 
		\begin{enumerate}[label=\roman*.]
			\item Ele cometa um ato de Interfer\^encia depois que a Obstru\c{c}\~ao \'e apontada, ou sofra uma apela\c{c}\~ao legal por 
			
			\begin{enumerate}[label=\arabic*)]
				\item ter omitido uma base, mas desde que n\~ao tenha sido obstru\'ido nessa base e impedido de toc\'a-la; 
				\item por ter deixado uma base antes de uma bola \gls{fly} ter o primeiro contato com um defensor; ou 
				\item por ter passado a base que teria alcan\c{c}ado se n\~ao tivesse ocorrido a Obstru\c{c}\~ao; o corredor obstru\'ido pode ser declarado \gls{out} e a bola permanece viva. 
			\end{enumerate}
			\item Se o corredor obstru\'ido obt\'em com seguran\c{c}a a base que lhe teria sido concedida, na opini\~ao do \'arbitro, e h\'a uma jogada subsequente sobre outro corredor, ele n\~ao estar\'a mais protegido no espa\c{c}o -- entre as bases -- onde fora obstru\'ido, e poder\'a ser declarado \gls{out}. A bola permanece viva. Os corredores obstru\'idos s\~ao ainda obrigados a tocar todas as bases em ordem correta; e se n\~ao o fizerem, poder\~ao ser declarados declarado \gls{out}s em uma apela\c{c}\~ao correta da equipe na defensiva. EXCE\c{C}\~AO: Se um corredor for obstru\'ido numa base e n\~ao conseguir tocar essa base. 
		\end{enumerate}
	\end{enumerate}
	
	\subsection{CORREDORES S\~AO DECLARADO \glspl{out}} 
	
	\begin{enumerate}[label=\alph*)]
		\item Um corredor \'e declarado \gls{out} e a bola permanece viva quando: 
		\begin{enumerate}[label=\roman*.]
			\item Enquanto corre para qualquer base em ordem normal ou inversa, desvia mais de um (1) metro [tr\^es (3) p\'es] do caminho da base, para evitar ser tocado. 
			\item Enquanto a bola est\'a em jogo, e ele n\~ao est\'a em contato com uma base, \'e tocado por um defensor. 
			\item Numa Jogada For\c{c}ada, e antes de o corredor ter contato com a base para a qual \'e for\c{c}ado a avan\c{c}ar, um defensor, enquanto controla a bola em sua (s) m\~ao (s), toca a base, ou toca a bola na base, ou toca o corredor. Se um corredor obrigado deixar a base, ap\'os tocar a base seguinte retorna, por alguma raz\~ao, \`a base que estava ocupando, a situa\c{c}\~ao de Jogada For\c{c}ada \'e restabelecida. 
			\item N\~ao retorna para tocar a base que estava ocupando anteriormente, ou que omitira, e a equipe na defensiva apela legalmente. 
			\item Qualquer pessoa, exceto outro corredor, presta-lhe ajuda f\'isica enquanto a bola est\'a em jogo. Quando a bola se torna morta ap\'os um \gls{homerun}, uma bola \gls{foul} n\~ao pega ou uma concess\~ao de bases, a bola permanece morta. 
			\item Ultrapassa fisicamente um corredor precedente, antes que esse corredor tenha sido declarado \gls{out}. A bola permanece viva. O corredor n\~ao ser\'a declarado \gls{out} se a bola se tornar um \gls{foulball} ou um \gls{foul fly} n\~ao pego no ar, ou se um corredor ultrapassar um corredor precedente numa jogada em que a bola fica morta. A bola continuar\'a morta. 
			\item Deixa a sua base para avan\c{c}ar a outra base antes que uma bola \gls{fly} pega no ar tenha tocado um defensor. 
			\item Deixa de tocar a (s) base (s) intermedi\'aria (s) em ordem normal ou invera, a menos que seja obstru\'ido no momento em que est\'a tentando toc\'a-la (s). 
			\item O batedor-corredor que se tornara um corredor, tocando a primeira base, tenta correr para segunda base, ap\'os ultrapassar a primeira base, e \'e tocado enquanto est\'a fora da base. 
			\item Ap\'os passar correndo ou deslizando pelo \gls{homeplate}, sem pis\'a-lo, n\~ao tenta retornar para reparar a falha, e um defensor, com a bola na (s) m\~ao (s) e enquanto mant\'em contato com o \gls{plate}, apela ao \'arbitro por uma decis\~ao. 
			\item Abandona uma base e entra na \'area de sua equipe, ou deixa o campo de jogo, enquanto a bola est\'a viva. 
			\item Em qualquer bola \gls{fly}, se posiciona atr\'as da base e n\~ao fica em contato com ela, para iniciar a corrida \`a base seguinte tomando impulso a partir dessa posi\c{c}\~ao, ou seja, para fazer um \gls{running start}. 
			\item Quando corredores mudam de posi\c{c}\~oes nas bases. 
		\end{enumerate}
		\item  Um corredor \'e declarado \gls{out} e a bola torna-se morta quando: 
		\begin{enumerate}[label=\roman*.]
			\item N\~ao obedece \`a ordem do \'arbitro exigindo o uso de um capacete aprovado para batedores. 
			\item N\~ao fica em contato com a base para a qual foi autorizado a ir at\'e que a bola arremessada legalmente deixe a m\~ao do arremessador. \'E declarado um \textsl{Arremesso Nulo} e os outros corredores t\^em de retornar \`a \'ultima base que estavam ocupando no momento do arremesso. 
			\item Est\'a afastado de sua base legalmente ap\'os um arremesso, ou em raz\~ao de o batedor ter completado a sua vez de bater, e enquanto o arremessador, com a bola na m\~ao, est\'a dentro do C\'irculo do Arremessador, n\~ao retorna imediatamente \`a sua base, ou n\~ao tenta avan\c{c}ar \`a base seguinte. 
			
			Uma vez que o corredor retorna a uma base por qualquer raz\~ao, ele ser\'a declarado \gls{out} se deixar essa base. Um corredor n\~ao ser\'a declarado \gls{out} se: 
			
			\begin{enumerate}[label=\arabic*)]
				\item for feita uma jogada sobre ele ou outro corredor (uma simula\c{c}\~ao de toque \'e considerada uma jogada); 
				\item o arremessador n\~ao est\'a mais dentro do C\'irculo do Arremessador, com a bola na m\~ao; ou 
				\item o arremessador efetua um arremesso ao batedor. 
				
				Uma base por \glspl{ball}, ou um terceiro \gls{strike} n\~ao agarrado em que o corredor \'e autorizado a correr, \'e tratado da mesma forma que uma bola batida. O batedor-corredor pode continuar avan\c{c}ando ap\'os ultrapassar a primeira base, e \'e autorizado a correr em dire\c{c}\~ao \`a segunda base, desde que ele n\~ao pare na primeira base. Se ele parar ap\'os ultrapassar a primeira base fazendo uma curva, ter\'a de retornar \`a base, ou continuar avan\c{c}ando \`a segunda base, imediatamente. 
			\end{enumerate}
			\item O batedor-corredor \'e declarado \gls{out} por ter interferido numa jogada no \gls{homeplate} para tentar evitar uma elimina\c{c}\~ao evidente de um corredor que est\'a avan\c{c}ando para \gls{home}. O corredor que est\'a avan\c{c}ando \'e declarado \gls{out} e os outros corredores t\^em de retornar \`a \'ultima base que estavam ocupando no momento do arremesso.
		\end{enumerate}
		\item Um corredor \'e declarado \gls{out}, a bola torna-se morta, e os outros corredores t\^em de retornar \`a \'ultima base que estavam ocupando legalmente no momento em que ocorreu a Interfer\^encia; a bola ficou bloqueada (\gls{blocked ball}); ou um \gls{out} foi declarado (a menos que tenham sido for\c{c}ados a avan\c{c}ar porque o batedor se tornou um batedor-corredor), quando: 
		\begin{enumerate}[label=\roman*.]
			\item \'E atingido por uma bola batida \gls{fair} n\~ao tocada, em territ\'orio \gls{fair}, enquanto est\'a fora da base, e na opini\~ao do \'arbitro algum defensor tinha oportunidade de fazer uma jogada para fazer um \gls{out}. 
			\item Chuta, intencionalmente, uma bola que um defensor n\~ao tenha conseguido defender. 
			\item Interfere na a\c{c}\~ao de um defensor que est\'a tentando pegar uma bola batida \gls{fair}, sem levar em considera\c{c}\~ao se ela foi tocada antes pelo defensor ou por outro defensor, incluindo o arremessador, ou estorva um defensor que est\'a efetuando um lan\c{c}amento, ou interfere, intencionalmente, numa bola lan\c{c}ada. 
			\item Interfere na a\c{c}\~ao de um defensor que est\'a tentando pegar uma bola batida \gls{foul fly}, ou num \gls{foul fly} que um defensor est\'a tentando pegar. Se essa Interfer\^encia, na opini\~ao do \'arbitro, \'e uma evidente tentativa de evitar um \gls{double play} (jogada dupla), o corredor subsequente imediato deve tamb\'em ser declarado \gls{out}. O batedor-corredor volta a bater, com um \gls{strike} adicional pela bola \gls{foul}, se a contagem de arremessos anterior \`a batida era menos de dois \glspl{strike}. Se essa Interfer\^encia causar o terceiro \gls{out}, o batedor-corredor voltar\'a a bater como o primeiro batedor no pr\'oximo \gls{inning}, com a contagem original de \gls{ball} e \gls{strike} cancelada. 
			
			\item Depois de um corredor, batedor ou batedor-corredor ter sido declarado \gls{out}, ou ap\'os um corredor ter anotado ponto, tal corredor, batedor ou 		batedor-corredor interfere na a\c{c}\~ao de um defensor que tem oportunidade de fazer uma jogada sobre outro corredor. O corredor que est\'a mais perto do \gls{homeplate} no momento da Interfer\^encia deve ser declarado \gls{out}. Se um corredor continuar correndo ap\'os ter sido declarado \gls{out} e atrair um lan\c{c}amento, tal ato ser\'a tratado como uma forma de Interfer\^encia. 
			\item  Um ou mais membros da equipe na ofensiva param em, ou se re\'unem ao redor de, uma base para a qual um corredor est\'a avan\c{c}ando, e assim confundem os defensores e contribuem para dificultar a execu\c{c}\~ao de uma jogada. \'E considerado membro de uma equipe o \gls{bat boy} (pessoa encarregada de recolher os \glspl{bat} ao \gls{bench}) ou qualquer outra pessoa autorizada a permanecer no \gls{bench} da equipe. 
			
			\item O \gls{coach} da terceira base corre na dire\c{c}\~ao do \gls{homeplate}, sobre a/perto da linha de base, enquanto um defensor est\'a tentando fazer uma jogada sobre uma bola batida ou lan\c{c}ada e, assim, atrai um lan\c{c}amento ao \gls{homeplate}. O corredor que est\'a mais perto do \gls{homeplate} deve ser declarado \gls{out}. 
			\item  Um \gls{coach}, ou qualquer membro da equipe na ofensiva que n\~ao seja um batedor, batedor-corredor, Batedor Prevenido ou corredor, interfere, intencionalmente, numa bola lan\c{c}ada enquanto est\'a no \gls{coachsbox}, ou estorva a equipe na defensiva, que tem oportunidade de fazer uma jogada sobre um corredor ou batedor-corredor. O corredor que est\'a mais perto do \gls{homeplate} no momento da Interfer\^encia deve ser declarado \gls{out}. 
			\item  Movimentando-se em p\'e, colide, intencionalmente, com um jogador da defensiva que, com a bola na (s) m\~ao (s), est\'a preparado para toc\'a-lo. Se a m\'a inten\c{c}\~ao for flagrante, o infrator ser\'a expulso. 
			\item  Corre as bases em ordem inversa, ou fica afastado da linha de base enquanto n\~ao est\'a tentando avan\c{c}ar, para confundir os defensores ou ridicularizar o jogo. 
			\item  O Batedor Prevenido interfere na a\c{c}\~ao de um jogador da defensiva que est\'a tentando eliminar um corredor; o corredor que est\'a mais perto do \gls{homeplate} deve ser declarado \gls{out}. 
			\item  Um equipamento n\~ao oficial da equipe na ofensiva causa um \gls{blocked ball} (bola bloqueada) e provoca uma Interfer\^encia no momento em que est\'a ocorrendo uma jogada sobre o corredor. Se esse corredor tiver anotado ponto antes de ser declarado um \gls{blocked ball}, o corredor que est\'a mais perto do \gls{homeplate} deve ser declarado \gls{out}. 
		\end{enumerate}
		\item  Quando o \'arbitro de \gls{home} -- ou a sua roupa -- interfere na a\c{c}\~ao do receptor que est\'a tentando eliminar um corredor num \gls{stealing} (roubo de base), ou numa tentativa de \gls{pickoff play} (jogada em que o arremessador tenta segurar o corredor na base, ou eliminar o corredor que est\'a fora da base). Se, num \Gls{passedball} (bola defens\'avel que passa para tr\'as do receptor) ou \gls{wild pitch} (arremesso descontrolado), a bola lan\c{c}ada pelo receptor atinge o \'arbitro, n\~ao \'e uma Interfer\^encia do \'Arbitro, e a bola permanece viva. 
	\end{enumerate}
	
\end{multicols}
\subsection*{EFEITOS} 

\resizebox{.95\textwidth}{!}{
	\begin{tabular}{p{.08\columnwidth}p{.15\columnwidth}|p{.70\columnwidth}}
		\multicolumn{2}{c|}{Regra} & Efeito \\\hline\hline 
		
		5.10.3 (a) (vii a x) & Deixa a base antecipadamente numa bola \gls{fly}, omite uma base ou tenta chegar \`a segunda base ou omite o \gls{homeplate}.& O corredor n\~ao ser\'a declarado \gls{out}, a menos que a equipe na defensiva fa\c{c}a uma apela\c{c}\~ao legal.\\\hline
		\multicolumn{3}{p{\textwidth}}{
			EXCE\c{C}\~AO: Um corredor que tenha deixado uma base antecipadamente numa bola \gls{fly} pega no ar, ou tenha omitido uma base, pode tentar retornar a essa base enquanto a bola est\'a morta.} \\\hline
		
		5.10.3 (a) (xiii) & Corredores mudam de posi\c{c}\~oes nas bases. 
		
		& Esta \'e uma jogada de apela\c{c}\~ao. Quando a apela\c{c}\~ao \'e feita corretamente, cada corredor que tiver mudado de posi\c{c}\~oes nas bases, se for descoberto, ser\'a declarado \gls{out}, e o \Gls{coach} Principal ser\'a expulso por conduta antidesportiva. A ordem das elimina\c{c}\~oes ser\'a determinada pela posi\c{c}\~ao dos corredores imediatamente depois da mudan\c{c}a. O corredor que, ap\'os mudar de posi\c{c}\~oes, estiver mais perto do \gls{homeplate} ser\'a declarado \gls{out} primeiro. O pr\'oximo corredor que tiver chegado mais perto do \gls{homeplate} ap\'os mudar de posi\c{c}\~oes nas bases ser\'a o segundo a ser declarado \gls{out} etc. A apela\c{c}\~ao pode ser feita a qualquer momento antes que todos os corredores que mudaram de posi\c{c}\~oes estejam no \Gls{dugout}, ou o \gls{inning} tenha terminado. Se um dos corredores que mudaram de bases estiver numa base, ele e todos os corredores que tiverem mudado de bases ser\~ao declarados \gls{out}s, mesmo que tenham pisado o \gls{homeplate}, e o (s) ponto (s) anotado (s) por esse (s) corredor (es) incorreto (s) ser\'a (\~ao) anulado (s). \\\hline
		
		5.10.3 (c) (i), (c) (ii) e (c) (iii) & &Se essa Interfer\^encia, na opini\~ao do \'arbitro, \'e uma evidente tentativa de evitar uma jogada dupla, o corredor subsequente imediato tamb\'em deve ser declarado \gls{out}. 
		\\\hline
		
		5.10.3 (d) & Interfer\^encia do \'arbitro. 
		
		& Deve ser sinalizada uma Bola Morta Demorada; a bola permanece viva at\'e a conclus\~ao da jogada. 
		\begin{enumerate}[label=\roman*.]
			\item Se o corredor sobre o qual est\'a ocorrendo a jogada \'e declarado \gls{out}, a elimina\c{c}\~ao \'e mantida e a bola continua viva. 
			
			\item Se \'e declarado \gls{safe}, a bola torna-se morta e todos os corredores t\^em de retornar \`a \'ultima base que estavam ocupando no momento do lan\c{c}amento. 
		\end{enumerate}
		\\\hline
\end{tabular}}


\begin{multicols}{2}
	\subsection{O CORREDOR N\~AO \'E DECLARADO \gls{out}} 
	
	\begin{enumerate}[label=\alph*)]
		\item Quando corre atr\'as ou \`a frente do defensor, e fora do caminho da base, a fim de evitar interferir na a\c{c}\~ao de um defensor que est\'a tentando pegar a bola batida no caminho da base. 
		
		\item  Quando n\~ao corre em linha reta para a base, desde que o defensor que est\'a na linha reta n\~ao esteja de posse da bola. 
		
		\item  Quando mais de um defensor tenta pegar uma bola batida, e ele tem contato com aquele que, na opini\~ao do \'arbitro, n\~ao tinha o direito de pegar a bola. 
		
		\item  Quando, enquanto est\'a fora da base, \'e atingido por uma bola batida \gls{fair} n\~ao tocada, sobre a qual, na opini\~ao do \'arbitro, nenhum defensor teria conseguido fazer uma jogada para fazer um \gls{out}. 
		
		\item  Quando \'e atingido por uma bola batida \gls{fair} n\~ao tocada, em territ\'orio \gls{foul}, sobre a qual, na opini\~ao do \'arbitro, nenhum defensor teria conseguido fazer uma jogada para fazer um \gls{out}.
		\item Quando \'e atingido por uma bola batida \gls{fair} ap\'os esta ter tocado, ou ter sido tocada por qualquer defensor, incluindo o arremessador, e ele n\~ao podia evitar o contato com a bola. 
		
		\item  Quando \'e atingido por uma bola batida \gls{fair} n\~ao tocada enquanto est\'a em contato com sua base, a menos que interfira, intencionalmente, no curso da bola, ou na a\c{c}\~ao de um defensor que est\'a fazendo uma jogada. A bola torna-se morta ou permanece 
		viva, dependendo da posi\c{c}\~ao do defensor que est\'a mais perto da base. 
		
		\item  Quando \'e tocado enquanto est\'a fora da base: 
		\begin{enumerate}[label = \arabic*)]
			\item com uma bola que n\~ao est\'a firmemente segura por um jogador da defensiva; ou 
			
			\item com a m\~ao, enquanto segura a bola com a luva, ou com a luva, enquanto a bola est\'a na outra m\~ao. 
		\end{enumerate}	
		\item  Quando, numa Jogada de Apela\c{c}\~ao, a equipe na defensiva n\~ao solicita a decis\~ao do \'arbitro antes que seja efetuado o arremesso seguinte, legal ou ilegal, ou antes que\textbf{ todos os jogadores da defensiva tenham deixado o territ\'orio \gls{fair}, a caminho do \gls{bench} ou \Gls{dugout}}. 
		
		\item  Quando um batedor-corredor se torna um corredor ap\'os tocar a primeira base, ultrapassa-a e depois retorna diretamente \`a base. 
		
		\item  Quando n\~ao lhe \'e concedido tempo suficiente para retornar a uma base. Ele n\~ao ser\'a declarado \gls{out} por estar fora da base antes de o arremessador soltar a bola, e poder\'a avan\c{c}ar como se tivesse deixado a base legalmente. 
		
		\item  Quando tiver iniciado um avan\c{c}o legalmente. Ele n\~ao pode ser interceptado pelo arremessador que est\'a recebendo a bola enquanto est\'a sobre o \gls{pitcher's plate}, nem por aquele que est\'a tocando o \gls{pitcher's plate} enquanto est\'a segurando a bola. 
		
		\item  Quando permanece na sua base e tenta avan\c{c}ar \`a base seguinte depois que uma bola \gls{fly} tem contato com um defensor. 
		
		\item  Quando desliza para uma base e a desloca de sua posi\c{c}\~ao correta. Considera-se que a base tenha seguido o corredor. Um corredor que chega a salvo (\gls{safe}) a uma base n\~ao deve ser declarado \gls{out} por n\~ao estar em contato com a base deslocada. Ele pode retornar a essa base, sem o risco de ser declarado \gls{out}, quando ela tiver sido recolocada. Um corredor corre o risco de ser declarado \gls{out} se tentar avan\c{c}ar al\'em da base deslocada antes que ela esteja outra vez na posi\c{c}\~ao correta. 
		
		\item  Quando um \gls{coach} interfere, intencionalmente, numa bola lan\c{c}ada ou batida enquanto est\'a dentro do \gls{coachsbox}.
		
		\item Quando a bola tem contato com um equipamento n\~ao oficial da equipe na ofensiva e n\~ao \'e evidente que poderia ocorrer jogadas. A bola torna-se morta e todos os corredores t\^em de retornar \`a \'ultima base que estavam ocupando no momento em que a bola foi declarada morta, mas no retorno n\~ao precisam tocar as bases intermedi\'arias. 
		
	\end{enumerate}
\end{multicols}

\section{CONCESS\~AO DE BASES (EXCETO POR OBSTRU\c{C}\~AO)}

\begin{multicols}{2}

EFEITO -- Regra ou Ocorr\^encia. 


\begin{enumerate}[label=\alph*)]
	\item Concess\~ao de Uma Base: 
	\begin{enumerate}[label=\roman*.]
		\item O batedor-corredor adquire o direito de ocupar a primeira base, desde que avance e toque a base, e todos os outros corredores podem avan\c{c}ar uma base, se for\c{c}ados, a partir da base que ocupavam no momento do arremesso, nas seguintes circunst\^ancias: 
		\begin{enumerate}[label=\arabic*)]
			\item  Quando o \'arbitro de \gls{home} declara o quarto \gls{ball}; a bola \'e viva. 
			\item Quando o arremessador deixa o batedor ‘andar', intencionalmente; a bola torna-se morta. 
			\item Quando o batedor \'e obstru\'ido, e a equipe na ofensiva opta pelo direito de ele ocupar a primeira base; a bola torna-se morta. 
			\item Quando uma bola batida tem contato com um \'arbitro ou corredor antes de passar um defensor, excluindo o arremessador; a bola torna-se morta. 
			\item Quando o batedor \'e atingido por um arremesso; a bola torna-se morta. 
		\end{enumerate}
		
		\item Um corredor \'e autorizado a avan\c{c}ar uma base nas seguintes circunst\^ancias; a bola torna-se morta, exceto nos 6 casos abaixo: 
		\begin{enumerate}[label=\arabic*)]
			\item  Quando o arremessador faz um arremesso ilegal, e esse arremesso n\~ao \'e batido; ou se \'e batido, o t\'ecnico da equipe na ofensiva opta pelo direito de o corredor avan\c{c}ar uma base ao inv\'es de aceitar o resultado da jogada; a bola torna-se morta. 
			\item   Quando a bola arremessada sai do campo de jogo ou fica alojada no \gls{backstop} (barreira situada atr\'as do \gls{homeplate}); a concess\~ao \'e a partir da base que o corredor ocupava no momento do arremesso. 
			\item   Quando um defensor leva, involuntariamente, uma bola para fora do campo de jogo; a concess\~ao \'e a partir da \'ultima base tocada no momento em que o defensor deixou o campo de jogo. Um defensor que, de posse de uma bola viva, entra no \Gls{dugout} ou na \'area da equipe para tocar um corredor \'e considerado como se a tivesse levado \`a \'area de bola morta, involuntariamente. 
			\item   Quando um jogador perde a posse da bola durante uma jogada, e essa bola vai para \'area de bola morta, a concess\~ao \'e a partir da \'ultima base tocada no momento em que a bola entrou na \'area de bola morta. 
			\item   Quando um equipamento da defensiva causa uma Bola Bloqueada, a concess\~ao \'e a partir da \'ultima base tocada no momento do arremesso quando isso ocorre com uma bola arremessada. 
			\item   Quando um equipamento que est\'a fora do lugar apropriado tem contato com uma bola arremessada. 
			
			Se uma bola arremessada escapa do receptor, e este a recupera com equipamento fora do lugar apropriado quando nenhum corredor est\'a avan\c{c}ando, nenhuma poss\'ivel jogada \'e evidente, ou n\~ao ocasiona qualquer vantagem, n\~ao deve ser concedida uma base a nenhum corredor; a bola permanece viva, e o batedor pode avan\c{c}ar \`a primeira base somente quando obt\'em base por \glspl{ball} (\gls{base on balls}), ou quando \'e aplicada a Regra do Terceiro \gls{strike}; ele pode avan\c{c}ar al\'em da primeira base a seu pr\'oprio risco. 
		\end{enumerate}
	\end{enumerate}
	\item  Concess\~ao de duas bases: 
	\begin{enumerate}[label=\roman* -]
		\item O batedor-corredor e o (s) corredor (es) s\~ao autorizados a avan\c{c}ar duas bases a partir da base que ocupavam no momento do arremesso nas seguintes circunst\^ancias, e a bola torna-se morta: 
		\begin{enumerate}[label=\arabic*)]
			\item Quando uma bola batida \gls{fair} vai para fora de um campo de jogo a uma dist\^ancia menor do que as dimens\~oes de um campo oficial. 
			\item Quando uma bola batida \gls{fair fly} (\gls{fly} que est\'a em territ\'orio \gls{fair}) toca a luva ou o corpo do defensor e passa sobre a cerca em territ\'orio \gls{foul}. 
			\item Quando uma bola batida \gls{fair fly} toca a cerca, desvia ap\'os ter contato com um defensor e passa por cima da cerca. 
			\item Quando uma bola batida \gls{fair} \'e tocada por um defensor que est\'a em territ\'orio de bola morta e, na opini\~ao do \'arbitro, a bola n\~ao teria passado sobre a cerca em territ\'orio \gls{fair}. 
			\item Quando uma bola batida \gls{fair} pula sobre uma cerca, ou rola por baixo ou atrav\'es de uma cerca, ou vai para fora da linha que delimita o campo de jogo. 
			\item Quando uma bola batida \gls{fair} \'e desviada 
			
			
			\begin{enumerate}[label=\alph*)]
				\item por um jogador da defensiva ou \'arbitro, ou 
				
				\item  por um corredor, depois de ter passado um defensor, exceto arremessador, e desde que nenhum outro defensor tenha tido uma chance de fazer um \gls{out} e a bola tenha ficado fora de jogo em territ\'orio \gls{foul}. 
			\end{enumerate}
			\item Quando uma bola batida \gls{fair} tem contato com um defensor que est\'a em territ\'orio de bola morta e, na opini\~ao do \'arbitro, a bola n\~ao teria passado sobre a cerca em territ\'orio \gls{fair}. 
		\end{enumerate}
		%VERIFICAR ESTRUTURA DE NUMERACAO DOS ITENS
		\item Quando a bola lan\c{c}ada sai do campo de jogo ou \'e bloqueada, a concess\~ao \'e a partir da base que o corredor ocupava no momento em que a bola deixou a m\~ao do defensor. Se dois corredores est\~ao entre as mesmas duas bases, a concess\~ao \'e baseada na posi\c{c}\~ao do corredor precedente. Se um corredor toca a base seguinte e retorna para sua base original, essa base original \'e considerada a \textsl{\'ultima base tocada} para os prop\'ositos de concess\~ao de bases em raz\~ao de um lan\c{c}amento descontrolado (\gls{overthrow}). 
		
		\item Quando um equipamento da defensiva causa uma Bola Bloqueada, a concess\~ao \'e: 
		
		\begin{enumerate}[label=\arabic*)]
			\item a partir da \'ultima base tocada no momento do lan\c{c}amento; 
			\item a partir da \'ultima base tocada no momento do arremesso, numa bola batida \gls{fair}. 
		\end{enumerate}	
		\item Quando uma bola lan\c{c}ada tem contato com um equipamento fora do lugar apropriado, deve ser declarada uma Bola Morta Demorada. 
		
		\item Um corredor \'e autorizado a avan\c{c}ar somente duas bases, e a bola torna-se morta quando, na opini\~ao do \'arbitro, um defensor leva, chuta, empurra ou lan\c{c}a uma bola viva, da \'area de jogo para \'area de bola morta, intencionalmente. A concess\~ao \'e a partir do momento em que a bola foi chutada, empurrada ou lan\c{c}ada, ou do momento em que a bola foi levada para \'area de bola morta. 
	\end{enumerate}
	
	\item  Concess\~ao de Tr\^es Bases: 
	
	O batedor-corredor e os corredores s\~ao autorizados a avan\c{c}ar tr\^es (3) bases, e deve ser declarada uma Bola Morta Demorada, quando um equipamento fora do lugar apropriado tem contato com uma bola batida \gls{fair}. 
	
	\item  Concess\~ao de Quatro Bases: 
	
	O batedor-corredor e os corredores s\~ao autorizados a ir para \gls{home}, e a bola torna-se morta, nas seguintes circunst\^ancias: 
	
	\begin{enumerate}[label=\roman* -]
		\item Quando o \'arbitro declara um \gls{homerun}; 
		\item Quando uma bola \gls{fair} tem contato com um equipamento fora do lugar apropriado, e, na opini\~ao do \'arbitro, ela teria passado sobre a cerca do campo externo, em voo. 
	\end{enumerate}
	\item  Concess\~oes de acordo com a opini\~ao do \'arbitro: 
	
	O batedor-corredor e os corredores s\~ao autorizados a avan\c{c}ar \`a base que, na opini\~ao 
	do \'arbitro, teriam alcan\c{c}ado se n\~ao tivesse ocorrido a Interfer\^encia; a bola torna-se morta. 
	\begin{enumerate}[label=\roman*.]
		\item Quando uma pessoa, exceto um membro da equipe, interfere numa bola batida \gls{ground} ou numa bola lan\c{c}ada, ou estorva um defensor que est\'a preparado para pegar uma bola, incluindo bolas \gls{fly}. 
		\item Quando a bola fica alojada no equipamento ou roupa do \'arbitro ou na roupa de um jogador da ofensiva. 
	\end{enumerate}
	%	\end{enumerate}
\end{enumerate}	
\end{multicols}