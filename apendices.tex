\chapter{DESENHO DO CAMPO DE JOGO E \textit{DIAMOND}}
\minitoc% Creating an actual minitoc


\section{DIMENSÕES OFICIAIS DO CAMPO DE JOGO}

\section{DIMENSÕES OFICIAIS DO DESENHO DO \textit{DIAMOND}}

\section{DIMENSÕES OFICIAS DAS BASES}

\section{DIMENSÕES OFICIAIS DO \textit{BATTER'S BOX} E \textit{CATCHER'S BOX}}

\section{DIMENSÕES OFICIAIS DO \textit{HOME PLATE} E \textit{PITCHER'S PLATE}}

\section{TABELA DE REFERÊNCIA RÁPIDA \textit{BACKSTOP} E LINHAS LATERAIS (LINHA DE BOLA MORTA/CERCA LATERAL)}

\begin{description}

 \item[\textit{BACKSTOPS}] (barreira situada atrás da área do \gls{homeplate}) e as linhas/cercas laterais devem estar situadas a 7,62m (25 pés), no mínimo, e a 9,14m (30 pés), no máximo, atrás das linhas de \gls{foul}.
 A área entre as linhas de \gls{foul} e o \glspl{backstop}, e entre as linhas de \gls{foul} e as linhas/ cercas laterais, tem de estar  desobstruída.

\item [BASES]
 Distâncias:
 \begin{itemize}
 	\item  \gls{homeplate} até primeira/terceira base: 18,29m (60 pés) da parte de trás da placa até a parte de trás da base.
 	\item \gls{homeplate} até segunda base: 25,86m (84 pés 10 \textonequarter{}  polegadas) da parte de trás da placa até o meio da base.

 \end{itemize}

 As bases devem ser feitas de lona ou outro material apropriado, e devem estar firmemente fixadas em sua posição.

 Metade da Base Dupla (Primeira Base) é fixada em território \gls{fair}, e é parte do território \gls{fair}, e a outra metade (de cor bem diferente e contrastante), em território \gls{foul}, e é parte do território \gls{foul}.

\item [\textit{BATTER'S BOXES}]
 Um em cada parte do \gls{homeplate}. Devem medir 0,91m (3 pés) por 2,13m (7 pés). As linhas internas do \gls{battersbox} devem estar a 15,20cm (6 polegadas)
 do \gls{homeplate}. A linha dianteira do \gls{box} deve estar a 1,22m (4 pés) na frente de uma linha traçada através do centro do \gls{homeplate}. As linhas são consideradas dentro do \gls{battersbox}.


\item [\textit{CATCHER'S BOX}]
 Medida: 3,05m (10 pés) de comprimento dos cantos externos traseiros dos \glspl{batter's box} e deve ter 2,57m (8 pés 5 polegadas) de largura. As linhas são consideradas dentro do \gls{catcher's box}.

\item [\textit{COACHES' BOXES}]
 É aquela área atrás de uma linha de 4,57m (15 pés) traçada fora do \gls{diamond} (campo). A linha é paralela à linha da primeira/terceira base, e está a 3,66m (12 pés) dessas linhas, estendidas das bases em direção ao \gls{homeplate}.

 TABELA DE DISTÂNCIAS

\begin{center}
	\begin{tabular}{*{6}{|c|c}|}\hline
		\multicolumn{2}{|c|}{\multirow{2}{*}{CATEGORIA}} &
		\multicolumn{2}{c|}{"H. PLATE"-"P. PLATE"}&
		\multicolumn{2}{c|}{\parbox{40mm}{CERCAS DO CAMPO EXTERNO (mínimas)}}\\\cline{2-6}
		\multicolumn{2}{|c|}{}&m&pés&m&pés\\\hline
		Junior Fem. &( 16u) &12,19 & 40 & 67,06 & 220 \\\hline
		Junior Fem. &( 19u) &13,11 & 43 &67,06 &220 \\\hline
		Mulheres Adulto& &13,11 & 43 & 67,06  &220 \\\hline
		Junior Masc. &(16u)& 14,02 &46 & 76,20 &250  \\\hline
		Junior Masc. &(19u)& 14,02 &46 & 76,20  &250  \\\hline
		Homens Adulto& &14,02 &46 & 76,20  &250  \\\hline
	\end{tabular}
\end{center}


\item [\textit{HOME PLATE}] Deve ter cinco lados.

	A borda voltada para o arremessador deve ter 43,20cm (17polegadas) de largura.

	Os lados devem ser paralelos às linhas internas do \gls{battersbox} e devem ter 21,60cm (8 \textonehalf  polegadas) de comprimento.

	Os lados da ponta voltada ao receptor devem ter 30,50cm (12 polegadas) de comprimento.

\item [CAMPO INTERNO] É aquela parte do campo, sem grama, que forma um arco a 18,29m (60 pés) do centro da borda dianteira do \gls{pitcher's plate}.

\item [LINHAS] Devem ter 50mm a 100mm (2 a 4 polegadas) de largura.

\item [CÍRCULO DO BATEDOR PREVENIDO] É um círculo com 1,52m (5 pés), 0,76m (2 pés e 6 polegadas) de raio, localizado próximo ao fim da área do \gls{bench} ou \gls{dugout} dos jogadores mais perto do \gls{homeplate}.

\item [LINHA DE UM METRO] Linha traçada paralelamente à linha de base, e a um metro (3 pés) dessa linha, partindo de um ponto onde inicia a segunda metade da distância entre o \gls{homeplate} e a primeira base.

\item [CÍRCULO DO ARREMESSADOR] É um círculo de 4,88m (16 pés), com raio de 2,44m (8 pés), traçado do centro da borda dianteira do \gls{pitcher's plate}. As linhas são consideradas dentro do círculo.

\item [\textit{PITCHER'S PLATE}]
 É feito de borracha e tem 61cm (24 polegadas) de comprimento e 15,2cm (6 polegadas) de largura. A parte superior da placa tem de estar no mesmo nível do solo.

\item [ZONA DE ADVERTÊNCIA]
 Deve estar marcada a 3,66m (12 pés), no mínimo, e 4,57m (15 pés), no máximo, da cerca do campo externo e/ou das cercas laterais.

 A marcação deve ser feita com material (terra, cascalho) equivalente (mas diferente) ao da superfície do campo.

 O material tem que ser distinguível do material da superfície do campo externo, e deve chamar a atenção dos jogadores quando eles estão se aproximando da cerca.
\end{description}

\section{TRAÇANDO UM \textit{DIAMOND}}

Esta seção serve como um exemplo para traçar um campo (\gls{diamond}) com distância de 18,29m (60 pés) entre as bases e 14,02m (46 pés) entre o \gls{homeplate} e o \gls{pitcher's plate}.

\begin{enumerate}[label = (\arabic*)]


\item   Para determinar a posição do \gls{homeplate},
\begin{enumerate}
	\item trace uma linha na direção em que deseja situar o campo.
	\item fixe uma estaca no canto do \gls{homeplate} mais perto do receptor.
	\item amarre um cordão nessa estaca e dê nós ou marque a corda de outra forma %após medir
	os seguintes comprimentos:

\begin{center}
	\begin{tabular}{c|l}\hline
		m & pés polegadas\\\hline
		14,02 &46 '\\\hline
		18,29 &60 '\\\hline
		25,86 &84 ' e 10 \textonequarter{} " \\\hline
		36,58 &120 '\\\hline
	\end{tabular}
\end{center}


\end{enumerate}

\item   Coloque o cordão (sem esticar) ao longo da linha diretora e coloque uma estaca onde marca 14,02m (46 pés). Esta será a linha de frente no meio do \gls{pitcher's plate}. Ao longo da mesma linha, fixe uma estaca onde marca 25,86m (84 pés e 10 \textonequarter{} polegadas). Este será o centro da segunda base.

\item  Coloque o ponto onde marca 36,58m (120 pés) no local determinado para o centro da segunda base e, pegando o cordão no ponto onde marca 18,29m (60 pés), ande para a direita da linha diretora até esticá-lo e pregue uma estaca no ponto onde marca 18,29m (60 pés) -- este será o canto externo da primeira base, e o cordão, agora, formará a linha entre a primeira e a segunda bases.

\item  Segurando, outra vez, o cordão no ponto onde marca 18,29m (60 pés), atravesse o campo e, da mesma maneira, marque o canto externo da terceira base. O \gls{homeplate}, a primeira base e a terceira base estão inteiramente na parte interna do campo (\gls{diamond}).
\item  Para conferir as medidas do campo (\gls{diamond}), coloque a ponta da corda que marca o \gls{homeplate} na estaca da primeira base, e o ponto onde marca 36,58m (120 pés), na terceira base. O ponto onde marca 18,29m (60 pés) deve, agora, coincidir com os locais marcados para o \gls{homeplate} e a segunda base.
\item  Confira todas as distâncias com uma fita métrica metálica sempre que possível.
\end{enumerate}


\chapter{ESPECIFICAÇÕES DO \textit{BAT}}
\minitoc% Creating an actual minitoc

\section{\textit{BAT} OFICIAL}

\begin{enumerate}[label=(\arabic*)]
	\item  O \gls{bat} tem de ser feito com uma peça só, com várias peças juntadas definitivamente, ou com duas peças trocáveis.
	\item  Quando o \gls{bat} é projetado para ser feito com componentes trocáveis, tem de levar em conta o seguinte critério:

		\begin{enumerate}[label=\roman*.]
			\item os componentes acoplados devem ter um dispositivo de segurança especial para evitar que equipamento com combinações não aprovadas seja usado no campo; e
			\item os \glspl{bat} confeccionados com combinações de componentes têm de seguir os padrões estabelecidos como se fossem um \gls{bat} feito com uma peça só. Os  componentes têm de seguir os padrões estabelecidos como se fossem partes de um \gls{bat} feito com uma peça só.
		\end{enumerate}

	\item  Um \gls{bat} pode ser feito com um pedaço de madeira de lei (madeira dura), ou com um bloco de madeira composto de dois ou mais pedaços de madeira colados entre si com um adesivo, de tal forma que a direção das fibras de todas as peças seja paralela ao comprimento do \gls{bat}.
	\item  Um \gls{bat} pode ser de metal, bambu, plástico, grafite, carbono, magnésio, fibra de vidro, cerâmica, ou qualquer outro material composto aprovado pela WBSC-SD ou ISF \gls{ESC}.
	\item  Um \gls{bat} pode ser laminado, mas deve conter somente madeira ou adesivo, e ter um acabamento perfeito (quando pronto).
	\item  A parte mais grossa do \gls{bat} (do início da parte cônica até a ponta do \gls{bat}) deve ser redonda e lisa.
	\item  Não deve ter mais de 86,40cm (34 polegadas) de comprimento, nem pesar mais de 1077,00g (38 onças).
	\item  Não deve ter mais de 5,70cm (2 \textonequarter{} polegadas) de diâmetro em sua parte mais grossa. É permitida uma tolerância de 0,80mm (1/32 polegada) devido à dilatação que pode haver no material.
	\item  Um taco que tenha quaisquer rebites expostos, pinos, bordas ásperas ou afiadas ou qualquer prendedor externo que seja ou apresente um risco é um taco ilegal. Um taco que não seja de madeira deve estar livre de rebarbas e rachaduras.
	\item  Um taco ou bastão que não seja de madeira, não deve ter cabo de madeira.

	\item  Um \gls{bat} tem de ter uma empunhadura de segurança de cortiça, fita (fita plástica não lisa) ou material composto.

	A empunhadura de segurança não deve ter menos de 25,40cm (10 polegadas) de comprimento e não deve se estender mais de 38,10cm (15 polegadas) da extremidade do cabo.

	É permitido aplicar	resina, alcatrão de pinho ou substâncias em spray somente na empunhadura de segurança, para aumentar a sua eficiência.

	A fita aplicada a qualquer \gls{bat} tem	de ser em espiral contínua. Não precisa ser uma camada sólida de fita.

	Não deve exceder duas camadas.

	\item  Um taco que não seja de madeira e não seja feito de uma peça com a extremidade do cano fechada, deve ter uma borracha ou plástico de vinil ou outro material que seja aprovado pela Comissão de Padrões de Equipamentos WBSC, que esteja firmemente preso na extremidade maior do mesmo."

	\begin{enumerate}[label=\roman*.]
		\item A tampa colocada na extremidade aberta da parte grossa do \gls{bat} tem de estar firme e permanentemente lacrada, para que ela não possa ser removida por qualquer pessoa, exceto o fabricante, sem danificá-la ou  destruí-la.
		\item O \gls{bat} não deve causar ruídos. Um \gls{bat} que causa ruídos será considerado um \gls{bat} ilegal. \gls{bat} não deve ter sinais de adulteração. Um \gls{bat} que mostra sinais de adulteração será considerado um \gls{bat} Adulterado.
	\end{enumerate}

	\item  Um \gls{bat} tem que ter um dispositivo de segurança (saliência arredondada na extremidade do cabo) de, no mínimo, 0,60cm (\textonequarter{} de polegada) ressaltando, a um ângulo de 90 graus, do cabo, e não deve ter bordas afiadas.

	O dispositivo de segurança pode ser moldado, torneado, soldado e permanentemente fixo; pode ser coberto com fita.

	\item  Um \gls{bat} que tenha a informação \gls{bat} Oficial Aprovado" ilegível, devido ao desgaste pelo uso, pode ainda ser utilizado se todos os outros aspectos estiverem de acordo com as regras, e desde que isso possa ser constatado com razoável segurança.

	\item  O peso, a distribuição do peso, ou o comprimento do \gls{bat} têm de ser estabelecidos permanentemente por ocasião da fabricação, e não podem ser modificados de maneira alguma depois disso, excetuando-se algo diferente que esteja especificamente previsto nesta Regra, ou haja uma especificação aprovada pela WBSC-SD ou ISF \gls{ESC}.
\end{enumerate}

\section{\textit{BAT} PARA FAZER AQUECIMENTO}

É um \gls{bat} -- exceto um \gls{bat} oficial -- que tem de ser feito com uma peça só, e deve sujeitar-se aos requisitos exigidos aos dispositivos de segurança (empunhadura de segurança e saliência arredondada na extremidade do cabo) do \gls{bat} oficial.

Tem de estar marcado \textit{warm-up}, com letras de 3,20cm (1 \textonequarter{}  polegada), na extremidade do cilindro.

A extremidade do cilindro tem que ter mais de 5,70cm (2 \textonequarter{}  polegadas).

\chapter{PADRÕES DE BOLA}
\minitoc% Creating an actual minitoc

\section{BOLA OFICIAL}

\begin{enumerate}[label=(\arabic*)]
	\item   Tem que ser uma bola com formato regular, emendas lisas, pontos de costura não salientes ou com superfície plana.
	\item  Tem que ter um núcleo central feito tanto de fibra longa de paina de primeira qualidade, de uma mistura de cortiça e borracha, de uma mistura de poliuretano, como de outros materiais aprovados pela WBSC-SD \gls{ESC}.
	\item  Pode ser enrolada (manualmente ou a máquina) com fio trançado de boa qualidade e coberta com cola de látex ou borracha.
	\item  Tem que ter:
	\begin{itemize}
		\item 	uma cobertura costurada com fio encerado de algodão ou linho, colada à bola mediante aplicação de substância aderente na face inferior (da cobertura), ou
		\item uma cobertura moldada colada ao núcleo, ou
		\item uma cobertura integralmente moldada com o núcleo.
	\end{itemize}


	As peças moldadas devem ter uma reprodução autêntica da costura aprovada pela WBSC-SD \gls{ESC}.
	\item  Tem que ter uma cobertura da melhor qualidade, feita de couro de cavalo ou vaca curtido em cromo No 1, ou de material sintético ou outros materiais aprovados pela WBSC-SD \gls{ESC}.
\end{enumerate}

\section{DIMENSÕES E ESPECIFICAÇÕES}

\begin{enumerate}[label=(\arabic*)]
	\item  A bola de 30,50cm (12 polegadas), pronta, deve

	\begin{itemize}
		\item ter entre 30,20cm (11 7/8 polegadas) e 30,80cm (12 1/8 polegadas) de circunferência, e
		\item pesar entre 178,00g (6 \textonequarter{} onças) e 198,40g (7 onças).
	\end{itemize}


	O tipo "costura plana" deve ter, no mínimo, 88 pontos em cada cobertura, costurados pelo método de duas agulhas.
	\item  A bola pronta deve ter um coeficiente de restituição e um padrão de compressão, que serão determinados e instituídos pela WBSC-SD \gls{ESC}.
	\item  COR significa Coeficiente de Restituição de uma bola quando medido pelo método de teste para medir o Coeficiente de Restituição de bolas da ASTM (American Society for Testing and Materials).
	\item  A bola de 30,50cm (12 polegadas), com costura branca ou vermelha ou cobertura amarela com um COR de .47 ou menos, deve ser usada em jogos de Campeonato da WBSC-SD, nas seguintes categorias: Adultos (Masculino e Feminino), Júnior (Masculino e Feminino). As bolas devem ter a logomarca da WBSC-SD.
	\item  Em bolas usadas em Jogos de Campeonato da WBSC-SD, a força de carga exigida para comprimir a bola 0,64cm (0,25 polegadas) não precisa exceder 170,10kg (375 libras) quando tais bolas são testadas de acordo com o método de provas para medir compressão-deslocamento de bolas de softbol da ASTM, método esse aprovado pela WBSC-SD \gls{ESC}.
\end{enumerate}

 Abaixo, estão relacionados os padrões estabelecidos para cada bola:

 \begin{center}
 \begin{tabular}{ll}\hline
	Bola&  30,5 cm Diametro ( 12')\\\hline
	Cor da Bola&  Branca ou Amarela WBSC SD\\\hline
	Cor da Costura &  Branca ou Vermelho prespontado\\\hline
	Tamanho Minimo & 30,2 cm ( 11-7/8'')\\\hline
	Tamanho Maximo& 30,8 cm ( 12-1/8'')\\\hline
	Peso Minimo & 178.0 g. (6 \textonequarter{}  oz.)\\\hline
	Peso Maximo & 198.4 g. (7 oz.)\\\hline
\end{tabular}
 \end{center}



\chapter{ESPECIFICAÇÕES DA LUVA}
\minitoc% Creating an actual minitoc

ESPECIFICAÇÕES DAS DIMENSÕES:


	\begin{tabular}{llll}\hline
	\multirow{2}{*}{Posição} &
	\multirow{2}{*}{Descrição}&
	\multicolumn{2}{c}{medidas}\\
	&&cm & pol.\\\hline
	A& Largura da palma (parte superior)&20,30&8\\\hline
	B& Largura da palma (parte inferior)&21,60&8 \textonehalf \\\hline
	C& Abertura da parte superior do trançado&12,70&5 \\\hline
	D&  Abertura da parte inferior do trançado&11,50&4 \textonehalf \\\hline
	E&  Topo à base do trançado&18,40&7 \textonequarter{}  \\\hline
	F&  Costura da forquilha do primeiro dedo&19,00&7 \textonehalf \\\hline
	G&  Costura da forquilha do polegar&19,00&7 \textonehalf \\\hline
	H&  Costura da forquilha&44,50&17 \textonehalf \\\hline
	I&  Topo do polegar à borda inferior&23,50&9 \textonequarter{} \\\hline
	J&  Topo do primeiro dedo à borda inferior&35,60&14\\\hline
	K&   Topo do segundo dedo à borda inferior&33,70&13 \textonequarter{} \\\hline
	L&  Topo do terceiro dedo à borda inferior&31,10&12 \textonequarter{} \\\hline
	M& Topo do quarto dedo à borda inferior&27,90&11 \\\hline

\end{tabular}















\chapter{ÁRBITROS}
\minitoc% Creating an actual minitoc

\section{INFORMAÇÕES GERAIS PARA ÁRBITROS}

\begin{enumerate}[label=(\alph*)]
	\item  O árbitro oficial não deveria ser um membro de nenhuma das equipes.

 Exemplos: jogador, \gls{coach}, técnico, dirigente, anotador ou patrocinador.

 	\item O árbitro deve estar seguro quanto à data, ao horário e ao local do jogo, e deve chegar ao campo de jogo com antecedência de 20 a 30 minutos, iniciar o
jogo pontualmente e deixar o campo depois de encerrá-lo.

	\item O árbitro (masculino e feminino) tem de usar:

	\begin{enumerate}[label=(\arabic*)]
		\item Uma camisa azul-claro, com mangas longas ou curtas.
		\item  Meias azul-marinho escuro.
		\item  Calças azul-marinho escuro.
		\item  Boné azul-marinho escuro, com a marca WBSC (em letras brancas decoradas com contornos azuis) pregada na frente.
		\item  Bolsa para bolas azul-marinho escuro (somente árbitro de \gls{home}).
		\item  Jaqueta e/ou pulôver azul-marinho escuro.
		\item  Sapatos e cinto pretos.
		\item  Uma camiseta branca, que deve ser usada por baixo da camisa azul-claro.
	\end{enumerate}

\item Árbitros não devem usar joias expostas que possam oferecer risco.

 EXCETO braceletes e/ou colares com fins medicinais.

\item O árbitro de \gls{home}, na modalidade Arremesso Rápido, tem de usar uma máscara para rosto preta, com estofamento preto ou bege, um protetor de garganta preto, um protetor de tórax e caneleiras que protejam também os joelhos. (Pode ser usada uma máscara que já vem dotada de um protetor de garganta na parte inferior da armação.)
\item Os árbitros devem apresentar-se aos capitães, técnicos e anotadores.
\item  Os árbitros devem inspecionar as delimitações do campo de jogo, o equipamento etc. e esclarecer todas as regras de campo para ambas as equipes e seus \glspl{coach}.
\item  Cada árbitro tem o poder de tomar decisões sobre as infrações cometidas a qualquer momento durante o desenrolar da partida, ou enquanto ela está paralisada, até o jogo ser encerrado.

\item   Nenhum árbitro tem autoridade para desprezar ou questionar as decisões tomadas por outro árbitro dentro dos limites de seus respectivos deveres, conforme está especificado nestas regras.

\item  Um árbitro pode consultar seu companheiro a qualquer momento. Contudo, a decisão final deve ser do árbitro que, mesmo tendo autoridade exclusiva para decidir, recorreu à opinião de outro árbitro.
\item  Para definir suas respectivas obrigações, o árbitro que julga as bolas arremessadas (\gls{ball} ou \gls{strike}) será designado como o "Árbitro de Home", e o árbitro que julga as decisões nas bases, como o "Árbitro de Base".

\item  O árbitro de \gls{home} e o árbitro de base devem ter a mesma autoridade para:

	\begin{enumerate}[label=(\arabic*)]
		\item Declarar declarado \gls{out} um corredor que sai da base antecipadamente.
		\item Declarar \gls{time} para paralisar o jogo.
		\item  Remover, ou expulsar do jogo, um jogador, \gls{coach} ou técnico, por violação de regras.
		\item  Declarar todos os Arremessos Ilegais.
		\item  Determinar e declarar um \gls{infieldfly}. Quando parecer evidente que uma bola batida será um \gls{infieldfly}, o árbitro deverá declarar, imediatamente, \gls{infieldfly} SE FOR \gls{fair}, O BATEDOR É \gls{out}, para beneficiar os corredores.
	\end{enumerate}

\item  O árbitro deve declarar \gls{out} o batedor, batedor-corredor ou corredor, sem esperar por uma apelação para tal decisão, em todos os casos em que esse jogador é declarado \gls{out} de acordo com estas regras.
\item   A menos que haja uma apelação, o árbitro não deve

 declarar \gls{out} um jogador, ou
  penalizá-lo, por
  \begin{itemize}
  	\item não ter tocado uma base;
  	\item ter deixado uma base antecipadamente numa bola \gls{fly} pega no ar;
  	\item ter batido fora de ordem;
	\item ter entrado no jogo como substituto sem ser anunciado ao árbitro;
	\item ter reingressado ilegalmente;
	\item ter entrado no jogo como Jogador de Emergência, ou
	\item ter retornado ao jogo após ter sido removido de acordo com a Regra de Jogador de Emergência, sem comunicação ao árbitro;
	\item ter mudado de posições nas bases com outro corredor; ou
	\item ter tentado ir à segunda base depois de chegar à primeira base, conforme está estabelecido nestas regras.
  \end{itemize}

\item   Os árbitros não devem penalizar uma equipe por infração de uma regra quando a imposição da penalidade pode resultar em vantagem à equipe infratora.
\item  A inobservância, pelos árbitros, das instruções contidas no Anexo 5 não é motivo para protesto. Estas instruções são normas de procedimento para árbitros.
\end{enumerate}

\section{SINAIS}

\begin{enumerate}[label=(\alph*)]
	\item  Para indicar que a partida deve começar, ou ser retomada, o árbitro deve declarar \gls{play ball} e, ao mesmo tempo, sinalizar para o arremessador efetuar arremesso.
	\item Para indicar um \gls{strike}, o árbitro deve levantar a mão direita acima do ombro (dobrar o cotovelo, de modo que forme um ângulo de 90 graus) e, ao mesmo tempo, declarar \gls{strike} com voz clara e firme.
	\item Para indicar um \gls{ball}, não deve ser usado sinal algum com o braço.
	\item Para indicar a CONTAGEM de \glspl{ball} e \glspl{strike}, o árbitro deve declarar a quantidade de \glspl{ball} primeiro.
	\item Para indicar um \gls{foul}, o árbitro deve declarar \gls{foulball} e estender ambos os braços, verticalmente, acima da cabeça.
	\item Para indicar uma bola \gls{fair}, o árbitro deve estender um braço na direção do centro do campo, agitando-o para a frente e para trás, imitando um movimento de bombeamento.
	\item  Para indicar que um batedor ou corredor é \gls{out}, o árbitro deve levantar a mão direita, com o punho cerrado, acima do ombro direito.
	\item  Para indicar que um jogador é \gls{safe}, o árbitro deve estender ambos os braços, horizontalmente, para os lados do corpo, com as palmas das mãos viradas para o solo.

	\item Para indicar a paralisação da partida, o árbitro deve declarar \gls{time} e, ao mesmo tempo, estender ambos os braços acima da cabeça. Os outros árbitros devem confirmar a paralisação da partida, imediatamente, fazendo o mesmo gesto.

	\item  Para indicar uma BOLA MORTA DEMORADA (\gls{delayeddeadball}), o árbitro deve estender o braço esquerdo, horizontalmente, mantendo o punho cerrado.
	\item  Para indicar um \gls{trappedball}(bola bloqueada), o árbitro deve estender ambos os braços, horizontalmente, para os lados do corpo, com as palmas das mãos viradas para o solo.

	\item Para indicar que o batedor-corredor (ou corredor) tem direito a duas bases (\gls{ground rule double}), o árbitro deve estender a mão direita
	 acima da cabeça e, ao mesmo tempo, indicar com dois dedos o número de bases concedidas.
	\item Para indicar um \gls{homerun}, o árbitro deve estender a mão direita com o punho cerrado, acima da cabeça, e fazer um movimento circular em sentido horário.
	\item Para indicar um \gls{infieldfly}, o árbitro deve declarar \gls{infieldfly}. se for \gls{fair}, o batedor é \gls{out}. O árbitro deve estender um braço acima da cabeça.
	\item Para indicar que o arremessador não deve efetuar o arremesso (\gls{nottopitch}), o árbitro deve levantar uma mão com a palma da mão virada para o arremessador. Deve ser declarado "NO PITCH" (arremesso nulo) se o arremessador efetua o arremesso enquanto o árbitro está com sua mão na posição mencionada.
\end{enumerate}

\chapter{ANOTAÇÃO}
\minitoc% Creating an actual minitoc

\section{REGISTRO DE DADOS (\textit{BOX SCORE})}

		O nome de cada jogador e a posição, ou posições, a ser(em) ocupada(s) devem ser relacionadas na ordem em que ele bateu, ou teria batido, a menos que o jogador seja substituído legalmente, expulso ou removido do jogo, ou o jogo termine antes de sua vez de bater.

		Quaisquer dados estatísticos acumulados pelo Jogador de Emergência enquanto estava no jogo são creditados a esse jogador, mesmo que ele seja um substituto constante da lista que, eventualmente, não entre no jogo para substituir outro jogador.

		Quaisquer dados estatísticos acumulados por um Corredor Temporário devem ser creditados ao jogador por quem ele está correndo.

		\begin{enumerate}[label=\arabic*)]
			\item O Jogador Designado (JD) é opcional, mas, se vai utilizar um, seu nome tem de ser anunciado antes do início do jogo e estar relacionado no Formulário de Anotações, na ordem correta em que vai atuar como batedor. Serão relacionados dez nomes, e o décimo nome será o do "JOGADOR FLEX"{} por quem o JD está batendo.

			%% (a) Os dados referentes à atuação de cada jogador, na ofensiva e na defensiva, têm de estar tabulados.

			\item A primeira coluna deve indicar o número de vezes que cada jogador atua como batedor (\textit{at bat}), mas não deve ser imputado um \textit{at bat} contra o jogador quando esse batedor

				\begin{enumerate}[label=(\alph*)]
					\item Bate um \gls{fly} de sacrifício que ocasiona a marcação de um ponto.
					\item É autorizado a "andar" (\gls{walk}, \gls{baseonballs}).
					\item É autorizado a ir à primeira base por ter sofrido Obstrução.
					\item Executa um \gls{bunt} de sacrifício.
					\item É atingido por uma bola arremessada (\gls{hitbypitch}).
			\end{enumerate}

			\item A segunda coluna deve indicar a quantidade de pontos anotados por cada jogador.

			\item A terceira coluna deve indicar a quantidade de \glspl{basehit} (batidas indefensáveis) batidos por cada jogador. Uma batida indefensável é aquela que permite que o batedor chegue a salvo (\gls{safe}) a uma base.

				\begin{enumerate}[label=(\alph*)]
				\item Quando um batedor-corredor chega a salvo (\gls{safe}) à primeira base, ou a qualquer base subsequente, numa bola \gls{fair} que fica no campo, transpõe a cerca, ou atinge a cerca antes de ser tocada por um defensor.
				\item Quando um batedor-corredor chega a salvo (\gls{safe}) à primeira base numa bola \gls{fair} -- muito forte, ou muito lenta, ou que dá um salto incomum -- impossível de ser defendida por um defensor, com um esforço normal, a tempo de eliminar esse batedor-corredor.
				\item Quando uma bola \gls{fair} que não tenha tido contato com um defensor se torna "morta" por ter tocado o corpo ou a roupa de um corredor ou árbitro.
				\item Quando o defensor tenta, sem sucesso, eliminar um corredor precedente e, na opinião do anotador, o batedor-corredor não teria sido declarado \gls{out} na primeira base com uma defesa perfeita.
				\item Quando o batedor termina o jogo com uma batida indefensável que empurra os pontos necessários para colocar a sua equipe em vantagem no placar, a ele deve ser creditado somente um \gls{basehit} de tantas bases quantas tiver avançado o corredor que anotou o ponto da vitória, com a condição de que ele (o batedor) avance o mesmo número de bases.
			\end{enumerate}

			EXCEÇÃO: Quando o batedor termina o jogo com um \gls{homerun} batido para fora do campo, ele deve ser creditado com um \gls{homerun}, e todos os corredores, incluindo ele, devem ser autorizados a anotar ponto.

			\item A quarta coluna deve indicar a quantidade de adversários declarado \gls{out}s por cada jogador (\gls{putout}).
				\begin{enumerate}[label=(\alph*)]
					\item Um \gls{putout} é creditado a um defensor cada vez que ele:

					\begin{enumerate}[label=\arabic*)]
						\item Pega uma bola batida para o ar (\gls{fly}) ou uma batida em linha reta (\gls{line drive}).
						\item Pega uma bola lançada que elimina um batedor ou corredor.
						\item Toca um corredor com a bola quando esse corredor está fora da base na qual tem o direito de permanecer.
						\item Está mais perto da bola quando um corredor é declarado \gls{out} por ter sido atingido por uma bola \gls{fair} ou por ter estorvado um defensor.
						\item Está mais perto do jogador substituto não anunciado, que é declarado \gls{out} de acordo com a Regra 3.2.8.
						\item Está mais perto de um corredor que é declarado \gls{out} por correr fora do caminho da base.
					\end{enumerate}
					\item Um \gls{putout} é creditado ao receptor:

					\begin{enumerate}[label=\arabic*)]
						\item Quando é declarado um terceiro \gls{strike}.
						\item Quando o batedor não bate na ordem correta.
						\item Quando o batedor interfere na ação do receptor.
						\item Quando o batedor é declarado \gls{out} por ter batido ilegalmente.
						\item Quando o batedor é declarado \gls{out} em razão de uma tentativa de \gls{bunt} depois de dois \glspl{strike} ter resultado em \gls{foulball}.
						\item Quando o batedor é declarado \gls{out} por usar um \gls{bat} ilegal ou Adulterado.
						\item Quando o batedor é declarado \gls{out} por ter mudado de um \gls{battersbox} a outro.
					\end{enumerate}
				\end{enumerate}

			\item A quinta coluna deve indicar a quantidade de assistências ("assists") dadas nas eliminações por cada defensor. Deve ser creditada uma assistência

				\begin{enumerate}[label=(\alph*)]
					\item A cada jogador que maneja a bola em qualquer série de jogadas que resulte na eliminação do corredor. Deve ser atribuída somente uma assistência -- não mais -- a um jogador que maneja a bola em qualquer eliminação. Um jogador que tenha auxiliado numa Jogada de Perseguição (\gls{run-down play}) ou outra jogada do gênero pode ser creditado com um "assist" e um \gls{putout}.
					\item A cada jogador que maneja, ou lança, a bola de tal maneira que poderia ter contribuído na eliminação de um corredor se não ocorresse um erro subsequente de um companheiro de equipe.
					\item A cada jogador que, desviando uma bola batida, ajuda na eliminação de um corredor.
					\item A cada jogador que maneja a bola numa jogada que resulte na eliminação de um corredor, por Interferência, ou por correr fora da linha de base.
				\end{enumerate}
			\item A sexta coluna deve indicar a quantidade de erros cometidos por cada jogador.

			Erros são registrados nas seguintes situações:

				\begin{enumerate}[label=(\alph*)]
				\item A cada jogador que executa uma má jogada (\gls{misplay}) que prolongue o turno do batedor, ou a vida de um corredor que está ocupando alguma base.
				\item Ao defensor que deixa de tocar a base após receber a bola para eliminar o corredor, numa \gls{jogadaforcada} ou quando esse corredor é obrigado a retornar à base.
				\item Ao receptor, se um batedor é autorizado a ir à primeira base por Obstrução.
				\item Ao defensor que deixa de completar uma Jogada Dupla ("Double Play") por ter derrubado a bola.
				\item Ao defensor, se um corredor avança uma base em razão da falha desse defensor em parar ou tentar parar uma bola lançada corretamente a uma base, contanto que tenha havido motivo para o lançamento. Quando mais de um jogador poderia ter recebido o lançamento, o anotador tem de determinar a quem atribuir o erro.
			\end{enumerate}

		\end{enumerate}


\section{NÃO DEVEM SER REGISTRADOS \textit{basehitS} (BATIDAS INDEFENSÁVEIS)}

		 Não deve ser anotado um \gls{basehit} nos seguintes casos:

		\begin{enumerate}[label=(\alph*)]
			\item  Quando um corredor é declarado \gls{out} em \gls{jogadaforcada}, ou teria sido declarado \gls{out} em \gls{jogadaforcada} se um defensor não tivesse cometido erro.
			\item Quando um jogador que pega uma bola batida elimina um corredor precedente com um esforço normal.
			\item Quando um defensor tenta, mas não consegue eliminar um corredor precedente e, na opinião do anotador, o batedor-corredor poderia ter sido declarado \gls{out} na primeira base.
			\item Quando um batedor-corredor chega a salvo (\gls{safe}) à primeira base porque um corredor precedente é declarado \gls{out} por ter interferido numa bola batida ou na ação de um jogador da defensiva.

		 EXCEÇÃO: Se, na opinião do anotador, o batedor teria chegado a salvo (\gls{safe}) à primeira base se não tivesse ocorrido a Interferência, a ele deve ser creditado um \gls{safehit} (batida por meio da qual o batedor-corredor chega a salvo à primeira base).
		\end{enumerate}

	\section{\textit{FLY} DE SACRIFÍCIO}

		É anotado um \gls{fly} de sacrifício quando, com menos de dois \glspl{out},

			\begin{enumerate}[label=(\alph*)]
			\item  O batedor empurra um ponto com um \gls{fly} que é pego no ar; ou
			\item Um corredor anota ponto depois que um defensor do campo externo (ou um defensor do campo interno que tenha corrido para o campo externo) derruba um \gls{fly} ou \gls{line drive} após ter tocado a bola, e, na opinião do anotador, esse corredor poderia ter pisado o \gls{homeplate}, mesmo que o defensor a tivesse pegado no ar.
		\end{enumerate}

	\section{PONTOS EMPURRADOS (\textit{RUN BATTED IN})}

		São os pontos anotados por meio de:

			\begin{enumerate}[label=(\alph*)]
				\item  Um \gls{safehit}.
				\item Um \gls{bunt} de sacrifício ou um "slap hit" (SOMENTE AR), ou um \gls{fly} de sacrifício (AR e AL).
				\item Um \gls{foul fly} pego no ar.
				\item Um \gls{infieldputout} (eliminação feita no campo interno) ou \gls{fielder'schoice} (opção feita por um defensor na hora de executar uma jogada).
				\item Ocorrências que forçam um corredor a avançar para \gls{home}: por causa de Obstrução, porque o batedor é atingido por uma bola arremessada (\gls{hitbypitch}), ou em razão da concessão de uma base por \glspl{ball}.
				\item Um \gls{homerun} e todos os pontos anotados como decorrência desse \gls{homerun}.
			\end{enumerate}

	\section{ARREMESSADOR CREDITADO COM UMA VITÓRIA}

		 Um arremessador será creditado com uma vitória, nas seguintes situações:
			\begin{enumerate}[label=(\alph*)]
				\item  Quando, na condição de arremessador abridor, tiver arremessado pelo menos quatro \glspl{inning}, e sua equipe, que estava ganhando no momento em que foi substituído, permanecer liderando o placar pelo resto do jogo.
				\item Quando, numa partida encerrada depois de jogados cinco \glspl{inning}, o arremessador abridor tiver arremessado pelo menos três \glspl{inning}, e sua equipe tiver anotado mais pontos do que a outra no momento em que a partida é dada por terminada.
			\end{enumerate}

	\section{ARREMESSADOR DEBITADO COM UMA DERROTA}

		 Um arremessador será debitado com uma derrota, independentemente do número de \glspl{inning} que tenha arremessado, se for substituído quando sua equipe está perdendo e, depois disso, ela não consegue empatar ou tomar a dianteira no placar.

	\section{RESUMO DO JOGO}

		O resumo deve relacionar os seguintes itens, nesta ordem:
		\begin{enumerate}[label=(\alph*)]
			\item A quantidade de pontos por \gls{inning} e a contagem final.
			\item Os pontos empurrados (\gls{runbattedin}) e quem os empurrou.
			\item \glspl{hit} de duas bases (\glspl{two-base hit}) e quem os bateu.
			\item \glspl{hit} de três bases (\glspl{three-base hit}) e quem os bateu.
			\item \glspl{homerun} e quem os bateu.
			\item \glspl{fly} de sacrifício (\glspl{sacrificefly}) e quem os bateu.
			\item Jogadas Duplas (\glspl{doubleplay}) e os jogadores que delas participaram.
			\item Jogadas Triplas (\glspl{tripleplay}) e os jogadores que delas participaram.
			\item Quantidade de \glspl{walk} (\gls{baseonballs}) concedidos por cada arremessador.
			\item Quantidade de batedores declarado \gls{out}s por \gls{strike} (\glspl{strikeout}) por cada arremessador.
			\item Quantidade de \glspl{hit} e pontos permitidos por cada arremessador.
			\item O nome do arremessador vencedor.
			\item O nome do arremessador perdedor.
			\item O tempo de duração do jogo.
			\item Os nomes dos árbitros e anotadores.
			\item Bases roubadas (\glspl{stolenbase}) e por quem.
			\item \glspl{bunt} de sacrifício.
			\item Os nomes dos batedores atingidos por uma bola arremessada e dos arremessadores que os atingiram.
			\item A quantidade de \glspl{wild pitch} feitos por cada arremessador.
			\item A quantidade de \glspl{passedball} feitos por cada receptor.
		 \end{enumerate}

	\section{BASES ROUBADAS (SOMENTE AR)}

			Deve-se creditar uma base roubada (\gls{stolenbase}) a um corredor, sempre que ele avança uma base sem a ajuda de um \gls{hit}, um \gls{putout}, um erro, uma eliminação forçada, um \gls{fielder'schoice}, um \gls{passedball}, um \gls{wild pitch} ou um arremesso ilegal. Isso inclui um batedor-corredor que avança à segunda base numa base por \glspl{ball} concedida.

	\section{DADOS DE JOGOS CONFISCADOS (\textit{FORFEITED GAMES})}

 	Todos os dados de um jogo confiscado devem ser incluídos nas anotações oficiais, exceto aquele registro de arremessador ganhador/perdedor.

\chapter*{GLOSSÁRIO}

No Softball existem muitos termos que são utilizados pelos seus praticantes, árbitros e até anotadores que não são de conhecimento geral e podem causar um certo desconforto para os árbitros e espectadores que não estão familiarizados com eles. Com o intuito de esclarecer alguns destes termos, que podem ser até termos japoneses devido a influência nipônica e sua importância na prática deste esporte no Brasil, foi criado este glossário básico.

 2B -- Double: um batedor ganha um 2B quando a batida que ele deu proporcionou avanço de duas bases.

 3B -- Triple: um batedor ganha um 3B quando a batida que ele deu proporcionou avanço de três bases.

 Assists: termo utilizado para designar uma assistência de outro defensor para que o assistido consiga eliminar fisicamente um adversário no jogo.

 APP -- Appearance: é creditado uma appearance todas vezes que o pitcher entra em um determinado jopgo

Appeal Plays (Jogadas de Apelação): O time na defensiva tem direito a apelar de algumas jogadas que não tiveram a regra correta aplicada por uma árbitro no entendimento do técnico. Estas apelações servem para alertar o árbitro de infrações que poderiam ser permitidas sem serem apeladas. Nota. Não existem apelações para jogadas de decisão (Ball, strike, safe, out, foul, fair)

 AVG -- Batting Average: é um índice conseguido dividindo-se a quantidade de batidas conseguidas por um batedor pela quantidade de vezes que esteve batendo. É uma das mais comentadas estatísticas neste jogo. Expressa entre zero ( .000 ) e um ( 1.000).

 Ball: é a bola "ruim", que é arremessada fora da "zona de strike". A cada quatro bolas ruins lançadas pela arremessadora, o time adversário ganha o direito de avançar uma base.

 Boro: é o termo japonês para "Bola" ou \gls{ball}.

 Bor Baco: é o termo japonês para \gls{ballback} ou a informação dada pelo árbitro para recolherem as bolas, pois o jogo irá iniciar.

 Bat: taco, bastão.

 Bata: é o termo japonês para "Bastão" ou \gls{bat}.

 Batter's Box (Área do Batedor): um campo regular tem duas áreas de batedores. À esquerda e à direita da placa de home, de onde ele deve se posicionar para bater a bola.

 BB -- Base on Balls: é uma concessão de bases. Ao batedor é permitido avançar à primeira base mediante 4 arremessos inválidos cedidos pelo técnico do time na defensiva. Geralmente ocorre quando o batedor é um bom batedor e time está receoso de tomar vários pontos.

 Blocked Ball: bola desviada por algum material estranho espalhado no campo, bola rebatida, lançada ou arremessada que fica alojada na cerca.

 Bola Morta: diz-se bola morta quando por algum motivo ou necessidade o árbitro paralisa o jogo, as jogadas não podem ocorrer, e os jogadores de ataque devem permanecer nas suas bases. Uma bola rebatida fora do campo é bola morta; uma interferência, é bola morta.

 Bola Viva: quando a bola não está invalidada. As jogadas e eliminações podem ocorrer e os atacantes podem, por sua conta e risco, tentar roubar bases.

 C -- Chances: representa o número de oportunidades que o jogador teve com o intuito de eliminar um adversário. É utilizado em estatísticas de jogo.
 Catcher: é o receptor, jogador do time da defesa que se posiciona atrás do rebatedor e apanha as bolas arremessadas e não são rebatidas.

 Catcher's Interference (Interferência do Catcher): se o catcher ou qualquer outro defensor interferir com o batedor durante um arremesso, é concedida a primeira base ao batedor. A interferência pode ser, por exemplo, encostar a luva no bastão do batedor na hora da batida.

 CH -- Changeup: é um tipo de arremesso mais lento do softball e faz com que a bola caia bem perto do \gls{homeplate}. É um arremesso de efeito.

 CI -- Catcher's Interference: quando um catcher (ou outro defensor) interfere com o batedor em qualquer momento em que ele está tentando realizar uma
 rebatida. Neste caso, o batedor ganha a primeira base. (de novo?)

 Collision at Home Plate (Colisão no Home): o corredor não deve de maneira nenhuma desviar do seu caminho direto para o Homebase. Mas o Catcher deve,
 se não tiver de posse da bola, abrir espaço para a passagem do corredor. Caso haja um contato físico entre os dois, caracteriza esta colisão.

 CS -- Caught Stealing: número de vezes que um jogador é eliminado (por toque) ao tentar roubar uma base.

 CSB -- Caught Stealing (Pitcher/Catcher): número de vezes que um determinado jogador foi eliminado por tentar roubar bases.

 CU -- Curveball: bola curva. É um arremesso de efeito.

 Diamond: diamante ou campo de jogo. Campo interno.

 Deto Boro: é o termo japonês para "bola morta".

 Double Play: é o ato de fazer dois eliminados durante a mesma jogada contínua.

 As jogadas duplas são relativamente comuns, pois podem ocorrer sempre que houver pelo menos um corredor em base e menos de dois outs.


 E -- Errors: um defensor é creditado um erro quando, no julgamento do anotador oficial, ele falhou ao tentar converter uma eliminação que um defensor normal conseguiria.

 ERA -- Earned Run Average: é o numero de pontos concedidos pelo arremessador dividido pelo número de innings jogados (o que depende da categoria). Importante que os pontos considerados são os pontos que foram concedidos sem ajuda de algum erro ou bolas passadas (não agarradas pelo catcher).

 Fair Ball -- Bola Válida: é uma bola rebatida que autoriza o batedor a tentar alcançar a primeira base.

 FA -- Fastball (Bola rápida): é um arremesso direto e rápido.

 First-base Coach: é o técnico ou jogador que fica na área de técnico ao lado da primeira base, e geralmente sinaliza jogadas aos corredores e batedores para orientá-los.

 FLD\% -- Fielding Percentage: indica quão frequentemente um defensor ou até um time consegue realizar jogadas corretas ao receber bolas arremessadas para realizar eliminações. Geralmente tem fórmula que a descreve: número total de putouts e assistências feitas por um defensor dividido pelo numero total de chances (putouts assists e erros).




 Foul Ball (Bola Inválida): é uma bola rebatida que não autoriza o batedor a tentar alcançar a primeira base.

 \gls{foultip}: uma bola batida que vai direto do bastão para as mãos do catcher e é legalmente capturada.

 Furay : é o termo japonês para \gls{fly} ou "bola aérea".

 GDP -- Ground into Double Play: ocorre quando um defensor agarra uma bola rasteira e consegue efetuar jogadas eliminando dois ou mais jogadores nas bases.

 Globo: é o termo japonês para \gls{glove} (luva).

 glove: Luva (Globo)

 Gorô: é o termo japonês para \gls{ground} ball, ou bola rasteira, no chão.

 Ground Ball (Gorô): uma bola rebatida ao chão que vai rolando aos defensores.



 Hit (Batida): acontece quando o batedor consegue acertar a bola e esta cai em território válido, dentro do campo.

 Hito: é o termo japonês para \gls{hit}.

 Home Plate: formalmente designada nas regras como \gls{home}, é a base final que um jogador deve tocar para marcar ponto.

 Home run: jogada em que o rebatedor lança a bola para fora do limite do campo, acima da cerca de proteção, sem que esta toque no chão. Desta forma, o jogador é capaz de percorrer as quatro bases numa mesma corrida, marcando um ponto para a sua equipe

 HR -- Home Run- ocorre quando um batedor consegue acertar uma bola para fora de campo, ou correr as quatro bases sem ser eliminado.

 IP -- Illegal Pitch: ocorre quando o arremessador efetua alguma manobra proibida antes, durante ou após o arremesso. Existem várias possibilidades para um arremesso ilegal e o árbitro deve estar atento a regra.

% \gls{jogadaforcada}: é uma situação em que um corredor de base é compelido (ou forçado, obrigado) a desocupar sua base por conta de outro corredor que está chegando, e assim tentar avançar a salvo para a próxima base.

 Kuniguê: é aquela jogada que no terceiro strike (termo japonês isturaiku) o catcher deixa a bola escapar e o corredor corre para a primeira base. Nota: kuiniguê é comer e fugir (sem pagar a conta).

% Lana: é o \gls{runner}, é o corredor.

 Line Ball: uma bola rebatida em linha reta para dentro do terreno de jogo.

 Obstruction (Obstrução): é considerado o ato de um defensor que não está de posse da bola ou no processo de pegá-la impede o avanço de qualquer corredor.

 Passed Ball: bola defensável que passa para trás do receptor.

 PB -- Passed Ball (Catcher): é um termo estatístico que determina a quantidade de bolas lançadas ao catcher que ele não consegue segurar e como resultado deste erro um jogador consegue avançar uma base.

 Pickle: sanduíche -- é quando um corredor fica correndo entre bases para evitar ser tocado por um defensor que está de posse da bola | a rundown.

 Pickoff Play: jogada em que o arremessador tenta segurar o corredor na base, ou eliminar o corredor que está fora da base.

 Pitcher: arremessador, jogador da defesa que faz os lançamentos ao batedor.

 PO -- Put Outs: um defensor é creditado com um \gls{putout} quando ele, fisicamente, consegue eliminar um jogador do time adversário (quer seja por
 tocá-lo com a bola ou pisando uma base com a posse da bola em jogadas forçadas, ou até mesmo catando um terceiro strike). Também usado em
 estatísticas de jogo.

 R -- Run (Corridas ou Pontos): ocorre quando o corredor consegue cruzar a base principal (home) e marcar um ponto.

 RBI -- Run Batted In: é creditado estatísticamente a um batedor na maioria dos casos nos quais ele vai aparecer no bater box pra bater e pelo menos um ponto é conseguido. Existem algumas exceções: por exemplo, ele não ganha um RBI quando o ponto for resultado de um erro de um defensor.

 Rise: é uma bola de efeito na qual o arremessador da um efeito de giro na bola fazendo com que a mesma "suba" na hora que o batedor iria efetuar a batida, provocando o erro dele.

% Runner (Corredor/Lana): nomenclatura usada para denominar o rebatedor que, depois de bater na bola, chega salvo a uma das bases (seja ela a primeira, a segunda ou a terceira). Sua nova função consiste em correr para alcançar o maior número possível de bases enquanto o seu time estiver no ataque (rebatendo a bola).


 Sado: é o termo japonês para \gls{third}. Geralmente designa o defensor que joga na posição F5 ou na terceira base.

 Sanchin: é o termo japonês para \gls{strikeout}, ou o ato de um batedor ser eliminado pelo terceiro strike virado no qual não acerta a bola.

 SB -- Stolen Base: número de vezes que um jogador consegue roubar uma base.

 SBA -- Stolen Bases Allowed (Pitcher and Catcher): número de vezes que um jogador conseguiu "avançar bases" sem ser eliminado e sem que houvesse uma
 jogada ocorrendo (bases roubadas).

 Secano: é o termo japonês para \gls{second}. Geralmente designa o defensor que joga na posição F4 ou na segunda base.

 SF -- Sacrifice Fly: ocorre quando um batedor acerta uma batida com bola aérea para fora do campo interno, que pode ser facilmente pega por um defensor, mas que permite um corredor marcar ponto (após retocar a base depois da catada).

 SH -- \gls{foultip}: ocorre quando um jogador consegue marcar pontos por uma batida no campo interno ("Bunt") feita por um batedor com este intuito.

 SHO -- \gls{shutout}: um arremessador é premiado com um \textit{Shutout} quando ele entra para arremessar e arremessa o jogo inteiro pelo seu time sem permitir que
 adversário consiga marcar nenhum ponto.

 Shotto: é o termo japonês para \gls{shortstop}. Geralmente designa o defensor que joga na posição F6 ou na interbases (entre a segunda e terceiras bases).

 SO -- Strike Out: ocorre quando o arremessador consegue eliminar o batedor por uma combinação de 3 viradas de bastão ou 3 strikes determinados pelo árbitro de \gls{home}.

 SO -- Strike Out: representa a quantidade de vezes que o batedor foi eliminado pelo 3 strike (quer seja apenas olhando ou virando o bastão).

 Southpaw: é um arremessador canhoto.

 Squeeze Play: é uma manobra que consiste em um sacrifício com um corredor na terceira base. O batedor bate na bola, esperando ser eliminado na primeira base, mas proporcionando ao corredor na terceira base uma oportunidade para marcar um ponto.

 Strike: o mesmo que "bola boa". É o arremesso válido feito pela defesa, que não é rebatido pelo ataque.

 Suberi: é o termo japonês para \gls{slide}, é quando o corredor tenta entrar na base escorregando com a perna esticada e tentando se esquivar de um possível toque do defensor da base.

 TB -- \glspl{totalbase}: quantidade de bases conquistadas por um batedor através de suas batidas.

 Terreno \gls{fair}: é definido como a área do campo de jogo entre as duas linhas laterais que definem o campo de jogo, e inclui as próprias linhas e os postes delimitadores.

 Terreno \gls{foul}: é definido como qualquer área fora do campo de jogo.

 Triple Play: é o ato raro de fazer três eliminações durante durante a mesma jogada contínua. Um \gls{foultip} agarrado pelo catcher é considerado um 3o strike, portanto conta como\gls{strikeout}.



 Zona de Strike: para que um arremesso seja considerado válido, a bola precisa se manter na chamada zona de strike, um retângulo imaginário de mais ou menos 35 centímetros de largura e cuja altura se mede do joelho até axilas do rebatedor. A bola arremessada fora desta área é considerada "ruim". A análise dos arremessos é feita por um juiz que fica posicionado atrás da receptora do time que está defendendo (e do rebatedor).
